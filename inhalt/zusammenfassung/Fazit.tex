\section{Fazit}

Nachdem im ersten Kapitel auf die Beweggründe dieser Arbeit eingegangen wurde, folgten die Erklärung der benötigten Grundlagen und eine nähere Beschreibung der vier betrachteten clientseitigen Webframeworks. Anschließend wurde näher auf den Ist-Zustand eingegangen, woraufhin die Vorstellung der Lösungsansätze folgte. \\
Die fünf vorgestellten Lösungsansätze \quotes{Clientlib}, \quotes{Content Ordner}, \quotes{Java-Servlet}, \quotes{JavaScript} und \quotes{Proxy} ermöglichen es, unter gewissen Umständen das gesetzte Ziel zu erreichen. Jeder Lösungsansatz besitzt, damit dieser praktisch umgesetzt werden kann, bestimmte Anforderungen, die zuvor erfüllt werden müssen. Daher sind manche Lösungsansätze nur unter gewissen Einschränkungen realisierbar. \\
Insgesamt wurden verschiedene Lösungsansätze erarbeitet, die eine Ablage der App-Ressourcen sowohl in die \ac{jcr}-Struktur des \ac{aem} als auch auf einen separaten Webserver erlauben. Somit kann je nach Wunsch die Lösung kompakt erfolgen, in dem nur das \ac{aem} Verwendung findet oder diese durch weitere Schichten erweitert werden. Jeder Ansatz hat seine Vor- und Nachteile. Daher empfiehlt es sich, bei komplexeren Webanwendungen eine Kombination aus mehreren Lösungsansätzen. Somit lassen sich benötigte Anpassungen und Manipulationen an der Webanwendung an mehreren Stellen vornehmen und die Webanwendung strukturiert integrieren. Für einfachere Lösungsansätze hingegen reicht zumeist eine einfache Clientlib. \\
Weiterhin wurde gelöst, wie die Webanwendung korrekt von einer Suchmaschine indiziert werden kann, trotz nachgeladenem Content via JavaScript und einer möglichen Umsetzung als Single Page Application.