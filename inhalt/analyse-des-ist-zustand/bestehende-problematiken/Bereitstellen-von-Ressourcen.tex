\subsection{Bereitstellen von Web-Ressourcen}
Eine weitere Problematik kann das Bereitstellen der Web-Ressourcen der Webanwendung darstellen. Der Grund hierfür ist die mehrstufige Auflösung einer Anfrage des Apache Sling Webframeworks, wie der \autoref{img:sling} zu entnehmen ist.\\
Clientseitige Webanwendungen werden zumeist in lokalen Entwicklungsumgebungen erstellt und anschließend in das bestehende serverseitige System integriert. Bei Webservern wie dem Apache HTTP-Server reicht es häufig, die fertige Webanwendung auf den Webserver in ein entsprechendes Verzeichnis hochzuladen, um diese dem Besucher zur Verfügung zu stellen. Das heißt, die komplette Verzeichnisstruktur kann 1:1 erhalten bleiben. \\
Da bei \ac{aem} jedoch Ressourcen in das \ac{jcr} abgelegt und über Apache Sling freigegeben werden, ist eine exakte Beibehaltung der Struktur nicht ohne Weiteres möglich. \\
Allgemein erfordern Apache Sling und somit auch \ac{aem} ein fundamental anderes Programmierparadigma im Vergleich zu anderen Frameworks für die Erstellung von serverseitigen Webanwendungen. Es wird der Fokus stark auf die Konfiguration gelegt und jede Konvention lässt sich durch entsprechende Einstellungen umgehen.\\
So ergibt es sich, dass die ursprüngliche Verzeichnisstruktur der Webanwendung bei der Integration in das \ac{aem} meistens abgeändert werden muss. Ressourcen werden in Gruppen wie Skripte und Bilder eingeteilt und unter verschiedenen Knoten des \ac{jcr} abgelegt. Somit verschiebt sich die Verzeichnisstruktur und die relativen Pfade untereinander verändern sich.