\subsectiontbd{Konflikte mit anderen Skripten}

Innerhalb der Autoren-Bedienoberfläche wird die Seite so dargestellt, wie diese auch bei dem Besucher der Webseite in seinem Browser erscheinen würde. Im Hintergrund lädt \ac{aem} weitere JavaScript-Dateien, welche es den Autoren erlauben die Seite zu bearbeiten und zu konfigurieren. Es gilt zu überprüfen, ob eine integrierte Webanwendung möglicherweise zu Problemen in der Autoren-Bedienoberfläche führt. Dies beinhaltet zum einen Konflikte zwischen Skripten von \ac{aem} und der Webanwendung und zum anderen auch Darstellungsprobleme, also unterschiedlichen Darstellungen innerhalb der Autoren-Bedienoberfläche und beim Besucher der Webpräsenz. \\
Weiterhin können Versionskonflikte auftreten, falls die Webanwendung mit einer älteren Version des Webframeworks entwickelt wurde, innerhalb von \ac{aem} jedoch eine neuere Version Verwendung finden. Es gilt zu prüfen, ob derartige Konflikte bei den ausgewählten Webframeworks bestehen und auftreten können. Weiterhin bedarf es der Untersuchung, was geschieht wenn innerhalb einer Webseite das gleiche Webframework in unterschiedlichen Versionen, oder mehrere unterschiedliche Webframeworks Verwendung finden und ob dies womöglich zu Kollisionen führt.