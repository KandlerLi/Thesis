\section{Anforderungen}
\label{sec:anforderungen}
Die integrierte Webanwendung, bzw. die \ajc müssen gewisse Anforderungen erfüllen.


\begin{enumerate}[label=A\arabic*:]
	
	\item Die Webanwendung soll als \ajc in die Webseite integriert werden.
	\item Die \ajc soll sich über die Autoren-Bedienoberfläche innerhalb einer Webseite platzieren lassen.
	\item Die \ajc soll sich über die Autoren-Bedienoberfläche, sofern erforderlich, konfigurieren lassen.
	\item Die \ajc soll innerhalb der Autoren-Bedienoberfläche genau wie im Webbrowser des Besuchers dargestellt und bedienbar sein.
	
	\item Die \ajc wird in verschiedenen Formen benötigt, die sich in der Art der Integration unterscheiden.
	\begin{enumerate}[label=A\arabic{enumi}.\arabic*:]
		\item Erstellung einer \ajc als vollständige \ac{aem}-Komponente.
		\item Erstellung einer \ajc und Laden von App-Ressourcen mit Java.
		\item Erstellung einer \ajc und Laden von App-Ressourcen mit JavaScript.
	\end{enumerate}

	\item Die Webanwendung soll vor der Integration optimiert werden.
	\item Die Webanwendung soll von gängigen Suchmaschinen korrekt indiziert werden.
	
	\item Es soll die Möglichkeit geben Anfragen vom Server zu manipulieren.
	\item Es sollen keine Konflikte zwischen Skripten entstehen.
	\item Hyperlinks und Referenzen sollen korrekt sein.
\end{enumerate}

