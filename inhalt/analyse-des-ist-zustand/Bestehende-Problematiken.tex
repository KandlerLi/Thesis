\section{Entwicklungsprozess und bestehende Problematiken}
Die Integration soll vorzugsweise in Form einer \ajc erfolgen, um die Positionierung der Webanwendung innerhalb einer Webseite und deren Konfiguration für Autoren zu vereinfachen. Wird diese in eine Webseite eingebunden, sollen ohne weiteres zutun alle benötigten App-Ressourcen geladen werden und es wird gewährleistet, dass die clientseitige Webanwendung in der Webseite wie vorgesehen dargestellt wird. Der Entwicklungsprozess von einer Webanwendung in eine \ajc ist hierbei wie folgt vorgegeben.

\begin{description}
	\item[1. Entwicklung der Webanwendung] Zunächst wird die Webanwendung innerhalb einer lokalen Entwicklungsumgebung außerhalb einer \ac{aem}-Instanz entwickelt. Als Webserver dient zum Beispiel ein Apache HTTP Server oder ein Node.js Server.
	\item[2. Entwicklung der \ajc] Nachdem die Webanwendung wie gewünscht funktioniert, wird diese an eine \ajc angepasst. Dies geschieht jedoch nicht auf der Zielplattform, sondern innerhalb einer separaten \ac{aem}-Instanz, welche eigens für Entwicklungszwecke läuft.
	\item[3. Auslieferung der \ajc] Sofern die \ajc fertiggestellt ist, wird diese an den Kunden ausgeliefert und schlussendlich von ihm produktiv auf die Zielplattform gestellt.
\end{description}



%Der Entwicklungsprozess sieht hierbei vor, dass zunächst die clientseitige Webanwendung unter Ausschluss einer \ac{aem}-Instanz entwickelt wird. Erst nach Fertigstellung wird diese als \ac{aem}-Komponente angepasst und ggf. erweitert, um beispielsweise die Konfigurierbarkeit innerhalb der Autoren-Bedienoberfläche zu ermöglichen. Abschließend erfolgt die Integration in die Zielplattform.\\
Beim zweiten und dritten Schritt, also der Transformation einer Webanwendung in eine \ajc und den Versuch der Integration können allerdings verschiedene Problematiken auftreten. Diese entstammen den Eigenschaften von \ac{aem} und den Webframeworks, resultieren aber auch durch gesetzte Rahmenbedingen und Restriktionen, welche die IT-Landschaft der Zielplattform und Integration betreffen. 

\subsection{IT-Landschaft der Zielplattform}

Zumeist weisen die unterschiedlichen Zielplattformen, also jene Plattformen, auf denen eine \ac{aem}-Instanz läuft und schlussendlich eine Webanwendung integriert wird, untereinander abweichende Konfigurationen auf. Dies kann das Bereitstellen der Webanwendungen beträchtlich erschweren. Unterschiede in der Konfiguration wären zum Beispiel, dass bei manchen Zielplattformen aus Sicherheitsgründen gewisse Dienste, beispielsweise die \acs{webdav}-Schnittstelle, deaktiviert sind. Eine weitere mögliche Rahmenbedingung kann sein, dass die App-Ressourcen der Webanwendung sich nicht im \ac{aem}, sondern auf einem separaten Webserver befinden sollen. \\
Somit ergeben sich, abhängig von den Rahmenbedingen und der Konfiguration, unterschiedliche Herangehensweisen und Lösungen für die gestellte Aufgabe. Manche Lösungen sind hierbei für gewisse Zielplattformen besser oder schlechter geeignet oder wegen der gesetzten Rahmenbedingungen als mögliche Lösung gar ausgeschlossen.
\subsectiontbd{Bereitstellen von Web-Ressourcen}
Ein Hindernis ist das Bereitstellen der Web-Ressourcen der Webanwendung. Der Grund hierfür ist die mehrstufige Auflösung einer Anfrage des Apache Sling Webframeworks, wie der \autoref{img:sling} zu entnehmen ist.\\
Clientseitige Webanwendungen werden zumeist in lokalen Entwicklungsumgebungen erstellt und anschließend in das bestehende serverseitige System integriert. Bei Webservern wie dem Apache HTTP Server reicht es häufig, die fertige Webanwendung auf den Webserver in ein entsprechendes Verzeichnis hochzuladen, um diese dem Besucher zur Verfügung zu stellen. Das heißt die komplette Verzeichnisstruktur kann 1:1 erhalten bleiben. \\
Da bei \ac{aem} jedoch Server-Ressourcen in das \ac{jcr} abgelegt und über Apache Sling freigegeben werden, ist eine exakte Beibehaltung der Struktur nicht ohne weiteres möglich. \\
Allgemein erfordert Apache Sling, und somit auch \ac{aem}, ein fundamental anderes Programmierparadigma im Vergleich zu anderen Frameworks für die Erstellung von serverseitigen Webanwendungen. Es wird sehr der Fokus auf die Konfiguration gelegt und jede Konvention lässt sich durch entsprechende Einstellungen umgehen.\\
So ergibt es sich, dass die ursprüngliche Verzeichnisstruktur der Webanwendung bei der Integration abgeändert wird. Ressourcen werden in Gruppen wie Skripte und Bilder eingeteilt und unter verschiedenen Knoten des \ac{jcr} abgelegt. Somit verschiebt sich die Verzeichnisstruktur und die relativen Pfade untereinander verändern sich. Zudem ist es das Ziel die Webanwendung in eine \ac{aem}-Komponente einzubetten, was eine zusätzliche Konfiguration voraussetzt. 
\subsectiontbd{Konflikte mit anderen Skripten}

Innerhalb der Autoren-Bedienoberfläche wird die Seite so dargestellt, wie diese auch bei dem Besucher der Webseite in seinem Browser erscheinen würde. Im Hintergrund lädt \ac{aem} weitere JavaScript-Dateien, welche es den Autoren erlauben die Seite zu bearbeiten und zu konfigurieren. Es gilt zu überprüfen, ob eine integrierte Webanwendung möglicherweise zu Problemen in der Autoren-Bedienoberfläche führt. Dies beinhaltet zum einen Konflikte zwischen Skripten von \ac{aem} und der Webanwendung und zum anderen auch Darstellungsprobleme, also unterschiedlichen Darstellungen innerhalb der Autoren-Bedienoberfläche und beim Besucher der Webpräsenz. \\
Weiterhin können Versionskonflikte auftreten, falls die Webanwendung mit einer älteren Version des Webframeworks entwickelt wurde, innerhalb von \ac{aem} jedoch eine neuere Version Verwendung finden. Es gilt zu prüfen, ob derartige Konflikte bei den ausgewählten Webframeworks bestehen und auftreten können. Weiterhin bedarf es der Untersuchung, was geschieht wenn innerhalb einer Webseite das gleiche Webframework in unterschiedlichen Versionen, oder mehrere unterschiedliche Webframeworks Verwendung finden und ob dies womöglich zu Kollisionen führt.
