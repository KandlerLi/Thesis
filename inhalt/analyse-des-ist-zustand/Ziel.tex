\section{Ziel}
Der Gedanke ist hier, die gewünschten Eigenschaften von clientseitigen und serverseitigen Webframeworks zu kombinieren. Statische Inhalte wie Bilder und feste Texte werden über das \ac{aem} gepflegt. Sich häufig ändernde Inhalte, wie Datenbankinhalte oder Informationen von teils externen \ac{rest}-Schnittstellen, lassen sich über besagte clientseitige Webanwendungen zur Laufzeit der HTML-Seite nachladen und anzeigen. Die clientseitige Webanwendung wird den Autoren in Form einer \ac{aem}-Komponente zur Verfügung gestellt, damit diese leicht über die Autoren-Bedienoberfläche zu integrieren ist. Ein \ac{aem}-Komponente, welche primär serverseitige Technologien verwendet, wird hier durch eine clientseitige Webanwendung erweitert. Somit wird aus einer statischen Webseite eine Webseite mit einer eingebetteten clientseitigen Webanwendung, wie in \autoref{img:ws}. Das Ziel ist es, Webanwendungen, die mit clientseitigen Webframeworks entwickelt wurden, in Webseiten, die mit \ac{aem} entwickelt wurden, zu integrieren.\\

\begin{figure}[H]
	\begin{center}
		\includegraphics[width=1\textwidth]{ws.png}
		\caption{Ausgangssituation und Ziel}
		\label{img:ws}
	\end{center}
\end{figure}

Zu beachten ist, dass in der Abbildung und im Folgenden der Begriff \quotes{Webanwendung} kurz für eine \quotes{clientseitige Webanwendung} steht. 
Weiterhin wird an dieser Stelle der Begriff \ajc\index{AJC} definiert. Eine \ajc ist eine \ac{aem}-Komponente, die App-Ressourcen zur Verfügung stellt und damit eine clientseitige Webanwendung abstrahiert.
