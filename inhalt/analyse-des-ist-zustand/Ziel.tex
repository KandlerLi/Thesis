\section{Ziel}
Der Gedanke ist hier, die gewünschten Eigenschaften von clientseitigen und serverseitigen Webframeworks zu kombinieren. Statische Inhalte wie Bilder und Nachrichten werden über das \ac{aem} gepflegt. Sich häufig ändernde Inhalte, wie Datenbankinhalte oder Informationen von teils externen \ac{rest}-Schnittstellen lassen sich über besagte clientseitige Webframeworks zur Laufzeit der HTML-Seite nachladen und anzeigen. Die Webanwendung wird den Autoren in Form einer \ac{aem}-Komponente zur Verfügung gestellt, damit diese leicht über die Autoren-Bedienoberfläche integriert werden kann. Somit wird aus einer statisches Webseite ein Webseite mit einer eingebetteten Webanwendung, wie in \autoref{img:ws}.\\

\begin{figure}[H]
	\begin{center}
		\includegraphics[width=1\textwidth]{ws.png}
		\caption{Ausgangssituation und Ziel}
		\label{img:ws}
	\end{center}
\end{figure}

Zu beachten ist, dass im Folgenden der Begriff \quotes{Webanwendung} kurz für eine \quotes{clientseitige Webanwendung} steht. 
Weiterhin wird an dieser Stelle der Begriff \ajc definiert. Eine \ajc ist eine \ac{aem}-Komponente, die JavaScript beinhaltet und damit eine clientseitige Webanwendung abstrahiert.

%Eine mit \ac{aem} verfasste Webanwendung wird \ac{aem}-Webanwendung genannt.


% Diese \ac{aem}-Komponente entspricht der Webanwendung und lässt sich frei in der Autoren-Bedienoberfläche platzieren. Somit wird aus einer statisches Webseite durch ein Webseite mit einer eingebetteten Webanwendung, wie in \autoref{img:ws}.
%
%\begin{figure}[H]
%	\begin{center}
%		\includegraphics[width=1\textwidth]{ws.png}
%		\caption{Ausgangssituation und Ziel}
%		\label{img:ws}
%	\end{center}
%\end{figure}