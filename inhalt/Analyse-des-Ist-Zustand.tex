\chapter{Analyse des Ist-Zustandes}
\label{sec:ist-zustand}

Durch den Einsatz von \ac{aem} lassen sich bereits umfangreiche Webpräsenzen entwickeln. Dabei besteht jede hiermit erstellte Webseite erfahrungsgemäß aus einen statischen HTML-Rahmen, in dem sich mehreren wiederverwertbaren Komponenten befinden. Im Folgenden wird davon ausgegangen, dass bereits eine Webpräsenz innerhalb einer \ac{aem}-Instanz angesiedelt ist. Es besteht der Wunsch, diese durch weitere \ac{aem}-Komponenten, wie jene in \autoref{sec:komponenten} beschrieben, zu erweitern.
%Dabei wird abgesondert von der Zielplattform innerhalb einer Entwicklungsumgebung die \ac{aem}-Komponente entwickelt. Nach der Fertigstellung erfolgt die Integration der \ac{aem}-Komponente in die \ac{aem}-Instanz der Zielplattform.
%Die \ac{aem}-Komponente kann nun ein Autor innerhalb der Autoren-Bedienoberfläche aufrufen und in eine Webseite integrieren.
\\
Besagte Komponenten basieren jedoch in erster Linie zunächst auf serverseitigen Technologien. Bei der Interaktion mit dem Besucher resultiert dies zu dem Ergebnis, dass neue Inhalte nur serverseitig generiert werden können. Somit wird hier die Webseite komplett neu geladen, wie in \autoref{sec:server-webanwendung} beschrieben wurde. Dieses Verhalten kann sich negativ auf das Nutzererlebnis des Besuchers auswirken. Er ist es gerade von mobilen Geräten wie Tablets und Smartphones gewohnt, dass deren Anwendungen schnell und ohne größere Wartezeiten auf Benutzerinteraktionen reagieren und das gewünschte Resultat anzeigen \cite[S. 78]{Rizvanoglu2013}. Um dies auch in Webanwendungen zu ermöglichen, lassen sich clientseitige Webframeworks verwenden, um so clientseitige Webanwendungen, wie jene in \autoref{sec:client-webanwendung} beschrieben, zu verfassen.\\


\section{Ziel}
Der Gedanke ist hier, die gewünschten Eigenschaften von clientseitigen und serverseitigen Webframeworks zu kombinieren. Statische Inhalte wie Bilder und feste Texte werden über das \ac{aem} gepflegt. Sich häufig ändernde Inhalte, wie Datenbankinhalte oder Informationen von teils externen \ac{rest}-Schnittstellen, lassen sich über besagte clientseitige Webanwendungen zur Laufzeit der HTML-Seite nachladen und anzeigen. Die clientseitige Webanwendung wird den Autoren in Form einer \ac{aem}-Komponente zur Verfügung gestellt, damit diese leicht über die Autoren-Bedienoberfläche zu integrieren ist. Ein \ac{aem}-Komponente, welche primär serverseitige Technologien verwendet, wird hier durch eine clientseitige Webanwendung erweitert. Somit wird aus einer statischen Webseite eine Webseite mit einer eingebetteten clientseitigen Webanwendung, wie in \autoref{img:ws}. Das Ziel ist es, Webanwendungen, die mit clientseitigen Webframeworks entwickelt wurden, in Webseiten, die mit \ac{aem} entwickelt wurden, zu integrieren.\\

\begin{figure}[H]
	\begin{center}
		\includegraphics[width=1\textwidth]{ws.png}
		\caption{Ausgangssituation und Ziel}
		\label{img:ws}
	\end{center}
\end{figure}

Zu beachten ist, dass in der Abbildung und im Folgenden der Begriff \quotes{Webanwendung} kurz für eine \quotes{clientseitige Webanwendung} steht. 
Weiterhin wird an dieser Stelle der Begriff \ajc\index{AJC} definiert. Eine \ajc ist eine \ac{aem}-Komponente, die App-Ressourcen zur Verfügung stellt und damit eine clientseitige Webanwendung abstrahiert.

%Eine mit \ac{aem} verfasste Webanwendung wird \ac{aem}-Webanwendung genannt.


% Diese \ac{aem}-Komponente entspricht der Webanwendung und lässt sich frei in der Autoren-Bedienoberfläche platzieren. Somit wird aus einer statisches Webseite durch ein Webseite mit einer eingebetteten Webanwendung, wie in \autoref{img:ws}.
%
%\begin{figure}[H]
%	\begin{center}
%		\includegraphics[width=1\textwidth]{ws.png}
%		\caption{Ausgangssituation und Ziel}
%		\label{img:ws}
%	\end{center}
%\end{figure}
\section{Entwicklungsprozess und bestehende Problematiken}
Das Ziel ist es, Webanwendungen, die mit clientseitigen Webframeworks entwickelt wurden, in Webseiten, die mit \ac{aem} entwickelt wurden, zu integrieren. Die Integration soll hierbei in Form einer \ajc erfolgen, um die Positionierung der Webanwendung innerhalb einer Webseite und dessen Konfiguration für Autoren zu vereinfachen. Wird diese in einer Webseite eingebunden, werden autonom alle benötigten Web-Ressourcen geladen und es wird gewährleistet, dass die clientseitige Webanwendung in der Webseite wie vorgesehen dargestellt wird. Der Entwicklungsprozess einer \ajc ist hierbei wie folgt vorgeben.

\begin{description}
	\item[1. Enwicklung der Webanwendung] Zunächst wird die Webanwendung innerhalb einer lokalen Entwicklungsumgebung unter Ausschluss einer \ac{aem}-Instanz entwickelt. Als Webserver hier dient zum Beispiel ein Apache HTTP Server oder ein Node.js Server.
	\item[2. Enwicklung der \ajc] Nachdem die Webanwendung wie gewünscht funktioniert, wird diese zu einer \ajc angepasst. Dies geschieht jedoch nicht auf der Zielplattform, sondern innerhalb einer separaten \ac{aem}-Instanz, welche lokal für Entwicklungszwecke läuft.
	\item[3. Auslieferung der \ajc] Sofern die \ajc fertig gestellt ist, wird diese an den Kunden ausgeliefert und schlussendlich von diesem auf die Zielplattform produktiv gestellt.
\end{description}



%Der Entwicklungsprozess sieht hierbei vor, dass zunächst die clientseitige Webanwendung unter Ausschluss einer \ac{aem}-Instanz entwickelt wird. Erst nach Fertigstellung wird diese als \ac{aem}-Komponente angepasst und ggf. erweitert, um beispielsweise die Konfigurierbarkeit innerhalb der Autoren-Bedienoberfläche zu ermöglichen. Abschließend erfolgt die Integration in die Zielplattform.\\
Beim zweiten und dritten Schritt, also der Transformation einer Webanwendung in eine \ajc und dem Versuch der Integration können allerdings verschiedene Problematiken auftreten. Diese entstammen den Eigenschaften von \ac{aem} und den Webframeworks, aber auch durch gesetzte Rahmenbedingen und Restriktionen, welche die IT-Landschaft der Zielplattform und Integration betreffen. 

\subsection{IT-Landschaft der Zielplattform}

Zumeist weisen die unterschiedlichen Zielplattformen, also jene Plattformen, auf denen eine \ac{aem}-Instanz läuft und schlussendlich eine Webanwendung integriert wird, untereinander abweichende Konfigurationen auf. Dies kann das Bereitstellen der Webanwendungen beträchtlich erschweren. Unterschiede in der Konfiguration wären zum Beispiel, dass bei manchen Zielplattformen aus Sicherheitsgründen gewisse Dienste, beispielsweise die \acs{webdav}-Schnittstelle, deaktiviert sind. Eine weitere mögliche Rahmenbedingung kann sein, dass die App-Ressourcen der Webanwendung sich nicht im \ac{aem}, sondern auf einem separaten Webserver befinden sollen. \\
Somit ergeben sich, abhängig von den Rahmenbedingen und der Konfiguration, unterschiedliche Herangehensweisen und Lösungen für die gestellte Aufgabe. Manche Lösungen sind hierbei für gewisse Zielplattformen besser oder schlechter geeignet oder wegen der gesetzten Rahmenbedingungen als mögliche Lösung gar ausgeschlossen.
\subsectiontbd{Bereitstellen von Web-Ressourcen}
Ein Hindernis ist das Bereitstellen der Web-Ressourcen der Webanwendung. Der Grund hierfür ist die mehrstufige Auflösung einer Anfrage des Apache Sling Webframeworks, wie der \autoref{img:sling} zu entnehmen ist.\\
Clientseitige Webanwendungen werden zumeist in lokalen Entwicklungsumgebungen erstellt und anschließend in das bestehende serverseitige System integriert. Bei Webservern wie dem Apache HTTP Server reicht es häufig, die fertige Webanwendung auf den Webserver in ein entsprechendes Verzeichnis hochzuladen, um diese dem Besucher zur Verfügung zu stellen. Das heißt die komplette Verzeichnisstruktur kann 1:1 erhalten bleiben. \\
Da bei \ac{aem} jedoch Server-Ressourcen in das \ac{jcr} abgelegt und über Apache Sling freigegeben werden, ist eine exakte Beibehaltung der Struktur nicht ohne weiteres möglich. \\
Allgemein erfordert Apache Sling, und somit auch \ac{aem}, ein fundamental anderes Programmierparadigma im Vergleich zu anderen Frameworks für die Erstellung von serverseitigen Webanwendungen. Es wird sehr der Fokus auf die Konfiguration gelegt und jede Konvention lässt sich durch entsprechende Einstellungen umgehen.\\
So ergibt es sich, dass die ursprüngliche Verzeichnisstruktur der Webanwendung bei der Integration abgeändert wird. Ressourcen werden in Gruppen wie Skripte und Bilder eingeteilt und unter verschiedenen Knoten des \ac{jcr} abgelegt. Somit verschiebt sich die Verzeichnisstruktur und die relativen Pfade untereinander verändern sich. Zudem ist es das Ziel die Webanwendung in eine \ac{aem}-Komponente einzubetten, was eine zusätzliche Konfiguration voraussetzt. 
\subsectiontbd{Konflikte mit anderen Skripten}

Innerhalb der Autoren-Bedienoberfläche wird die Seite so dargestellt, wie diese auch bei dem Besucher der Webseite in seinem Browser erscheinen würde. Im Hintergrund lädt \ac{aem} weitere JavaScript-Dateien, welche es den Autoren erlauben die Seite zu bearbeiten und zu konfigurieren. Es gilt zu überprüfen, ob eine integrierte Webanwendung möglicherweise zu Problemen in der Autoren-Bedienoberfläche führt. Dies beinhaltet zum einen Konflikte zwischen Skripten von \ac{aem} und der Webanwendung und zum anderen auch Darstellungsprobleme, also unterschiedlichen Darstellungen innerhalb der Autoren-Bedienoberfläche und beim Besucher der Webpräsenz. \\
Weiterhin können Versionskonflikte auftreten, falls die Webanwendung mit einer älteren Version des Webframeworks entwickelt wurde, innerhalb von \ac{aem} jedoch eine neuere Version Verwendung finden. Es gilt zu prüfen, ob derartige Konflikte bei den ausgewählten Webframeworks bestehen und auftreten können. Weiterhin bedarf es der Untersuchung, was geschieht wenn innerhalb einer Webseite das gleiche Webframework in unterschiedlichen Versionen, oder mehrere unterschiedliche Webframeworks Verwendung finden und ob dies womöglich zu Kollisionen führt.

\section{Anforderungen}
\label{sec:anforderungen}
Die integrierte Webanwendung, bzw. die \ajc müssen gewisse Anforderungen erfüllen.


\begin{enumerate}[label=A\arabic*:]
	
	\item Die Webanwendung soll als \ajc in die Webseite integriert werden.
	\item Die \ajc soll sich über die Autoren-Bedienoberfläche innerhalb einer Webseite platzieren lassen.
	\item Die \ajc soll sich über die Autoren-Bedienoberfläche, sofern erforderlich, konfigurieren lassen.
	\item Die \ajc soll innerhalb der Autoren-Bedienoberfläche genau wie im Webbrowser des Besuchers dargestellt und bedienbar sein.
	
	\item Die \ajc wird in verschiedenen Formen benötigt, die sich in der Art der Integration unterscheiden.
	\begin{enumerate}[label=A\arabic{enumi}.\arabic*:]
		\item Erstellung einer \ajc als vollständige \ac{aem}-Komponente.
		\item Erstellung einer \ajc und Laden von App-Ressourcen mit Java.
		\item Erstellung einer \ajc und Laden von App-Ressourcen mit JavaScript.
	\end{enumerate}

	\item Die Webanwendung soll vor der Integration optimiert werden.
	\item Die Webanwendung soll von gängigen Suchmaschinen korrekt indiziert werden.
	
	\item Es soll die Möglichkeit geben Anfragen vom Server zu manipulieren.
	\item Es sollen keine Konflikte zwischen Skripten entstehen.
	\item Hyperlinks und Referenzen sollen korrekt sein.
\end{enumerate}


%\section{Möglichkeit der Lösung durch Verwendung eines geeigneten Webframework}
\missingall