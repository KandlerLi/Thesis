\subsubsection{JSX}
Da es sich bei React um eine Bibliothek und kein Framework handelt stellt diese auch keine MV-* Architektur zur Verfügung. Vielmehr dient React mehr dazu die Präsentationsebene, also die View, zu erstellen. Hierfür kommt \ac{jsx} zum Einsatz.\\
Das Erstellen neuer Elemente wird mit der Funktion React.createElement bewerkstelligt. Als Beispiel soll folgender HTML-Code generiert werden.

\begin{lstlisting}[style=htmlcssjs, caption=Zu generierender HTML-Code, label=lst:reacthtml]
<Nav color="blue">
	<Profile>
		click
	</Profile>
</Nav>
\end{lstlisting}

Mit React würde dies mit der Funktion React.createElement geschehen.

\begin{lstlisting}[style=htmlcssjs, caption=Generierung von HTML in React, label=lst:withoutjsx]
var Nav, Profile;
var app = React.createElement(
Nav,
{color:"blue"},
React.createElement(Profile, null, "click")
);
\end{lstlisting}

Das \autoref{lst:withoutjsx} kann auch mit \ac{jsx} geschrieben werden. Diese in React enthaltene Bibliothek wandelt XML-ähnlichen Code in JavaScript um.

\begin{lstlisting}[style=htmlcssjs, caption=Generierung von HTML mit JSX, label=lst:withjsx]
var app = <Nav color="blue"><Profile>click</Profile></Nav>;
\end{lstlisting}

Somit wird durch den Einsatz von \ac{jsx} das \autoref{lst:withjsx} im Hintergrund in \autoref{lst:withoutjsx} umgewandelt.