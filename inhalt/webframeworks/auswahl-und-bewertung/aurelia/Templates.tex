\subsubsection{Templates}
Bei den Templates strebt Aurelia ebenfalls den Versuch an, sich möglichst nahe an die Spezifikationen von ECMAScript zu halten. Innerhalb des Templates werden daher die in ECMAScript 6 eingeführten String-Interpolationen als Platzhalter für das Model eingesetzt. Anstelle von \ac{mvc} wird hier häufig das \ac{mvvm}-Entwurfsmuster verwendet. Schlussfolgernd wird, um Model und View interagieren zu lassen, ein ViewModel benötigt \cite{Inc.2016c}. Mit Aurelia ist es jedoch auch möglich, eine Webanwendung mit dem \ac{mvc}-Entwurfsmuster zu erstellen \cite{Inc.2016b}.