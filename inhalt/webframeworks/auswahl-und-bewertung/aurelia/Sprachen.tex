\subsubsection{Sprachen}
Ähnlich wie bei AngularJS2 erlaubt es auch Aurelia, seine Webanwendungen in den Sprachen TypeScript und JavaScript zu schreiben. Bei JavaScript liegt die Wahl zwischen ECMAScript 6 und ESNext. Mit ESNext ist immer die neueste Version von ECMAScript gemeint, auch wenn diese sich noch in der Spezifikation befindet. Dies ist zum Zeitpunkt dieser Arbeit ECMAScript 7. Da diese zwar bereits spezifiziert, aber von den meisten Browsern noch nicht implementiert wird, siehe \autoref{tab:ecmasupport}, sollte bei Verwendung ebenfalls ein Transpiler, vergleichbar wie bei TypeScript, Einsatz finden, um den Quellcode in ECMAScript 5 oder 3, je nach gewünschtem Browsersupport, umzuwandeln.\\
Im Unterschied zu vielen anderen \ac{spa}-Frameworks kann mit Aurelia ein großer Teil der Webanwendung entwickelt werden, ohne die Aurelia-\ac{api} zu verwenden. Somit wird erreicht, dass im Falle einer Migration zu einem anderen Framework nahezu die gesamte Geschäftslogik der Webanwendung beibehalten werden kann \cite{Inc.2016b}.