\subsubsection{Templates}
Templates dienen bei AngularJS als Vorlage zum Erstellen von Views. Diese bestehen aus HTML und erweiterten dessen Spezifikation um weitere HTML-Tags und Attribute. Hierzu gehören Direktiven, siehe hierfür \autoref{sec:direktiven}, Markups, wie sie zum Beispiel bei der Datenbindung vorkommen, diverse Filter, welche beispielsweise den Text eines Models formatieren, und Elemente zum Kontrollieren von HTML-Formularen. Diese Templates werden durch den HTML-Compiler von AngularJS in reguläres HTML umgewandelt \cite{angularDoc}.