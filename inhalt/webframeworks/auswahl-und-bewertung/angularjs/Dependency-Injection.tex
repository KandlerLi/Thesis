\subsubsection{Dependency-Injection}

Alle Komponenten von AngularJS, also Module, Services etc. besitzen eine gewisse Abhängigkeit. So benötigt ein Controller oft einen Service zum Laden von Web-Ressourcen und Module, um Daten weiter zu verarbeiten. Ein Modul kann wiederum von weiteren Modulen abhängig sein. Zudem benötigt eine View häufig einen Controller. All diese Abhängigkeiten werden bei AngularJS mithilfe von \ac{di} aufgelöst. Die Komponente muss sich hierbei nicht um die Bereitstellung ihrer benötigten Komponenten sorgen. Das entsprechende Subsystem von AngularJS kümmert sich um das Initialisieren der Komponenten, deren Abhängigkeiten und stellt sie anderen Komponenten zur Verfügung. Dank \ac{di} können leicht einzelne Komponenten ausgetauscht werden, somit wird die Testbarkeit vereinfacht \cite{angularDoc}.