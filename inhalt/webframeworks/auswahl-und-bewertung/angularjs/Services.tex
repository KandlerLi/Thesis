\subsubsection{Service}

Ein Service dient zum Erstellen von Modulen und um die Geschäftslogik aus dem Controller auszulagern \cite[S. 333]{ste15}. So wird ein möglichst schlanker Controller erreicht. Zudem übernimmt der Service die mögliche Verwaltung von Zugriffsmethoden \cite[S. 265]{ste15}.\\
In AngularJS kann ein Service mithilfe eines service-, eines factory- und eines provider-Konstrukts erstellt werden. Ein Service, der mittels des service-Konstrukts gebildet wurde, unterscheidet sich von einem Service, der mit dem factory-Konstrukt erstellt wurde, darin, dass seine Logik ausgelagert werden kann. Somit können komplexere Strukturen in Verbindung mit \ac{di} leichter erstellt werden und die Möglichkeit der Modularisierung wird geschaffen. Ein Provider hingegen erzeugt einen oder auch mehrere Services, die sich vor ihrer Initialisierung zunächst konfigurieren lassen \cite[S. 340 f.]{ste15}.