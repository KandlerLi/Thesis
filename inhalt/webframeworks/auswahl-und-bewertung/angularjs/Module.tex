\subsubsection{Module}

Module sind wieder auftretende Codestücke, welche vom Entwickler angelegt werden können. Zudem bietet AngularJS von Haus aus einige Module, wie zum Beispiel einen Controller für das MVC-Muster, verschiedene Filter, um Texte zu formatieren, und noch einige mehr. Die Aufteilung der Codestücke in einzelne Module reduziert die Komplexität und steigert somit auch die Wartbarkeit und vereinfacht das Testen.
%In einem Modul finden sich öfters verwendete Codestücke. Zum Anlegen eines Modules gibt es von AngularJS Funktionen, um dies zu bewerkstelligen. \autoref{lst:angularjs_module} zeigt, wie dies mit einer Zeile JavaScript aussehen kann.
%
%\begin{lstlisting}[style=htmlcssjs, caption=Anlegen eines Modul in AngularJS, label=lst:angularjs_module]
%var module = angular.module('myModule', []);
%\end{lstlisting}
%
%\quotes{myModule} ist hierbei der Name des Modules. Dieser sollte innerhalb der Webanwendung eindeutig sein, um Namenskonflikte zu vermeiden. In die geschweiften Klammern kommen die Namen weiterer Module, sofern diese vom eigenen Modul benötigt werden. 