\subsubsection{Komponenten}

Unter AngularJS2 versteht man eine Komponente für einen für den Besucher sichtbaren Teil, der innerhalb der Webanwendung wiederverwendet werden kann \cite[S. 67]{Rangle.io2016}. Die Verschachtelung von Komponenten und der Datenaustausch untereinander ist möglich \cite[S. 71]{Rangle.io2016}. Eine Komponente entspricht in etwa den in AngularJS verwendeten Direktiven und Controllern \cite[S. 16]{Rangle.io2016}.\\
Die AngularJS2-Komponenten sind an Webkomponenten angelehnt. Eine Webkomponente ist ein Konzept aus der Webtechnologie, das vier Technologien vereint \cite[S. 3]{Deeleman2016}. Dies sind folgende:

\begin{description}
	\item[Templates:] Bereits aus früheren Kapiteln bekannt.
	\item[Custom Elements:] Eigens erstellte HTML-Elemente
	\item[Shadow DOM:] Virtueller DOM, siehe \autoref{sec:virtueller-dom}.
	\item[HTML Imports:] Importieren von HTML-Seiten von anderen HTML-Seiten aus.
\end{description}