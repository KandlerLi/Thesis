\subsubsection{Datenbindung}

In AngularJS2 hat sich der Mechanismus der Datenbindung drastisch geändert. Um Model und View zu synchronisieren, wurde in AngularJS das sogenannte Dirty Checking eingesetzt. Hierfür legte das Framework eine Liste an, in dem jede Eigenschaft eines Models, das in einer View projiziert wird, eingetragen ist. Es wird dann die Liste zyklisch durchlaufen und überprüft, ob sich eine Eigenschaft geändert hat. Dies konnte zur Folge haben, dass ein Model eine Direktive, eine Direktive ein Model, Direktiven andere Direktiven und Models andere Models anstoßen konnten, sich zu aktualisieren. In AngularJS2 wurde diese multidirektionale Aktualisierung abgeschafft. Stattdessen erfolgt diese nur noch in einer Richtung, sodass übergeordnete Komponenten immer vor dessen untergeordneten Komponenten aktualisiert werden. Dies vermeidet auch mögliche zyklische Änderungen, die in AngularJS noch möglich waren \cite[S. 144 ff.]{Rangle.io2016}.
