\newpage
\subsection{Vergleich}

Folgende \autoref{tab:vergleich} zeigt kurz zusammengefasst die wichtigsten Eigenschaft der zuvor genannten Webframeworks im Vergleich. \\
\begin{minipage}{\textwidth}
\begin{longtable}{|c||p{0.15\textwidth}|p{0.15\textwidth}|p{0.15\textwidth}|p{0.15\textwidth}|}
	\hline  
	\backslashbox{\thead{Eig.}}{\thead{Webfr.}}& \thead{AngularJS} & \thead{AngularJS2} & \thead{Aurelia} & \thead{React} \\  \hhline{|=||=|=|=|=|}
	
	\thead{Entwurfsmuster} & MVC, MVVM & MVC, MVVM & MVC, MVVM  & keines \\ 
	\hline 
	\thead{Sprache} & JS & JS/TS & JS/TS & JS \\
	\hline
	\thead{Lizenz} & MIT & MIT & MIT & 3-Klause-BSD \\
	\hline
	\thead{Besonderheiten} & Datenbindung & Verbesserte Laufzeit & ESNext, Support & Shadow DOM\\
	\hline
	\thead{Entwickler} & Google & Google & Blue Spire & Facebook, Twitter \\
	\hline
	\thead{Webpräsenz} & \pseudourl{http://angularjs.org/} & \pseudourl{http://angular.io} & \pseudourl{http://aurelia.io} & \pseudourl{http://facebook.github.io/react/} \\
	\hline

	\caption{Vergleich der vier Frameworks}\label{tab:vergleich}
\end{longtable}
\end{minipage}