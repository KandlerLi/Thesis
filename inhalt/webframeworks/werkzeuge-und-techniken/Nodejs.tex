\subsection{Node.js}
\label{sec:nodejs}\index{Node.js}

Node.js ermöglicht es serverseitige Webanwendungen in JavaScript zu verfassen \cite[S. 4]{Fedosejev2015}. Weiterhin können auch clientseitige Webanwendung hiermit zugänglich gemacht werden. Node.js bietet eine Erweiterung an, welche TypeScript automatisiert in JavaScript umwandelt. Dies beschleunigt die Entwicklung von in TypeScript verfassten Webanwendungen, da eine Umwandlung bei Änderung des Quellcodes nicht manuell angestoßen werden muss.