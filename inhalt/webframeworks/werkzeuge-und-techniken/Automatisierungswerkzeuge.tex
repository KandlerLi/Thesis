\subsection{Automatisierungswerkzeuge}
\label{sec:automatiserungswerkzeuge}
Oft muss eine Webanwendung, sobald diese in ihre produktive Umgebung überführt wird, einigen Anpassungen unterliegen. Hierzu zählen Dinge wie die Minimierung von Quellcode, Anpassen von Pfaden oder Konkatenieren von Dateien. Damit dies nicht der Entwickler händisch übernehmen muss und sich dabei mögliche Flüchtigkeitsfehler einschleichen, existieren speziell für JavaScript als Anwendungen entworfene Werkzeuge für die Automatisierung. Als populäre Beispiele hierfür wären Grunt\index{Grunt} \cite{gruntjs.com} und Gulp\index{Gulp} \cite{gulpjs.com} zu nennen \cite[S. 246 f.]{ste15}.