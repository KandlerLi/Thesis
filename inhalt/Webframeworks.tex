\chapter{Webframeworks}
\label{sec:webframeworks}
Um die Komplexität einer \ac{spa} zu bewältigen, haben sich in den vergangenen Jahren Teams von Entwicklern mit dem Ziel zusammen getan, Webframeworks zu entwickeln, um auf deren Grundlage die Entwicklung künftiger Webanwendungen zu vereinfachen. \\
Folgend soll zunächst auf Werkzeuge und Techniken eingegangen werden, deren Einsatz in allen Webframeworks möglich ist. Anschließend folgt eine Auflistung von Eigenschaften, die ein Webframework mitbringen sollte, damit sich eine Neuentwicklung einer Webanwendung hiermit lohnt. Anschließend wird eine Auswahl an Webframeworks genannt und näher erläutert.

\section{Werkzeuge und Techniken}
Die Entwicklung einer Webanwendung wird durch den richtigen Einsatz entsprechender Werkzeuge vereinfacht und automatisierte Vorgänge werden zum gegebenen Zeitpunkt ausgeführt. Folgend werden einige Werkzeuge samt entsprechenden Beispielprogrammen gelistet.

\subsection{Node.js}
\label{sec:nodejs}\index{Node.js}

Node.js ermöglicht es serverseitige Webanwendungen in JavaScript zu verfassen \cite[S. 4]{Fedosejev2015}. Weiterhin können auch clientseitige Webanwendung hiermit zugänglich gemacht werden. Node.js bietet eine Erweiterung an, welche TypeScript automatisiert in JavaScript umwandelt. Dies beschleunigt die Entwicklung von in TypeScript verfassten Webanwendungen, da eine Umwandlung bei Änderung des Quellcodes nicht manuell angestoßen werden muss.
\subsection{Paketverwaltung}
Für Node.js gibt es ein breites Spektrum an Erweiterungen. Zur Verwaltung dieser Erweiterungen existiert der \ac{npm}, welches derzeit über 330.000 Pakete umfasst \cite{DeBill2016}. Unter anderem wird so die einfache Installation von Werkzeugen für die Entwicklung oder auch das Laden von JavaScript-Bibliotheken und Frameworks ermöglicht. Alternativ kann über \ac{npm} auch die Installation anderer Paketverwaltungs-Programme wie Bower \cite{bower.io} erfolgen. 
\subsection{Automatisierungswerkzeuge}
\label{sec:automatiserungswerkzeuge}
Oft muss eine Webanwendung, sobald diese in ihre produktive Umgebung überführt wird, einigen Anpassungen unterliegen. Hierzu zählen Dinge wie die Minimierung von Quellcode, Anpassen von Pfaden oder Konkatenieren von Dateien. Damit dies nicht der Entwickler händisch übernehmen muss und sich dabei mögliche Flüchtigkeitsfehler einschleichen, existieren speziell für JavaScript als Anwendungen entworfene Werkzeuge für die Automatisierung. Als populäre Beispiele hierfür wären Grunt \cite{gruntjs.com} und Gulp \cite{gulpjs.com} zu nennen \cite[S. 246 f.]{ste15}.
\subsection{Modularisierung und Verwaltung von Abhängigkeiten}
Nicht selten besitzt ein Codestück Abhängigkeiten zu anderen Programmcodes, damit dieses korrekt funktioniert. Dies sind zum Beispiel Bibliotheken, Frameworks oder Codestücke von anderen Entwicklern. Zu gewährleisten, dass alle diese Abhängigkeiten korrekt aufgelöst werden, mag bei kleinen Projekten noch ohne Weiteres vom Entwickler allein zu bewältigen sein. Hierfür muss er alle Skripte in der korrekten Reihenfolge in das HTML-Dokument einbinden. Sobald ein Projekt aber wächst und mehrere Entwickler daran beteiligt sind, steigt auch die Komplexität und somit auch die Herausforderung, besagte Abhängigkeiten aufzulösen. \\
Lösungen hierfür sind Entwurfsmuster wie \ac{di}, \ac{ioc} \cite[S. 25 ff.]{GollDausmann2013} oder das Entwickeln gegen Schnittstellen, wie der \ac{amd} \cite[S. 266 ff.]{ste15}. Das Ziel ist dabei, den Code zu modularisieren, die Entkopplung von Abhängigkeiten und eine komponentenorientierte Entwicklung \cite[S. 261]{ste15}. Für die Zusammenführung der einzelnen Module haben sich gewisse Bibliotheken etabliert. Eine Zusammenfassung der derzeit bedeutendsten Bibliotheken und für welche Art von Modulen diese geeignet sind, ist \autoref{tab:dilist} zu entnehmen.

\begin{minipage}{\textwidth}
\begin{longtable}{|c|c|}
	\hline  
	\thead{Bibliothek} & \thead{Geeignet für}\\  \hhline{|=|=|}
	browserify &  npm-Module \\
	\hline
	RequireJS & ES 2015 Module, \ac{amd}   \\ 
	\hline 
	SystemJS &  ES 2015 Module, \ac{amd}, CommonJS  \\ 
	\hline 
	Webpack & CommonJS, \ac{amd}   \\ 
	\hline 
	
	\caption{Möglichkeiten für Dependency Injection}\label{tab:dilist}
\end{longtable}
\end{minipage}


In jedem Modul wird deklariert, welche anderen Modulabhängigkeiten dieses besitzt. Die Syntax jener Deklaration ist von dem genutzten Entwurfsmuster bzw. der genutzten Schnittstelle abhängig. Die verwendete Bibliothek löst die Abhängigkeiten auf und übergibt den Modulen Instanzen ihrer benötigten Module. 


\section{Anforderungen}
\label{sec:anforderungen_webframeworks}

Bei der Integration der Frameworks gilt es, gewisse Faktoren zu berücksichtigen. Dies sind Anforderungen, die ein Framework erfüllen muss, um einen langfristigen und zufriedenstellenden Einsatz gewährleisten zu können. Zudem müssen bei der Integration der final entwickelten Webanwendung einige Punkte Beachtung finden. \\
Die Vorgaben sind zumeist von extern, also von Kunden, und werden im Folgenden nicht immer erläutert, warum diese so und nicht anders ausgefallen sind. 

\subsection{Browserunterstützung}
Da die Webanwendungen bei möglichst vielen Besuchern lauffähig sein sollen, ist die Browserunterstützung hier ein wichtiger Aspekt. \\ Die Mindestversion der Browser ist \autoref{tab:browsersupport} zu entnehmen.

\begin{minipage}{\textwidth}
\begin{longtable}{|c|c|c|}
	\hline  
	\thead{Browser} & \thead{Mindestversion} & \thead{Veröffentlicht am}\\  \hhline{|=|=|=|}
	
	Chrome & 23 & 25.09.12 \\ 
	\hline 
	Firefox & 16 & 28.08.12 \\ 
	\hline 
	Safari & 6 & 25.07.12\\ 
	\hline 
	Internet Explorer / Edge & 10 & 04.09.12 \\ 
	\hline 
	Android Browser & 4.4 & 09.12.13\\ 
	\hline 
	iOS Safari & 6.1 & 28.01.13 \\ 
	\hline 
	\caption{Browsersupport der Webanwendungen}\label{tab:browsersupport}
\end{longtable}
\end{minipage}

\subsection{Integration}

Die Autoren-Bedienoberfläche von \ac{aem} inkludiert bereits eine Reihe von JavaScript und CSS-Ressourcen. Diese sind vonnöten, um verschiedene Funktionalitäten, wie zum Beispiel die Konfiguration der Komponenten, zu gewährleisten. Es ist somit darauf zu achten, dass Skripte, die vom Framework, von der Webanwendung oder von der Integration der Webanwendung in das \ac{aem} stammen, nicht mit Skripten seitens \ac{aem} kollidieren.

%\subsection{Suchmaschinenoptimierung}
\subsection{Indizierung von Suchmaschinen}
\label{sec:seo}

Das Internet ist eines der sich am schnellsten entwickelnden Medienlandschaften. Viele Unternehmen unterschiedlichster Größenordnung sind hier vertreten, manch eine Firma präsentiert sich sogar nur online \cite[S. 29 f.]{EstherKessler2015}. Da viele Benutzer täglich eine Suchmaschine verwenden oder eine als Startseite ihres Browsers festgelegt haben, sollte es für ein Unternehmen von hoher Wichtigkeit sein, dass seine Webpräsenz mit den richtigen Begriffen bei einer Suche möglichst weit oben erscheint \cite[S. 147]{EstherKessler2015}.\\

\subsubsection{Funktionsweise einer Suchmaschine}
Damit eine Webpräsenz mitsamt ihren Webseiten überhaupt in einem Suchergebnis erscheint, werden diese von der jeweiligen Suchmaschine zunächst indiziert. Dies übernehmen Programme, sogenannte Crawler\index{Crawler} oder auch Robots\index{Robot|see{Crawler}}. Diese durchsuchen das Internet und erstellen einen Katalog aller gefundenen Webseiten. Weitere Webseiten werden über interne und externe Hyperlinks, die sich in den zuvor gefundenen Webseiten befinden, erreicht. Externe Hyperlinks sind eine Möglichkeit, weitere Webpräsenzen zu finden. Auch diese werden wieder indiziert und auf weitere Hyperlinks durchsucht. Solch eine Navigationsstruktur kann wie in \autoref{img:navi-mpa} aussehen.

\begin{figure}[H]
	\begin{center}
		\includegraphics[width=.8\textwidth]{navi-mpa.png}
		\caption{Mögliche Navigationsstruktur einer Webseite}
		\label{img:navi-mpa}
	\end{center}
\end{figure}

Zu sehen ist, dass alle Webseiten von der Einstiegsseite (\pseudourl{index.html}) aus zu erreichen sind. Über diese wird auf die Nachrichtenübersicht (\pseudourl{/news}) und das Impressum (\pseudourl{/abbout}) verwiesen. Die beiden Nachrichteneinträge (\pseudourl{/news/1989/07/01} und \pseudourl{/news/2014/04/01}) sind über die Nachrichtenübersicht erreichbar. Im Impressum ist ein externer Hyperlink (\pseudourl{kandler.li/index.php}) vorzufinden.

Anhand von Schlagwörtern und Algorithmen werden die Webseiten analysiert und in den Katalog der Suchmaschine aufgenommen. Besagter Algorithmus entscheidet, welche Webseite durch welche Suchbegriffe in der Reihenfolge der Suchergebnisse erscheint. Dieser Algorithmus ist in der Regel äußerst komplex und umfasst am Beispiel Google über 200 zu berücksichtigende Faktoren \cite[S.163 f.]{EstherKessler2015}. 

Nun sollte eine Suchmaschine in der Lage sein, eine Webpräsenz samt all ihrer Webseiten zu indizieren. Die Suche nach \quotes{www.example.com Impressum} sollte hier im Idealfall als erstes Ergebnis das Impressum von \pseudourl{example.com} liefern. \\

\subsubsection{Probleme bei Single-Page-Webanwendungen}
Nun ist eine \ac{spa} so gestaltet, dass diese nur aus einer einzigen HTML-Seite besteht und ein Nachladen von Inhalten meist durch JavaScript und \ac{ajax} erfolgt. Wegen dieser Eigenschaften resultieren einige Probleme in der Indizierung.


\paragraph{Nachladen mit JavaScript}

Suchmaschinen analysierten früher zur Indizierung lediglich den Quellcode einer Webseite. Alle JavaScript-Anweisungen wurden ignoriert und nicht wie in einem Webbrowser interpretiert, womit eine \ac{spa} für die Suchmaschine oft ohne bedeutenden Inhalt erschien. Im Jahr 2009 gab Google bekannt, nun auch JavaScript und \ac{ajax} beim Crawling zu berücksichtigen \cite{Google2009}.\\

\paragraph{Hyperlinks innerhalb der SPA}
Problematisch ist aber noch die Navigation zwischen den einzelnen Webseiten. Bei einer traditionellen \ac{mpa} geschieht dies durch Hyperlinks. Über z. B. \pseudourl{http://www.example.com} wäre die Startseite der Webpräsenz zu erreichen, über \pseudourl{http://www.example.com/news} alle Neuigkeiten und hinter \pseudourl{http://www.example.com/about} könnte sich das Impressum verstecken, wie zum Beispiel in \autoref{img:navi-mpa} zu sehen. Jede Unterseite würde von einer Suchmaschine korrekt indiziert werden. Ein Hyperlink innerhalb einer \ac{spa} wurde oft mit einer Sprungmarke \cite{Consortium2016}\index{Sprungmarke}, zu erkennen an der Raute (\#), versehen. Die Neuigkeiten wären nun über \pseudourl{http://www.example.com/index.html\#news} und das Impressum über \pseudourl{http://www.example.com/index.html\#about} zu erreichen. Die Navigationsstruktur aus \autoref{img:navi-mpa} würde nun wie in \autoref{img:navi-spa} aussehen.

\begin{figure}[H]
	\begin{center}
		\includegraphics[width=.8\textwidth]{navi-spa.png}
		\caption{Mögliche Navigationsstreukur in einer SPA}
		\label{img:navi-spa}
	\end{center}
\end{figure}

Alle Seiten einer \ac{spa} wurden von gängigen Suchmaschinen nur als eine Seite, also \pseudourl{http://www.example.com/index.html}, interpretiert. Bis Oktober 2015 empfahl Google, um dieses Problem zu umgehen, die Raute um ein Ausrufezeichen (!) zu ergänzen, womit sich ein Hyperlink der Form \pseudourl{http://www.example.com/\#!about} ergab. Der Crawler ersetzt nun die Zeichenkombination \quotes{\#!}, welche auch Shebang\index{Shebang} genannt wird, durch \quotes{?\_escaped\_fragment\_=}, was zu \ac{url}s wie zum Beispiel \pseudourl{http://www.example.com/?\_escaped\_fragment\_=about} führt. Der Crawler erwartet nun vom Server unter dieser \ac{url} eine reine HTML-Seite ohne JavaScript und mit den gleichen Inhalten, wie ein Besucher dies auch beim Aufruf von \pseudourl{http://www.example.com/\#!about} nach der Ausführung aller JavaScript-Anweisungen erhalten hätte.

\paragraph{Lösung durch die pushState-Funktion}
\label{sec:pushState}\index{pushState-Funktion}
Die Indizierung durch Einsatz von Shebang funktioniert zwar noch, wird aber seit Oktober 2015 von Google offiziell nicht länger empfohlen. Anstelle dieser sollen Techniken und Methoden der progressiven Verbesserung Verwendungen finden, um Webpräsenzen auch zukünftig zuverlässig indizieren zu können. Hierbei kann unter anderem die in HTML5 eingeführte JavaScript-Funktion pushState Verwendung finden \cite{Google2015}. Diese Funktion kann den Browserverlauf manipulieren und diesem einen neuen Eintrag hinzuzufügen, ohne einen neuen HTTP-Request auszulösen. Bei dieser Technik werden Hyperlinks wieder im traditionellen Format angelegt, also z. B. auf \pseudourl{http://www.example.com/about}. Zur Verdeutlichung des Ablaufes soll \autoref{img:pushState} dienen.

\begin{figure}[H]
	\begin{center}
		\includegraphics[width=.5\textwidth]{pushState.png}
		\caption{Ablauf des Webseitenwechsels in SPAs mit Verwendung der pushState-Funktion}
		\label{img:pushState}
	\end{center}
\end{figure}

Navigiert der Besucher nun von der Startseite auf das Impressum (\pseudourl{example.com/about}), wird der HTTP-Request abgefangen. Der benötigte Inhalt für das Impressum wird über \ac{ajax} geladen und in die Seite eingefügt. Im Anschluss wird durch die pushState-Funktion der Browserverlauf manipuliert. Nun ist der letzte Eintrag im Browserverlauf \pseudourl{http://www.example.com/news} und in der Adresszeile wird nun die vom Hyperlink angeforderte \ac{url} angezeigt. Für den Benutzer scheint es so, als ob er die alte HTML-Seite verlassen hätte und sich nun auf einer neuen befindet.\\
Nun muss der Webserver noch so konfiguriert werden, dass alle Anfragen, die mit der Basis-\ac{url} der \ac{spa} beginnen, auf selbige umgeleitet werden, damit ein direkter Aufruf, zum Beispiel von einer externen Seite, funktioniert. Sowohl \pseudourl{http://www.example.com/news} als auch \pseudourl{http://www.example.com/about} würde der Server nun auf \pseudourl{http://www.example.com/} weiterleiten. Die \ac{spa} würde nun den Zusatz der ursprünglich angeforderten \ac{url} \pseudourl{/news} bzw. \pseudourl{/about} erkennen und den entsprechenden Inhalt darstellen. Durch diesen Trick kann nun der Google Crawler jede Webseite einer \ac{spa} durch eine eindeutige \ac{url} erreichen. \\
Es ist somit darauf zu achten, dass das Framework korrekt von gängigen Suchmaschinen interpretiert wird und dass alle Webseiten durch Verwendung gültiger Links indiziert werden.

\subsection{Lizenz}
Da das Framework auch kommerziell genutzt werden soll, ist auch auf dessen Lizenz zu achten. Gängige Lizenzen, welche für eine kommerzielle Nutzung geeignet wären, sind zum Beispiel die 3-Klause-BSD- \cite{3rdclausebsd} und MIT-Lizenz \cite{mit}.
\section{Auswahl \& Bewertung}
\label{sec:auswahl-und-bewertung}

Die in diesem Kapitel genannten Frameworks, also AngularJS, AngularJS2, Aurelia und React, sollen im \autoref{sec:integration-der-webframeworks-in-aem} versucht werden in das \ac{aem} zu integrieren. Zu beachten ist, dass im Folgenden React gelegentlich auch als Framework betitelt wird, auch wenn es sich hier streng genommen um eine Bibliothek handelt. \\

\subsection{AngularJS}
\label{sec:angularjs}\index{AngularJS}
AngularJS ist ein von Google Inc. entwickeltes clientseitiges Webframework. Mit diesem lassen sich Webanwendungen und \ac{spa}s entwickeln. Für den Entwickler ist es zudem möglich verschiedene MV*-Architektur zu realisieren, was durch vorgegebene Richtlinien zum Erstellen von Views, Controller Models erbracht wird. Hierzu gehören sowohl die \ac{mvc}- als auch die \ac{mvvm}-Entwurfsmuster \cite{ste15}.

\subsubsection{Datenbindung}
Eine Eigenschaft von AngularJS ist die automatisierte Datenbindung, um ein Model bzw. View-Model mit seiner View in beide Richtungen zu verknüpfen. Wird das Model geändert, so ändert sich auch die View entsprechend und umgekehrt. Siehe hierfür \autoref{lst:angularjs-databindung}.

\begin{lstlisting}[style=htmlcssjs, caption=Datenbindung bei AngularJS, label=lst:angularjs-databindung]
<html ng-app>
	<head>
		<meta charset="utf8" />
		<title>Datenbindung bei AngularJS</title>
		<script src="angular.js"></script>
	</head>
	<body>
		<form>
			<input placeholder="Benutzername" type="text" ng-model="username">
		</form>
		<p>Hallo, {{username}}</p>
	</body>
</html>
\end{lstlisting}

Um die Basisfunktionalität von AngularJS nutzen zu können, muss lediglich die Datei angular.js eingebunden werden, welche sich zum Beispiel von \url{https://www.angularjs.org} beziehen lässt. \\
AngularJS verwendet eine Palette von Attributen, welche mit \quotes{ng-} beginnen. Diese Attribute werden auch als Direktiven bezeichnet \cite[S. 40]{ste15}. Die Direktive ng-app weist AngularJS darauf hin, dass es sich bei folgender HTML-Seite um eine AngularJS-Webanwendung handelt. Wird nun der Inhalt des mit ng-model versehenen Eingabefeldes verändert, so wird der Platzhalter \{\{username\}\} mit der entsprechenden Eingabe ersetzt.
\subsubsection{Module}

Module sind wieder auftretende Codestücke, welche vom Entwickler angelegt werden können. Zudem bietet AngularJS von Haus aus einige Module, wie zum Beispiel einen Controller für das MVC-Muster, verschiedene Filter, um Texte zu formatieren, und noch einige mehr. Die Aufteilung der Codestücke in einzelne Module reduziert die Komplexität und steigert somit auch die Wartbarkeit und vereinfacht das Testen.
\subsubsection{Service}

Ein Service dient zum Erstellen von Modulen und um die Geschäftslogik aus dem Controller auszulagern \cite[S. 333]{ste15}. So wird ein möglichst schlanker Controller erreicht. Zudem übernimmt der Service die mögliche Verwaltung von Zugriffsmethoden \cite[S. 265]{ste15}.\\
In AngularJS kann ein Service mithilfe eines service-, eines factory- und eines provider-Konstrukts erstellt werden. Ein Service, der mittels des service-Konstrukts gebildet wurde, unterscheidet sich von einem Service, der mit dem factory-Konstrukt erstellt wurde, darin, dass seine Logik ausgelagert werden kann. Somit können komplexere Strukturen in Verbindung mit \ac{di} leichter erstellt werden und die Möglichkeit der Modularisierung wird geschaffen. Ein Provider hingegen erzeugt einen oder auch mehrere Services, die sich vor ihrer Initialisierung zunächst konfigurieren lassen \cite[S. 340 f.]{ste15}.
\subsubsection{Dependency-Injection}

Alle Komponenten von AngularJS, also Module, Services etc. besitzen eine gewisse Abhängigkeit. So benötigt ein Controller oft einen Service zum Laden von Web-Ressourcen und Module, um Daten weiter zu verarbeiten. Ein Modul kann wiederum von weiteren Modulen abhängig sein. Zudem benötigt eine View häufig einen Controller. All diese Abhängigkeiten werden bei AngularJS mithilfe von \ac{di} aufgelöst. Die Komponente muss sich hierbei nicht um die Bereitstellung ihrer benötigten Komponenten sorgen. Das entsprechende Subsystem von AngularJS kümmert sich um das Initialisieren der Komponenten, deren Abhängigkeiten und stellt sie anderen Komponenten zur Verfügung. Dank \ac{di} können leicht einzelne Komponenten ausgetauscht werden, somit wird die Testbarkeit vereinfacht \cite{angularDoc}.
\subsubsection{Templates}
Templates dienen bei AngularJS als Vorlage zum Erstellen von Views. Diese bestehen aus HTML und erweiterten dessen Spezifikation um weitere HTML-Tags und Attribute. Hierzu gehören Direktiven, siehe hierfür \autoref{sec:direktiven}, Markups, wie sie zum Beispiel bei der Datenbindung vorkommen, diverse Filter, welche beispielsweise den Text eines Models formatieren, und Elemente zum Kontrollieren von HTML-Formularen. Diese Templates werden durch den HTML-Compiler von AngularJS in reguläres HTML umgewandelt \cite{angularDoc}.
\subsubsection{Direktiven}
\label{sec:direktiven}

Direktiven kommen in Templates zum Einsatz, um dem HTML-Compiler mitzuteilen, dessen DOM-Elementen ein gewisses Verhalten zu geben \cite{angularDoc}. Zwei dieser Direktiven, und zwar ng-app und ng-model, wurden bereits in \autoref{lst:angularjs-databindung} gezeigt. Neben weiteren Direktiven, welche AngularJS bereits mitbringt, können Entwickler auch eigene erstellen.
\subsubsection{Scopes}
Jedes Modul, jede Direktive und die Anwendung selbst besitzen ihren eigenen Scope, welcher Variablen beherbergt, die nur im jeweiligen Teil der Anwendung zur Verfügung stehen. Der Scope wird genutzt, um Objekte der View bereitzustellen  \cite[S. 37]{ste15}. \\
Innerhalb des Gültigkeitsbereiches kann der bereits vorhandene Scope verwendet oder es können auch untergeordnete Scopes erstellt werden. \cite[S. 52]{ste15}
\subsection{AngularJS2}
\label{sec:angularjs2}\index{AngularJS2}

Zum Zeitpunkt dieser Arbeit ist Google Inc. dabei, den Nachfolger von AngularJS zu entwickeln, genannt AngularJS2. Hierbei handelt es sich um keine Weiterentwicklung von AngularJS, sondern es wird von Grund auf neu entwickelt.

\subsubsection{Unterschied zu AngularJS}

Eine Neuerung ist hier, neben einer anderen Syntax für Skripte und Templates, dass das gesamte Webframework unter anderem auch in TypeScript zur Verfügung gestellt wird. So stehen dem Entwickler einer Webanwendung neue Sprachelemente zur Verfügung. Da gängige Browser nativ kein TypeScript interpretieren können, ist es Voraussetzung, dieses zuerst mithilfe eines Transpilers in JavaScript umzuwandeln. \\
Für bestehende, in AngularJS verfasste Anwendungen, wird eine Anleitung für die Migration in AngularJS2 angeboten. Diese ist jedoch mit Vorsicht zu verwenden, da die Migration von komplexeren Anwendungen sich zum Teil schwieriger gestaltet und nicht immer auf Anhieb funktioniert \cite{Google2016}. \\
Wurden in AngularJS Module noch in einer eigenen Syntax geschrieben, sind diese nun als ECMAScript 6-Module verfasst. Controller und Service sind nun nicht mehr in JavaScript-Funktionen gekapselt, dafür kommen ECMAScript 6-Klassen. Die in AngularJS verwendeten Direktiven und Controller wurde durch Komponenten ersetzt \cite[S. 16]{Rangle.io2016}. \\
Folgende Liste zeigt einen Ausschnitt der wichtigsten Änderungen.

\begin{description}
	\item[Template:] Die Syntax der Templates wurde geändert. So würde die Anweisung für die Datenbindung aus \autoref{lst:angularjs-databindung}, Zeile 9, nicht mehr \textit{ng-model=\grqq username\grqq}, lauten, sondern \textit{[(ngModel)]=\grqq username\grqq}.
		
	\item[Module:] Module in AngularJS2 entsprechen den ECMAScript 2015-Modulen.
	
	\item[Sytlesheets:] Jeder AngularJS2-Komponente können eigene CSS-Dateien zugewiesen werden. Diese sind nur innerhalb der Komponente gültig und haben keinerlei Auswirkungen auf andere Komponenten.
\end{description}

Ein ausführlicher Vergleich beider Frameworks ist in der offiziellen Dokumentation von AngularJS2 zu finden \cite{Google2016b}.

\subsubsection{Komponenten}

Unter AngularJS2 versteht man eine Komponente für einen für den Besucher sichtbaren Teil, der innerhalb der Webanwendung wiederverwendet werden kann \cite[S. 67]{Rangle.io2016}. Die Verschachtelung von Komponenten und der Datenaustausch untereinander ist möglich \cite[S. 71]{Rangle.io2016}. Eine Komponente entspricht in etwa den in AngularJS verwendeten Direktiven und Controllern \cite[S. 16]{Rangle.io2016}.\\
Die AngularJS2-Komponenten sind an Webkomponenten angelehnt. Eine Webkomponente ist ein Konzept aus der Webtechnologie, das vier Technologien vereint \cite[S. 3]{Deeleman2016}. Dies sind folgende:

\begin{description}
	\item[Templates:] Bereits aus früheren Kapiteln bekannt.
	\item[Custom Elements:] Eigens erstellte HTML-Elemente
	\item[Shadow DOM:] Virtueller DOM, siehe \autoref{sec:virtueller-dom}.
	\item[HTML Imports:] Importieren von HTML-Seiten von anderen HTML-Seiten aus.
\end{description}

\subsubsection{Performance}

Das Framework wurde im Hinblick auf mobile Endgeräte wie Tablets und Smartphone entwickelt, um auch hier einen möglichst stabilen und lauffähigen Code zu erlauben. Diese Entscheidung führt zu einer verbesserten Performance im mobilen und im Desktop-Bereich \cite[S. 14]{Rangle.io2016}.

\subsubsection{Datenbindung}

In AngularJS2 hat sich der Mechanismus der Datenbindung drastisch geändert. Um Model und View zu synchronisieren, wurde in AngularJS das sogenannte Dirty Checking eingesetzt. Hierfür legte das Framework eine Liste an, in dem jede Eigenschaft eines Models, das in einer View projiziert wird, eingetragen ist. Es wird dann die Liste zyklisch durchlaufen und überprüft, ob sich eine Eigenschaft geändert hat. Dies konnte zur Folge haben, dass ein Model eine Direktive, eine Direktive ein Model, Direktiven andere Direktiven und Models andere Models anstoßen konnten, sich zu aktualisieren. In AngularJS2 wurde diese multidirektionale Aktualisierung abgeschafft. Stattdessen erfolgt diese nur noch in einer Richtung, sodass übergeordnete Komponenten immer vor dessen untergeordneten Komponenten aktualisiert werden. Dies vermeidet auch mögliche zyklische Änderungen, die in AngularJS noch möglich waren \cite[S. 144 ff.]{Rangle.io2016}.


%\begin{itemize}
%	\item Data-Binding ohne Dirty Checking http://blog.angular-university.io/introduction-to-angular2-the-main-goals/
%\end{itemize}


\subsection{Aurelia}
\label{sec:aurelia}\index{Aurelia}

Hauptverantwortlicher für das Framework Aurelia ist Rob Eisenberg. Dieser entwickelte bereits das JavaScript Framework Durandal und half bei der Entwicklung von AngularJS2 mit. Da dessen Entwicklung Eisenberg jedoch nicht zufrieden stellte, entschloss er sich, das Google-Team zu verlassen und erneut ein neues Framework unter dem Namen Aurelia zu kreieren \cite{Eisenberg2014}.

\subsubsection{Sprachen}
Ähnlich wie bei AngularJS2 erlaubt es auch Aurelia, seine Webanwendungen in den Sprachen TypeScript und JavaScript zu schreiben. Bei JavaScript liegt die Wahl zwischen ECMAScript 6 und ESNext. Mit ESNext ist immer die neueste Version von ECMAScript gemeint, auch wenn diese sich noch in der Spezifikation befindet. Dies ist zum Zeitpunkt dieser Arbeit ECMAScript 7. Da diese zwar bereits spezifiziert, aber von den meisten Browsern noch nicht implementiert wird, siehe \autoref{tab:ecmasupport}, sollte bei Verwendung ebenfalls ein Transpiler, vergleichbar wie bei TypeScript, Einsatz finden, um den Quellcode in ECMAScript 5 oder 3, je nach gewünschtem Browsersupport, umzuwandeln.\\
Im Unterschied zu vielen anderen \ac{spa}-Frameworks kann mit Aurelia ein großer Teil der Webanwendung entwickelt werden, ohne die Aurelia-\ac{api} zu verwenden. Somit wird erreicht, dass im Falle einer Migration zu einem anderen Framework nahezu die gesamte Geschäftslogik der Webanwendung beibehalten werden kann \cite{Inc.2016b}.

\subsubsection{Templates}
Bei den Templates strebt Aurelia ebenfalls den Versuch an, sich möglichst nahe an die Spezifikationen von ECMAScript zu halten. Innerhalb des Templates werden daher die in ECMAScript 6 eingeführten String-Interpolationen als Platzhalter für das Model eingesetzt. Anstelle von \ac{mvc} wird hier häufig das \ac{mvvm}-Entwurfsmuster verwendet. Schlussfolgernd wird, um Model und View interagieren zu lassen, ein ViewModel benötigt \cite{Inc.2016c}. Mit Aurelia ist es jedoch auch möglich, eine Webanwendung mit dem \ac{mvc}-Entwurfsmuster zu erstellen \cite{Inc.2016b}.

\subsubsection{Dependency Injection}
Auch Aurelia unterstützt das Konzept von \ac{di}. Es werden eine Reihe an Bibliotheken und \ac{api}s unterstützt. Innerhalb der Onlinedokumentation liegt der Fokus auf SystemJS, zudem werden auch alle auf \ac{amd}-basierende Lösungen wie RequireJS, Cajo oder Dojo, und Modul-Bundler wie Webpack unterstützt \cite{Inc.2016b}.

\subsubsection{Support}
Aurelia bietet als einziges \ac{spa}-Framework eine kommerzielle Unterstützung an. Google(AngularJS/AngularJS2) und Facebook(React) bieten zwar den Quellcode an, jedoch ohne garantierte Funktionalität. \\
Zudem wird ein zusätzlicher Support angeboten, unter anderem Training, Consulting, Code-Review und weitere sich noch in Entwicklung befindliche Produkte \cite{Inc.2016a}.

\subsection{React}
\label{sec:react}

Einen etwas anderen Ansatz verfolgt React. Es handelt sich bei dem von Facebook und Instagram entwickelten Code nicht um ein Framework, sondern um eine Bibliothek. 

\subsubsection{Virtueller DOM}
\label{sec:virtueller-dom}

React verfolgt einen anderen Ansatz der Datenbindung als die zuvor genannten Frameworks. Änderungen werden hier nicht direkt in den \ac{dom} eingetragen, sondern zunächst in einen virtuellen DOM. Dieser stellt eine abstrakte Kopie zum regulären DOM dar. Sollte ein Element verändert werden, wird diese Änderung zunächst im virtuellen DOM eingetragen. React ermittelt nun den Unterschied zum virtuellen und realen DOM und fügt in Letzteres nur die benötigten Änderungen ein. Dieser Vorgang des Vergleichens und dass nur die benötigten Teile aktualisiert werden, ist in der Regel sehr schnell \cite[S. 19]{Fedosejev2015}.
\subsubsection{JSX}
Da es sich bei React um eine Bibliothek und kein Framework handelt stellt diese auch keine MV-* Architektur zur Verfügung. Vielmehr dient React mehr dazu die Präsentationsebene, also die View, zu erstellen. Hierfür kommt \ac{jsx} zum Einsatz.\\
Das Erstellen neuer Elemente wird mit der Funktion React.createElement bewerkstelligt. Als Beispiel soll folgender HTML-Code generiert werden.

\begin{lstlisting}[style=htmlcssjs, caption=Zu generierender HTML-Code, label=lst:reacthtml]
<Nav color="blue">
	<Profile>
		click
	</Profile>
</Nav>
\end{lstlisting}

Mit React würde dies mit der Funktion React.createElement geschehen.

\begin{lstlisting}[style=htmlcssjs, caption=Generierung von HTML in React, label=lst:withoutjsx]
var Nav, Profile;
var app = React.createElement(
Nav,
{color:"blue"},
React.createElement(Profile, null, "click")
);
\end{lstlisting}

Das \autoref{lst:withoutjsx} kann auch mit \ac{jsx} geschrieben werden. Diese in React enthaltene Bibliothek wandelt XML-ähnlichen Code in JavaScript um.

\begin{lstlisting}[style=htmlcssjs, caption=Generierung von HTML mit JSX, label=lst:withjsx]
var app = <Nav color="blue"><Profile>click</Profile></Nav>;
\end{lstlisting}

Somit wird durch den Einsatz von \ac{jsx} das \autoref{lst:withjsx} im Hintergrund in \autoref{lst:withoutjsx} umgewandelt.


\newpage
\subsection{Vergleich}

Folgende \autoref{tab:vergleich} zeigt kurz zusammengefasst die wichtigsten Eigenschaft der zuvor genannten Webframeworks im Vergleich. \\
\begin{minipage}{\textwidth}
\begin{longtable}{|c||p{0.15\textwidth}|p{0.15\textwidth}|p{0.15\textwidth}|p{0.15\textwidth}|}
	\hline  
	\backslashbox{\thead{Eig.}}{\thead{Webfr.}}& \thead{AngularJS} & \thead{AngularJS2} & \thead{Aurelia} & \thead{React} \\  \hhline{|=||=|=|=|=|}
	
	\thead{Entwurfsmuster} & MVC, MVVM & MVC, MVVM & MVC, MVVM  & keines \\ 
	\hline 
	\thead{Sprache} & JS & JS/TS & JS/TS & JS \\
	\hline
	\thead{Lizenz} & MIT & MIT & MIT & 3-Klause-BSD \\
	\hline
	\thead{Besonderheiten} & Datenbindung & Verbesserte Laufzeit & ESNext, Support & Shadow DOM\\
	\hline
	\thead{Entwickler} & Google & Google & Blue Spire & Facebook, Twitter \\
	\hline
	\thead{Webpräsenz} & \pseudourl{http://angularjs.org/} & \pseudourl{http://angular.io} & \pseudourl{http://aurelia.io} & \pseudourl{http://facebook.github.io/react/} \\
	\hline

	\caption{Vergleich der vier Frameworks}\label{tab:vergleich}
\end{longtable}
\end{minipage}
