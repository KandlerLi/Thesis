\subsection{URL}
\label{sec:url}
Eine \ac{url} identifiziert und lokalisiert eine Ressource. Im der Webtechnologie verweist eine \ac{url} auf eine Web-Ressource. \\
Eine \ac{url} setzt sich aus verschiedenen Teilen, wie das verwendete Protokoll, den Port und die Domain zusammen. Eine \ac{url} kann hierbei wie folgend aussehen.

\begin{lstlisting}[style=jcr, caption=Eine URL, label=lst:url]
http://www.example.com:3000/spa/index.html?page=news
\end{lstlisting}

Die \ac{url} aus \autoref{lst:url} in \autoref{tab:url} erklärt.

\begin{longtable}{|c|c|c|c|c|} 
	
	
	\hline 
	http & www.example.com & 3000 & spa/index.html & page=news\\ 
	
	\hline 
	Schema/Protokoll & Domain & Port & Pfad & Parameter\\
	\hline
	\caption{Aufbau einer URL}\label{tab:url}
\end{longtable}