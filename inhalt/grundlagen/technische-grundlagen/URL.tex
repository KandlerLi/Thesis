\subsection{URL}
\label{sec:url}\index{URL}
Ein \acf{url} identifiziert und lokalisiert eine Ressource. In der Webtechnologie verweist eine \ac{url} auf eine Web-Ressource. \\
Eine \ac{url} setzt sich aus verschiedenen Teilen, wie das verwendete Protokoll\index{Protokoll}, den Port\index{Port} und dem Hostname \index{Hostname} zusammen. Eine \ac{url} kann hierbei wie folgend aussehen.

\begin{lstlisting}[style=jcr, caption=Eine URL, label=lst:url]
http://www.example.com:3000/spa/index.html?page=news
\end{lstlisting}

Die \ac{url} aus \autoref{lst:url} in \autoref{tab:url} erklärt.

\begin{longtable}{|*6{c|}} 
	
	
	\hline 
	http & www & example.com & 3000 & spa/index.html & page=news\\ 
	
	\hline 
	 & Subdomain\index{Subdomain|see{Domain Subdomain}} & Domain\index{Domain} & & & \\
	\hline 
	Schema/Protokoll & \multicolumn{2}{c|}{Hostname} & Port & Pfad & Parameter\\
	\hline
	\caption{Aufbau einer URL}\label{tab:url}
\end{longtable}