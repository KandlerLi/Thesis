\subsection{Ajax}
\label{sec:ajax}\index{Ajax}
Durch die Verwendung von JavaScript lässt sich das Konzept von \ac{ajax} bewerkstelligen. Unter Zuhilfenahme dieses Konzeptes lassen sich neue Inhalte von einem Server nachladen, ohne die bestehende Seite neu zu laden. Dafür wird ein HTTP-Request an den Webserver gestellt, welcher mit den entsprechenden Daten antwortet. Diese erhaltenen Daten können anschließend durch DOM-Manipulation in die Seite eingebettet werden. Dies senkt sowohl die Ladezeit als auch das zu ladende Volumen an Daten. \\
Besagte Daten werden häufig, wie man vom Namen schon ableiten kann, im XML-Format übertragen. Andere Formate, wie z. B. JSON, sind jedoch auch möglich.