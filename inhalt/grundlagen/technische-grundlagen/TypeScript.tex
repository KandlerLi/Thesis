\subsection{TypeScript}
Bei TypeScript handelt es sich um eine von Microsoft entwickelte Programmiersprache. Diese lehnt sich an ECMAScript Version 6 an. TypeScript erweitert deren Funktionsumfang aber um neue Elemente.\\ 
Jedoch unterstützten die Browser aus \autoref{tab:ecmasupport} kein TypeScript und können dieses nicht ausführen. Hierfür wird eine Anwendung benötigt, welche TypeScript in ECMAScript übersetzt. Solch eine Anwendung, genannt Transpiler, stellt unter anderem Microsoft zur Verfügung und übersetzt TypeScript wahlweise in ECMAScript 5 bzw. ECMAScript 3 \cite[S. 439 f.]{ste15}. Ein Transpiler ist ein Programm, das eine Programmiersprache in eine andere übersetzt. So ist einem Entwickler die Möglichkeit gegeben die Vorteile von ECMAScript 6, wie Klassen und Vererbung, zu nutzen. Zudem ist es in TypeScript, im Gegensatz zu JavaScript, möglich, Variablen einem Typ zuzuweisen, womit diese auch stark typisiert genutzt werden können.