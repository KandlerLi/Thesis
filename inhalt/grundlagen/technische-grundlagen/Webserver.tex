\subsection{Webserver}
\label{sec:webserver}\index{Webserver}
Die Entwicklung einer Webpräsenz ist prinzipiell auch ohne einen Webserver denkbar. Wird dieser lokal im Browser geöffnet, würde dies aber bei einem Ajax Aufruf, siehe hierzu \autoref{sec:ajax}, mit bei der Standardeinstellung gewisser Browser zu einem \ac{sop}-Fehler\index{Same Origin Policy} führen \cite[S. 45  f.]{ste15}.\\
Abhilfe schafft hier der Einsatz eines Webservers. Als Plattform stehen neben dem Apache HTTP Server, nginx oder dem Microsoft Webserver ein auf Node.js basierender Webserver zur Auswahl. Dieser wird in \autoref{sec:nodejs} näher erläutert. \\
Zudem wird ein Webserver für den Produktiveinsatz benötigt, um Besuchern den Zugriff mit einem Webbrowser zu erlauben.