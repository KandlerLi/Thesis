\subsubsection{Single-Page-Webanwendung}
\label{sec:spa}

Durch den Einsatz von \ac{ajax} und \ac{dom} lässt sich auch eine \acf{spa} entwickeln. Diese werden im Deutschen auch Single-Page-Webanwendung genannt. Hierbei besteht eine Webpräsenz lediglich aus einer HTML-Seite, man täuscht dem Besucher aber bei der Navigation mehrere Webseiten vor \cite[S. 32]{ste15}. Dies wird erreicht, indem der Inhalt der neu anzuzeigenden Webseite durch Ajax geladen und durch DOM-Manipulation in die Seite integriert wird. Der alte Inhalt wird ausgeblendet und ist später, ohne dass eine weitere Ajax-Anfrage nötig ist, wieder eingeblendet. Durch Einsatz dieser Technik müssen auf den einzelnen Webseiten wiederkehrende Elemente, wie zum Beispiel Navigation, Kopf- und Fußzeilen etc. nur einmal zu Beginn geladen werden. Die Inhalte der bereits besuchten Seiten liegen weiterhin ausgeblendet beim Clienten vor und können relativ zügig durch DOM-Manipulation wieder eingeblendet werden. Analog zum Begriff Single Page Application werden klassische Webanwendungen, bei denen jeder Webseite eine HTML-Seite zugewiesen wird, auch \ac{mpa} genannt. Zu beachten ist, dass der Begriff \ac{mpa} in der Literatur nicht üblich ist und nur vereinzelt Verwendung findet. 