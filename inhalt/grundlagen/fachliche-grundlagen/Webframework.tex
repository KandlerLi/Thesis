\subsection{Framework / Bibliothek}
Eine Bibliothek ist eine Codesammlung, mir der sich wiederkehrende Codestrukturen abbilden lassen, um einen kürzeren und stabileren Quellcode zu verfassen. Ein Framework liefert zudem noch ein  Programmiergerüst, welches vorgefertigte Funktionalitäten mitbringt. Dieses Gerüst liefert den Rahmen (Frame), innerhalb dessen sich Anwendungen erstellen lassen. Ein Framework legt im Gegensatz zu einer Bibliothek auch eine Steuerung von Verhaltensweisen bei der Verwendung fest \cite[S. 20]{steyer2011jquery}.  Eine Art solcher Frameworks sind Webframeworks. Mit diesen lassen sich Webanwendungen entwickeln. In Rahmen dieser Arbeit sind z. B. die Webframeworks AngularJS, AngularJS2, React und Aurelia, siehe \autoref{sec:auswahl-und-bewertung}, relevant.