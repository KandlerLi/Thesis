\paragraph{HTL}
\label{sec:htl}
Auch AEM verwendet Templates für seine Seiten. Als Model dienen Daten, welche über die \ac{jcr}-\ac{api} gewonnen werden. Als Template-Sprache kann man zwischen jener von \ac{jsp} oder auch \ac{htl} wählen. Letztere wurde von Adobe entwickelt und in \ac{aem} mit der Version 6.0 ausgeliefert und wird von Adobe als bevorzugte Templatesprache genannt \cite{Adobe2016}. \\
\ac{htl} ist an \ac{html} angelehnt, somit ist auch jedes in \ac{htl} verfasste Template zugleich gültiges HTML5 \cite[S. 5.11 f.]{Incorporated2015}. Mit dem folgenden \autoref{lst:htl} würden vom aktuellen \ac{jcr}-Knoten alle Kinder in einer HTML-Liste dargestellt und jeder Listeneintrag abwechselnd mit der HTML-Klasse \quotes{odd} bzw. \quotes{even} versehen werden.


\begin{lstlisting}[style=htmlcssjs, caption=Ein HTL Beispiel, label=lst:htl]
<ul data-sly-list.child="${currentPage.listChildren}">
	<li class="${ childList.odd ? 'odd' : 'even'}">${child.title}</li>
</ul>
\end{lstlisting}