\begin{longtable}{| p{.20\textwidth} | p{.80\textwidth} |} 
\hline 
	\thead{Begriff} & \thead{Definition} \\ 
\hline 

(Web-)Server & Ein Server ist ein Programm auf einem Rechner bzw. ein Zusammenschluss von Rechnern, der verschiedene Dienste und Ressourcen zur Verfügung stellt. Im Folgenden wird ein Server, sofern nicht anders beschrieben, als Synonym für einen Webserver verstanden. Ein Webserver dient in erster Linie dazu, für seine Clients eine Webpräsenz zur Verfügung zu stellen. Siehe auch \autoref{sec:webserver}.
\\ 

\hline 
(Web-)Client & Ein Client kommuniziert mit einem Server und kann durch ein entsprechendes Protokoll dessen Dienste verwenden und seine Ressourcen abrufen. Ein Webclient ist ein Client, der auf die Ressourcen eines Webservers zugreift und diese in einem Webbrowser darstellt. Im Folgenden wird ein Client, sofern nicht anders beschrieben, als Synonym für einen Webclienten verstanden.
\\ 

\hline 
Ressource & Ressourcen sind z. B. Webseiten und Bilder, welche über eine eindeutige Adresse, wie z. B. \pseudourl{example.com/index.html} oder \pseudourl{example.com/bilder/flowerpot.png}, erreichbar sind.
\\  

\hline
HTML-Dokument & \missingtext%Eine HTML-Seite ist eine Ressource, deren Quellcode aus HTML besteht und die von einem Webbrowser interpretiert und dargestellt wird.
\\
\hline

\caption{Definitionen}\label{tab:definitionen}
\end{longtable}