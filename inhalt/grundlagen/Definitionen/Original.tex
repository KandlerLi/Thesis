Zunächst soll auf einige Begriffsdefinitionen eingegangen werden.
%\begin{table}[H]
%\begin{tabularx}{\textwidth}{|l|X|}
\begin{longtable}{| p{.20\textwidth} | p{.80\textwidth} |} 
	\hline 
	\thead{Begriff} & \thead{Definition} \\ 
	\hline 
	Webbrowser & Ein Webbrowser ist ein Computerprogramm, das Ressourcen einer Webpräsenz, wie z. B. Webseiten und Bilder, aufrufen, darstellen und ggf. interpretieren kann.
	\\ 
\hline 
Webpräsenz & Eine Webpräsenz ist der Zusammenschluss von Dateien und Ressourcen, zwischen denen sich durch Verwendung eines Webbrowsers navigieren lässt \cite[S. 30]{JacobsenGidda2016}. Für die Pflege des Inhalts sind zumeist eine oder mehrere Personen verantwortlich. \\ 

\hline 
Webseite & Unter einer Webseite versteht man eine einzige Seite einer Webpräsenz, die in einem Webbrowser dargestellt werden kann \cite[S. 30]{JacobsenGidda2016}. Zwischen den einzelnen Webseiten kann navigiert werden.
\\ 

\hline 
Webanwendung & Eine Webanwendung ist ein Verbund von Webseiten, der, abhängig von gewissen Faktoren, anders dargestellt wird. Siehe hierzu \autoref{sec:webanwendung}. Eine Sonderform hiervon ist eine \ac{spa}, welche im \autoref{sec:spa} erklärt wird.
\\ 


\hline
HTML-Seite & Eine HTML-Seite ist eine Ressource, deren Quellcode aus HTML besteht und die von einem Webbrowser interpretiert und dargestellt wird.
\\

\hline 
(Web-)Server & Ein Server ist ein Programm auf einem Rechner bzw. ein Zusammenschluss von Rechnern, der verschiedene Dienste und Ressourcen zur Verfügung stellt. Im Folgenden wird ein Server, sofern nicht anders beschrieben, als Synonym für einen Webserver verstanden. Ein Webserver dient in erster Linie dazu, für seine Clients eine Webpräsenz zur Verfügung zu stellen. Siehe auch \autoref{sec:webserver}.
\\ 

\hline 
(Web-)Client & Ein Client kommuniziert mit einem Server und kann durch ein entsprechendes Protokoll dessen Dienste verwenden und seine Ressourcen abrufen. Ein Webclient ist ein Client, der auf die Ressourcen eines Webservers zugreift und diese in einem Webbrowser darstellt. Im Folgenden wird ein Client, sofern nicht anders beschrieben, als Synonym für einen Webclienten verstanden.
\\ 
\hline
Besucher & Als Besucher versteht man im Folgenden eine Person, die eine Webpräsenz und dessen Webseiten besucht. Als Synonym kann auch der Begriff \quotes{Anwender} Verwendung finden.
\\
\hline 
Entwickler & Hier ist aus dem Kontext heraus zu lassen, ob ein Entwickler eines Frameworks/einer Bibliothek gemeint ist oder ein Entwickler das besagte Framework/ die besagte Bibliothek nutzt. Hier sind mit Entwicklern meist eine oder mehrere Personen gemeint, die eine Webanwendung für einen Kunden entwickeln.
\\
\hline 
Kunde & Der Kunde ist der Auftraggeber für eine Webanwendung/eine SPA. Dessen IT-Landschaft unterscheidet sich zumeist von jener, in der ein Entwickler arbeitet. 
\\
\hline 
%\end{tabularx} 
\caption{Definitionen}\label{tab:definitionen}
%\end{table}
\end{longtable}