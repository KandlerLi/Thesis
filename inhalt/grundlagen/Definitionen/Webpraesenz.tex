\begin{longtable}{| p{.20\textwidth} | p{.80\textwidth} |} 
\hline 
	\thead{Begriff} & \thead{Definition} \\ 

\hline 
Webpräsenz & Eine Webpräsenz ist der Zusammenschluss Ressourcen. Diese wird zumeist unter einer Domain, wie zum Beispiel \pseudourl{example.com} betrieben \cite[S. 29 f.]{JacobsenGidda2016}. Für die Pflege des Inhalts sind zumeist eine oder mehrere Personen verantwortlich. \\ 

\hline 
Webseite & Unter einer Webseite versteht man eine einzige Seite einer Webpräsenz, die in einem Webbrowser dargestellt werden kann \cite[S. 30]{JacobsenGidda2016}. Zwischen den einzelnen Webseiten kann navigiert werden. Jede Webseite ist unter einer eigenen, eindeutigen Adresse aufrufbar.
\\ 

\hline 
Webinhalt & 
\\ 
\hline 

\caption{Definitionen für eine Webpräsenz}\label{tab:definitionen-webpraesenz}
\end{longtable}