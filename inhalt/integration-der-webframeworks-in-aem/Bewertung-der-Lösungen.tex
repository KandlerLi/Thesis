\sectiontbd{Bewertung der Lösungen}
In den vorherigen Abschnitten wurden verschiedene Lösungsansätze erarbeitet. Nachfolgende \autoref{img:entscheidung} dient als kleine Hilfe, welche Lösung für eine vorstehende Integration am geeignetsten wäre. \\
\begin{figure}[H]
	\begin{center}
		{\footnotesize\includesvg[width=1\textwidth]{entscheidung}}
		\caption{Entscheidungsbaum für eine adäquate Lösung}
		\label{img:entscheidung}
	\end{center}
\end{figure}
\missingall


Folgende \autoref{tab:erfuellung} listet erneut die Kürzel der in \autoref{sec:anforderungen} gestellten Anforderung und vergleicht, in welchem Kapitel dieser Arbeit diese erfüllt bzw. gelöst wurden.

\begin{longtable}{| c | c|} 
	\hline 
	\thead{Anforderung} & \thead{Gelöst in Kapitel} \\ 
	
	\hline 
	A1 &  \ref{sec:integrationen} \\ 
	\hline
	A2 &  \ref{sec:integrationen} \\  
	\hline
	A3 &  - \\  
	\hline 
	A4 &  \ref{sec:integrationen} \\   
	\hline
	A5 &  \ref{sec:integrationen} \\   
	\hline
	A5.1 &  \ref{sec:direkte-integration-in-das-aem}\\  
	\hline 
	A5.2 &  \ref{sec:ansatz-java} \\  
	\hline 
	A5.3 &  \ref{sec:ansatz-javascript}\\  
	\hline 
	A6 &  \ref{sec:aufbereiten} \\   
	\hline
	A7 &  \ref{sec:suchmaschinen} \\   
	\hline
	A8 &  \ref{sec:proxy}\\ 
	\hline 
	A9 & \ref{sec:konflikte}\\
	\hline
	\caption{Anforderungserfüllung}\label{tab:erfuellung}
\end{longtable}
