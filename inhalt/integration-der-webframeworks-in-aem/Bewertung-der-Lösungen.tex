\section{Bewertung der Lösungen}
In den vorherigen Abschnitten wurden verschiedene Lösungsansätze erarbeitet. Nachfolgende \autoref{img:entscheidung} dient als kleine Entscheidungshilfe, welche Lösung für eine vorstehende Integration am geeignetsten wäre. \\
\begin{figure}[H]
	\begin{center}
		{\footnotesize\includesvg[width=1\textwidth]{entscheidung}}
		\caption{Entscheidungsbaum für eine adäquate Lösung}
		\label{img:entscheidung}
	\end{center}
\end{figure}

Des Weiteren soll folgende \autoref{tab:vergleich_loesungen}  jeweils in einem Satz zusammenfassen, welcher Lösungsansatz bzw. welche Kombination aus Lösungsansätzen in welchen Situationen geeignet bzw. ungeeignet ist.

\begin{longtable}{|p{0.2\textwidth}|p{0.35\textwidth}|p{0.35\textwidth}|} 
	\hline 
	\thead{Lösungsansatz} & \thead{Gut wenn \dots} & \thead{Schlecht wenn \dots}\\ 
	\hhline{|=|=|=|}
	
	Clientlib
	&
	\dots es eine kleine Webanwendungen mit fast ausschließlich JavaScript und CSS ist.
	&
	\dots es sich um eine komplexe \ac{spa} handelt oder der AoT-Ansatz Verwendung findet.
	\\\hline 
	
	Content-Ordner
	& 
	\dots die Webanwendung eine komplette Webseite darstellt.
	&
	\dots als \ac{aem}-Komponente benötigt wird.
	\\\hline
	
	Java-Servlet
	& 
	\dots Webanwendung auf externen Server liegt.
	&
	\dots Viele unterschiedliche Webanwendung zu integrieren sind.
	\\\hline
	
	JavaScript
	&
	\dots hauptsächlich clientseitige Techniken für die Umsetzung gewünscht sind.
	&
	\dots externer Webserver nicht allen Anfragen einen entsprechenden Header hinzufügt (CORS).
	\\\hline
	
	Proxy
	&
	\dots Logik für Integration außerhalb von \ac{aem} und nicht zwingend mit Java bzw. JavaScript erfolgen soll.
	&
	\dots direkter Zugriff für Besucher verboten sein soll.
	\\\hline
	
	Clientlib + Content
	&
	\dots es eine kleine Webanwendungen mit weiteren App-Ressourcen ist.
	&
	\dots die Webanwendung auf mehreren Webseiten verwendet wird.
	\\\hline
	
	Proxy + Java
	&
	\dots der Proxy für den Besucher verborgen werden soll.
	&
	\dots es sich um eine kleine Webanwendung handelt und viel Konfiguration vermieden werden soll. 
	\\\hline
	
	\caption{Kurzzusammenfassung der Lösungsansätze}\label{tab:vergleich_loesungen}
\end{longtable}

Die \autoref{tab:erfuellung} listet erneut die Kürzel der in \autoref{sec:anforderungen} gestellten Anforderung und vergleicht, in welchem Kapitel dieser Arbeit diese erfüllt bzw. gelöst wurden.

\begin{longtable}{| c | c|} 
	\hline 
	\thead{Anforderung} & \thead{Gelöst in Kapitel} \\ 
	
	\hhline{|=|=|}
	A1 &  \ref{sec:integrationen} \\ 
	\hline
	A2 &  \ref{sec:integrationen} \\  
	\hline
	A3 &  - \\  
	\hline 
	A4 &  \ref{sec:integrationen} \\   
	\hline
	A5 &  \ref{sec:integrationen} \\   
	\hline
	A5.1 &  \ref{sec:direkte-integration-in-das-aem}\\  
	\hline 
	A5.2 &  \ref{sec:ansatz-java} \\  
	\hline 
	A5.3 &  \ref{sec:ansatz-javascript}\\  
	\hline 
	A6 &  \ref{sec:aufbereiten} \\   
	\hline
	A7 &  \ref{sec:suchmaschinen} \\   
	\hline
	A8 &  \ref{sec:sol_proxy}\\ 
	\hline 
	A9 & \ref{sec:konflikte}\\
	\hline
	A10 & \ref{sec:integrationen}\\
	\hline
	\caption{Anforderungserfüllung}\label{tab:erfuellung}
\end{longtable}

Der Punkt A3 \quotes{Die AJC soll sich über die Autoren-Bedienoberfläche, sofern erforderlich, konfigurieren lassen.} ließ sich nicht erfüllen. Zwar ist eine Konfiguration einer \ajc und somit auch einer Webanwendung generell möglich, es konnte aber innerhalb dieser Arbeit nicht weiter darauf eingegangen werden.