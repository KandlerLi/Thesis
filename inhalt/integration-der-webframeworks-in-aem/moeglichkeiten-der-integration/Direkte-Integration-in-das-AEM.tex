\subsection{Direkte Integration in das AEM}
\label{sec:direkte-integration-in-das-aem}
Innerhalb dieses Abschnittes sollen Lösungsansätze dargestellt werden, welche die direkte Integration in die \ac{jcr}-Struktur betreffen. Das bedeutet, dass alle benötigten App-Ressourcen in die \ac{jcr} abgelegt werden. Somit sind alle Ressourcen, die für die Ausführung der Webanwendung vonnöten sind, über das \ac{aem} erreichbar.\\
%Sofern der Zugriff auf das Ziel-\ac{aem}, insbesondere auf dessen \ac{jcr}-Strukur, besteht, lässt sich die Webanwendung direkt in selbiges durch die Verwendung von Clientlibs integrieren.\\
%Im ersten Schritt wird das gewünschte JavaScript-Framework als Clientlib, wie in \autoref{sec:clientlib} beschrieben, bereitgestellt. Sollte das Framework mehrmals innerhalb von \ac{aem} Verwendung finden, empfiehlt es sich dies unter \filefolder{/etc/clientlibs} zu platzieren, um global zur Verfügung zu stehen. Ansonsten wird das JavaScript-Framework in die zu entwickelnde \ajc abgelegt. \\
Um die Bereitstellung der App-Ressourcen für die Webanwendung zu ermöglichen, wurden zwei Lösungsansätze erarbeitet. In der ersten Lösung werden alle App-Ressourcen in eine Clientlib gepackt. Die zweite sieht als Lösung vor, diese unter dem Content-Ordner abzulegen.

\subsubsection{Lösungsansatz \quotes{Clientlib}}
\label{sec:sol_clientlib}

Der erste Lösungsansatz beschäftigt sich mit der Realisierung als Clientlib.

%\paragraphtbd{Transferieren der App-Ressourcen in das AEM}
Zunächst lässt sich über die \ac{webdav}-Schnittstelle mit einem entsprechenden Programm die Struktur des \ac{jcr} anzeigen und bearbeiten. Bei \acf{webdav}\index{WebDAV} handelt es sich um einen offenen Standard für die Bereitstellung von Daten über ein Netzwerk. Neben dem Löschen und Verschieben von Knoten ist hier auch das Hochladen von lokalen Dateien möglich. \\
Sofern die \ac{webdav}-Schnittstelle auf der Zielplattform nicht geöffnet sein sollte, gibt es die Alternative die App-Ressourcen als \ac{crx}-Paket bereit zu stellen. Hierfür werden zunächst wieder alle Inhalte in das \ac{jcr}, zum Beispiel über \ac{webdav}, geladen, dieses mal jedoch in die der Entwicklungsumgebung. Anschließend kann über die Autoren-Bedienoberfläche, siehe \autoref{sec:autor_ui}, ein \ac{crx}-Paket erstellt werden. Dieses ist ein Zip-Archiv mit Metainformationen, wie Name und Version des \ac{crx}-Pakets.\\
Das Erstellen eines \ac{crx}-Paketes erfolgt über eine entsprechende Seite des CRXDE Lite. Hier werden besagte Metainformationen gesetzt und alle benötigten Knoten für das \ac{crx}-Paket\index{CRX-Paket} angegeben. Nun erfolgt der Export des \ac{crx}-Pakets. Ist dies geschehen wird das \ac{crx}-Paket in die Zielplattform importiert. Der Inhalt des \ac{crx}-Paketes wird nun in das \ac{jcr} geschrieben. Bereits bestehende Knoten werden bei diesem Vorgang ggf. überschrieben.
\paragraph{Anpassen zur AJC}
Zunächst empfiehlt es sich, die App-Ressourcen vor der Anpassung zur \ajc aufzubereiten. Gerade das Konkatenieren der Ressourcen erleichtert die spätere Arbeit ungemein. Dieser Vorgang wird in \autoref{sec:konkat} erklärt. Auch sollte ab hier auf den \ac{aot}-Ansatz verzichtet und dafür der \ac{jit}-Ansatz bevorzugt werden. Beim \ac{aot} werden die Module in jeweils einzelne Dateien aufgeteilt, wohingegen bei \ac{jit} die Module zu einer Datei zusammengeschlossen werden können. \\
\autoref{lst:ajc-clientlib} zeigt, wie sich eine \ajc mit Clientlib zusammensetzen könnte.

\begin{lstlisting}[style=jcr, caption=Eine Beispiel-AJC mit Clientlib, label=lst:ajc-clientlib,escapechar=|]
/apps/integration/components/ |\label{line:component_start}|
  + aurelia-app
    + aurelia-app.html
    + cq:editConfig|\label{line:editconfig}|
      - cq:actions (String[])  = [edit, -, delete, insert] |\label{line:actions}| |\label{line:component_end}|

/etc/design/integration/aurelia-example/ |\label{line:clientlib_app_start}|
  + clientlib
    - jcr:primaryType (Name) = cq:ClientLibrary
    - catagories (String[]) = integration.aurelia.example
    - dependencies (String[]) = aurelia |\label{line:dependencies}|
    + css
      + style.css
    + css.txt
    + javascript
      + app.aurelia.js
    + js.txt
    + resources
      + templates
        + app.html
      + images
        + logo.png |\label{line:clientlib_app_end}|
    
/lib/clientlibs/ |\label{line:clientlib_start}|
  + aurelia
  	+ categories (Sting[]) = aurelia |\label{line:categorie_aurelia}|
    	+ javascript
      	+ aurelia.min.js
    	+ js.txt |\label{line:clientlib_end}|
\end{lstlisting}

Das gezeigte Beispiel stellt eine Aurelia-Anwendung dar und ist in drei Teile aufgeteilt. Der Knoten ab \autoref{line:clientlib_start}  beinhaltet die Ressourcen für das Webframework, hier Aurelia. Wichtig ist dabei, diese mindestens einer entsprechenden Kategorie, wie in \autoref{line:categorie_aurelia} zu sehen, zuzuordnen. \\
Alle weiteren App-Ressourcen werden unter den Knoten ab \autoref{line:clientlib_app_start} abgelegt. In \autoref{line:dependencies} werden die benötigten Abhängigkeiten zu anderen Clientlibs eingetragen. Diese werden anhand der jeweiligen Kategorie aufgelöst, wie diese im Beispiel in \autoref{line:categorie_aurelia} eingetragen wurde. Weitere App-Ressourcen wie HTML-Templates oder Bilder lassen sich auch unter diesen Knoten deponieren. \\
Der letzte Knoten, gezeigt in \autoref{line:component_start} bis \autoref{line:component_end}, stellt die \ajc dar und beinhaltet in \filefolder{aurelia-app.html} den Einstieg für die Webanwendung. Hier wird weiterhin der Aufruf für die Clientlib analog zum \autoref{lst:clientlib_useage} ausgelöst. Der Knoten in \autoref{line:editconfig} dient dazu, dass die Komponente in der Autoren-Bedienoberfläche genutzt und an ihre entsprechende Stelle platziert werden kann. \\
%Diese besteht neben den JavaScript- und CSS-Ressourcen aus einer HTML-Ressource, \autoref{line:html}, und einem cq:editConfig Knoten, \autoref{line:editconfig}. \\
%Im HTML befinden sich wie in \autoref{lst:clientlib_useage} entsprechende Anweisungen für Verwendung der Clientlib. Zusätzlich wird hier der Einsteig für die Webanwendung definiert. \\
%Durch das Hinzufügen des Dialog-Knoten erscheint die \ac{aem}-Komponente zur Auswahl in der Autoren-Bedienoberfläche. Fehlt dieser Knoten hat dies zur Folge, dass sich eine \ac{aem}-Komponente nur über die direkte Konfiguration des \ac{jcr} in eine Webseite einbinden lässt. \\
Um das Erstellen der beiden Clientlibs zu automatisieren, lässt sich die Grunt-Erweiterung \quotes{grunt-aem-clientlib-generator} \cite{wcmio2017} nutzen. Diese lässt sich dahingehend konfigurieren, die beiden Ordner \quotes{css} und \quotes{javascript} und die beiden Dateien \quotes{css.txt} und \quotes{js.txt} automatisiert zu erstellen.
\paragraph{Bewertung}

Die Lösung als Clientlib ist wie folgt zu bewerten.

\begin{minipage}[t]{0.5\textwidth}
	\textbf{Vorteile:}
	\begin{itemize}
		\item Lässt sich gut als \ac{crx}-Paket exportieren.
		\item Alles in einem System.
		\item Konfiguration über Autoren-Bedienoberfläche möglich.
	\end{itemize}
\end{minipage}
\begin{minipage}[t]{0.5\textwidth}
	\textbf{Nachteile:}
	\begin{itemize}
		\item Bestehende Verzeichnisstruktur muss abgeändert werden.
		\item Relative Pfade müssen angepasst werden.
		\item Wird etwas an der Webanwendung geändert, so muss diese auch in der \ajc geändert werden.
	\end{itemize}
\end{minipage}

\subsubsection{Lösungsansatz \quotes{Content Ordner}}
\label{sec:sol_content}
Eine simple Lösung wäre es, die Webanwendung unter den Content-Ordner  abzulegen. \\
Jeglicher Inhalt, der unter \filefolder{/content} abgelegt wird, ist direkt für den Client aufrufbar. Beispielsweise wäre unter der Standardkonfiguration von \ac{aem} die Datei \filefolder{/content/myapp/index.html} unter der \ac{url} \pseudourl{http://<Server-Hostname>:<AEM-Port>/content/myapp/index.html} erreichbar. Somit wäre es möglich, die Verzeichnisstruktur der Webanwendung 1:1 beizubehalten, sofern die komplette Webanwendung unter \filefolder{/content} abgelegt wird. Dieser Lösungsansatz verzichtet jedoch darauf, die Webanwendung als \ajc zu realisieren, was deren freie Positionierung und Konfigurierung über die Autoren-Bedienoberfläche ausschließt. Der Lösungsansatz \quotes{Content Ordner} empfiehlt sich somit nur für Webanwendungen, welche eine gesamte Webseite des \ac{aem} ausmachen und keine Konfiguration benötigen. 

\paragraph{Bewertung}

Die Lösung mit dem Content Ordner ist wie folgt zu bewerten.

\begin{minipage}[t]{0.5\textwidth}
	\textbf{Vorteile:}
	\begin{itemize}
		\item Beibehalten der Verzeichnisstruktur.
		\item Schnelle Integration.
		\item \ac{aot}-Ansatz möglich
	\end{itemize}
\end{minipage}
\begin{minipage}[t]{0.5\textwidth}
	\textbf{Nachteile:}
	\begin{itemize}
		\item Keine \ac{ajc}, somit auch kein einbetten in eine bestehende Webseite  über die Autoren-Bedienoberfläche des \ac{aem} möglich.
		\item Keine Konfiguration über die Autoren-Bedienoberfläche möglich.
	\end{itemize}
\end{minipage} 
