%\newpage\missing{Stark überarbeiten}\definecolor{lightred}{RGB}{255,230,230}\pagecolor{lightred}

\subsectiontbd{Direkte Integration in das AEM}
\label{sec:direkte-integration-in-das-aem}
Sofern der Zugriff auf das Ziel-\ac{aem}, insbesondere auf dessen \ac{jcr}-Strukur, besteht, lässt sich die Webanwendung direkt in selbiges durch die Verwendung von Clientlibs integrieren.\\
Im ersten Schritt wird das gewünschte JavaScript-Framework als Clientlib, wie in \autoref{sec:clientlib} beschrieben, bereitgestellt. Sollte das Framework mehrmals Verwendung finden empfiehlt es sich dies unter /etc/clientlibs zu platzieren. \\
Um die Bereitstellung der App-Ressourcen für die Webanwendung zu ermöglichen existieren mehrere Ansätze. Je nach verwendetem Framework lassen sich bestimmte Ansätze leichter erreichen.

\subsubsectiontbd{Verwendung als Clientlib}

Besteht die Webanwendung lediglich aus JavaScript und Stylesheets, bestellt die Möglichkeit diese komplett als Clientlib umzusetzen.
Hierfür wird wie in \autoref{sec:clientlib} beschrieben im \ac{jcr} eine entsprechende Struktur benötigt. Das Übertragen der App-Ressourcen in das \ac{jcr} kann über mehrere Wege erfolgen. \\

\paragraphtbd{Transferieren der App-Ressourcen in das AEM}
Zunächst lässt sich über die \ac{webdav}-Schnittstelle mit einem entsprechenden Programm die Struktur des \ac{jcr} anzeigen und bearbeiten. Neben dem Löschen und Verschieben von Knoten ist hier auch das Hochladen von lokalen Dateien möglich. \\
Sofern die \ac{webdav}-Schnittstelle auf der Zielplattform nicht zugänglich sein sollte, weil diese zum Beispiel deaktiviert wurde, gibt es die Alternative die App-Ressourcen als \ac{crx}-Paket bereit zu stellen. Hierfür werden zunächst wieder alle Inhalte in das \ac{jcr}, zum Beispiel über \ac{webdav}, geladen, dieses mal jedoch in dem vom Entwickler. Anschließend kann über die Autoren-Bedienoberfläche, siehe \autoref{sec:autor_ui}, ein \ac{crx}-Paket erstellt werden. Dieses ist ein Zip-Archiv mit Metainformationen, wie Name und Version des \ac{crx}-Pakets.\\
Das Erstellen eines \ac{crx}-Paketes erfolgt über eine entsprechende Seite des CRXDE Lite. Hier werden besagte Metainformationen gesetzt und alle benötigten Knoten für das \ac{crx}-Paket angegeben. Nun erfolgt der Export des \ac{crx}-Pakets. Ist dies geschehen wird das \ac{crx}-Paket in die Zielplattform importiert. Der Inhalt des \ac{crx}-Paketes wird nun in das \ac{jcr} geschrieben. Bereits bestehende Knoten werden bei diesem Vorgang ggf. überschrieben.

\subsubsectiontbd{Content Ordner}
Jeglicher Inhalt, der unter /content abgelegt wird, ist direkt über einen HTTP-Request aufrufbar. Unter der Standardkonfiguration von \ac{aem} ist die Datei \hl{/content/myapp/index.html} zu erreichen über \hl{http://<Server-Domain>:<Port>/content/myapp/index.app}. 
\subsubsectiontbd{Hybride}
Innerhalb einer Clientlib lassen sich lediglich JavaScript und CSS hinterlegen. Falls eine Webanwendung aber noch weitere App-Ressourcen wie Bilder und HTML beinhaltet, so werden diese unter einen anderen Knoten abgelegt. Es bietet sich an diese unter \hl{/content} abzulegen, da wie zuvor beschrieben der Zugriff auf hier abgelegte Dateien direkt zur Verfügung steht.

%\newpage\pagecolor{white}