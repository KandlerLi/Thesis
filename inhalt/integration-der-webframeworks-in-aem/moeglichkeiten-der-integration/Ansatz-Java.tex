\subsectiontbd{Ansatz Java}
\label{sec:ansatz-java}
Hier ist die Grundidee, dass die Webanwendung sich nicht im \ac{aem} befindet, sondern sich auf einen separaten Webserver liegt. Ziel ist es, dass die Webanwendung autonom auf dem separaten Webserver lauffähig ist und über eine \ac{aem}-Komponente auf Basis von Java und JavaScript in das \ac{aem} integriert wird.\\
Im Folgenden wird der separate Webserver, auf dem sich die App-Ressourcen für die Webanwendung befinden, kurz Webserver, und der Server mit der lauffähigen \ac{aem}-Instanz kurz \ac{aem}-Server genannt. Die Integration erfolgt somit vom Webserver in den \ac{aem}-Server. Die Webanwendung ist über \serverB aufrufbar und soll später über den \ac{aem}-Server über \serverA erreichbar sein.\\

\subsubsectiontbd{Erklärung}

In der Regel erfolgt das Laden einer Webanwendung wie in \autoref{sec:http} beschrieben in drei Schritten. Zunächst wird die HTML-Seite angefordert, anschließend wird der Quellcode der HTML-Seite nach weiteren App-Ressourcen durchsucht und auch diese geladen. Weitere App-Ressourcen werden nun über Ajax angefordert und vom Server retourniert. \\
Es wird davon ausgegangen, dass die Schritte unter \serverB korrekt durchlaufen werden und somit die Webanwendung wie gewünscht im Browser erscheint. \\
Innerhalb von \ac{aem} hat die \ac{aem}-Komponente die Aufgabe die Webanwendung vom Webserver zu laden. Dabei muss jedoch die HTML-Seite, die Ajax-Anfragen und ggf. auch weitere App-Ressourcen manipuliert werden. Die Komponente dient somit als Bindeglied zwischen \ac{aem} und dem Webserver.\\
Zuvor muss jedoch der Webserver und die Webanwendung gewissen Anpassungen unterliegen. Diese werden in \autoref{sec:anpassen-des-webservers} und \autoref{sec:anpassen-der-webanwendung} beschrieben.
\subsubsectiontbd{Anpassen des Webservers}
\label{sec:anpassen-des-webservers}
Gängige Webbrowser verbieten, dass JavaScript, und somit auch \ac{ajax}, auf externe Webserver zugreifen dürfen. Grund dafür ist das Sicherheitskonzept der \ac{sop}. \\
Unterscheiden sich die \ac{url} von der Webseite von der aus die Anfrage gestartet wurde und vom Ziel in Protokoll, Domain oder Port, so wird die Anfrage vom Webbrowser geblockt. Im zuvor beschriebenen Beispiel würden sich \serverA und \serverB im Host (www.example.com anstelle von aem.example.com) und dem Port (3000 anstelle von 80) unterscheiden. \\
Damit der Zugriff doch funktioniert lässt sich der Mechanismus des \ac{cors} verwenden.
\missing{CORS etwas detailierter erklären}
Hierfür antwortet der Zielwebserver der Anfrage mit einen entsprechenden HTTP-Header. Hier sind alle URLs gelistet, von denen gestattet sind eine Anfrage zu auszuführen. Der Browser vergleicht jetzt den HTTP-Header mit der URL, von dem die Anfrage gestartet wurde. Im Erfolgsfall blockiert der Browser nicht und der Aufruf wird zu Ende ausgeführt.
\subsubsectiontbd{Anpassen der Webanwendung}
\label{sec:anpassen-der-webanwendung}
Auch die Webanwendung muss unter Umständen noch vor dem Produktiveinsatz konfiguriert werden. \\
Sollten HTML-Templates in AngularJS Verwendung finden, kann es hier bei dem Ladenvorgang zu einer \ac{sce}-Fehlermeldung führen. Dies geschieht, da der HTML-Code von extern kommt und potenziell unsichereren Code mitführen kann. AngularJS blockt daher per Standardkonfiguration den Ladevorgang. Doch durch entsprechende Anpassungen der Webanwendung lassen sich Ausnahmen hinzufügen, oder das \ac{sce} auch vollständig deaktivieren \cite{Google2016d}.
\missing{Andere Frameworks?}

\subsubsectiontbd{Ablauf der AEM-Komponente}
Den ungefähren Ablauf der \ac{aem}-Komponente ist \autoref{img:java} zu entnehmen.
\begin{figure}[H]
	\begin{center}
		\includegraphics[width=1\textwidth]{servlet.png}
		\caption{Ungefährer Ablauf der AEM-Komponente auf Java-Basis}
		\label{img:java}
	\end{center}
\end{figure}

Der erste Schritt wird im folgenden \autoref{sec:html-seite-aufbereiten}, der zweite Schritt im darauffolgenden \autoref{sec:ajax-anfragen-anpassen} erklärt.

\paragraphtbd{HTML-Seite aufbereiten}
\label{sec:html-seite-aufbereiten}

Der Benutzer fordert zunächst die Seite \serverA an. In dieser befindet sich die \ac{aem}-Komponente, welche zuvor so konfiguriert wurde, dass an dessen Stelle die Webanwendung von \serverB erscheinen soll. Nun wird \filefolder{AppBuilder.java} damit beauftragt den HTML-Quellcode der Einstiegsseite der Webanwendung anzufordern, welche sich unter \serverB[index.html] befindet. \filefolder{AppBuilder.java} muss jetzt den Quellcode für die Integration aufbereiten. Einige dieser Aufbereitungsaufgabe sind Anwendungsspezifisch und vom jeweils verwendeten Webframework abhängig. Mehr zu den Unterschieden ist dem \autoref{sec:aufbereitungsaufgabe} zu entnehmen. Was immer geschehen muss, ist, dass relative Referenzen auf JavaScript und CSS-Dateien durch absolute ersetzt werden. Eine Referenz auf \pseudourl{./app.module.js} sollte nach besagten Schritt auf \serverB[app.module.js] verweisen.\\
Der Quellcode von \serverB[index.html] könnte wie in \autoref{lst:serverB} aussehen.

\begin{lstlisting}[style=htmlcssjs, caption=Ausgangssituation auf Server B, label=lst:serverB]
<!DOCTYPE html>
<html ng-app="phonecatAPP">
	<head>
		<link rel="stylesheet" href="app.css" />
		<script src="bower_components/angular/angular.js"></script>
		<script src="app.module.js"></script>
		<title>My Webbapplication</title>
	</head>
	<body>
		<div class="view-container">
			<div ng-view class="view-frame"></div>
		</div>
	</body>
</html>
\end{lstlisting}

Aus den Aufbereitung würde ein Quellcode wie in \autoref{lst:serverA} resultieren.

\begin{lstlisting}[style=htmlcssjs, caption=Aufbereiteter Quellcode, label=lst:serverA]
<link rel="stylesheet" href="http://www.example.com:3000/spa/app.css" />
<script src="http://www.example.com:3000/spa/bower_components/angular/angular.js"></script>
<script src="http://www.example.com:3000/spa/app.module.js"></script>
<div ng-app="phonecatApp">
	<div class="view-container">
		<div ng-view class="view-frame"></div>
	</div>
</div>

\end{lstlisting}

\ac{aem} ersetzt nun den Platzhalter der Komponente mit den aufbereiteten Quellcode. Diese HTML-Seite ist nun fertig gerendert und wird wieder an den Client geliefert. \\
\filefolder{AppBuilder.java} nutzt die Java-Bibliothek jsoup \cite{Hedley2016}. Dieser HTML-Parser bietet Klassen und Funktionen zum Laden, Traversieren und Manipulieren von HTML-Seiten an. Der \ac{dom} der geladenen \ac{html}-Seite kann mithilfe von \ac{css}-Selektoren nach HTML-Elementen, wie den script- und link-Elementen, durchsucht werden. Die Installation der Bibliothek zur Nutzung innerhalb von \ac{aem} erfolgt als OSGi-Bundle.
\paragraphtbd{Ajax-Anfragen anpassen}
\label{sec:ajax-anfragen-anpassen}

Im zweiten Schritt werden Ajax-Anfragen umgeleitet. Diese sind zumeist ebenfalls relativ und würden somit versuchen eine App-Ressource unter \serverA anfordern, die sich jedoch unter \serverB befindet. Gerade \ac{di}-Bibliotheken laden läufig Skripte nach. \\
Für die Umleitung überschreibt \filefolder{ajaxredirect.js} das \ac{xhr}-Objekt. Die \ac{xhr}-Schnittstelle wird vom \ac{w3c} spezifiziert und ist ein wesentlicher Bestandteil von \ac{ajax}, da hiermit Daten asynchron zwischen Server und Client auszutauschen. \\
Das überschriebene \ac{xhr}-Objekt überprüft ob es sich bei der angeforderten \ac{url} um eine relative oder absolute handelt. Falls diese relativ ist hat dies zu bedeuten, dass die App-Ressource sich unter \serverB befindet. In dem Fall wird die Anfrage nun über ein Servlet, hier mit dem Namen \filefolder{GetData.java}, umgeleitet. Die Umleitung über das Servlet anstelle einer direkten Anfrage an den Webserver hat zwei Vorteile. \\
Zum einen werden hier Probleme mit Zugriffsrechten, wie das in \autoref{sec:anpassen-des-webservers} erläuterte Sicherheitskonzept \ac{sop}, erleichtert. Ohne Servlet würde der Client direkt auf Server-Ressourcen des Webservers zugreifen, was in der Regel zu einem Fehler führt. Der Webserver müsste somit allen Clienten den Zugriff autorisieren. Über ein Servlet jedoch ist es so, dass der \ac{aem}-Server die Anfrage an den Webserver stellt. Somit kann der Webserver so konfiguriert werden, dass er nur Anfragen vom \ac{aem}-Server erlaubt.
\\
Zum anderen können an dieser Stelle noch Änderungen an den angefragten App-Ressourcen erfolgen. Zum Beispiel könnten hier die Templates einer AngularJS-Webanwendung nach relativen Hyperlinks durchsucht und diese angepasst werden.



Ist sie relativ, wird dieser die \serverB am Anfang angefügt. Je nach Anwendung wird auch die \ac{url} vom \ac{aem}, also \serverA, angehängt, zum Beispiel wenn sich die Bilder dort befinden.  Zuletzt wird die \ac{ajax}-Anfrage normal ausgeführt.
\subsubsectiontbd{Aufbereitungsaufgabe bei unterschiedlichen Frameworks}
\label{sec:aufbereitungsaufgabe}
\missingall
\subsubsectiontbd{Konfiguration}
Des weilen wird gewünscht die Komponente zu konfigurieren. Da sich diese nicht direkt im \ac{aem} befindet bedarf es einer Schnittstelle für den Austausch der Konfigurationen zwischen \ac{aem} und dem Webserver.
\missingtext
\subsubsectiontbd{Varianten}
\missingall