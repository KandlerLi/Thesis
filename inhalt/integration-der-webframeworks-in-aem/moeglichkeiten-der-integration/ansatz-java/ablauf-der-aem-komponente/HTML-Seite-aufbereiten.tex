\paragraphtbd{HTML-Seite aufbereiten}
\label{sec:html-seite-aufbereiten}

Der Benutzer fordert zunächst die Seite \serverA an. In dieser befindet sich die \ac{aem}-Komponente, welche zuvor so konfiguriert wurde, dass an dessen Stelle die Webanwendung von \serverB erscheinen soll. Nun wird \filefolder{AppBuilder.java} damit beauftragt den HTML-Quellcode der Einstiegsseite der Webanwendung anzufordern, welche sich unter \serverB[index.html] befindet. \filefolder{AppBuilder.java} muss jetzt den Quellcode für die Integration aufbereiten. Einige dieser Aufbereitungsaufgabe sind Anwendungsspezifisch und vom jeweils verwendeten Webframework abhängig. Mehr zu den Unterschieden ist dem \autoref{sec:aufbereitungsaufgabe} zu entnehmen. Was immer geschehen muss, ist, dass relative Referenzen auf JavaScript und CSS-Dateien durch absolute ersetzt werden. Eine Referenz auf \pseudourl{./app.module.js} sollte nach besagten Schritt auf \serverB[app.module.js] verweisen.\\
Der Quellcode von \serverB[index.html] könnte wie in \autoref{lst:serverB} aussehen.

\begin{lstlisting}[style=htmlcssjs, caption=Ausgangssituation auf Server B, label=lst:serverB]
<!DOCTYPE html>
<html ng-app="phonecatAPP">
	<head>
		<link rel="stylesheet" href="app.css" />
		<script src="bower_components/angular/angular.js"></script>
		<script src="app.module.js"></script>
		<title>My Webbapplication</title>
	</head>
	<body>
		<div class="view-container">
			<div ng-view class="view-frame"></div>
		</div>
	</body>
</html>
\end{lstlisting}

Aus den Aufbereitung würde ein Quellcode wie in \autoref{lst:serverA} resultieren.

\begin{lstlisting}[style=htmlcssjs, caption=Aufbereiteter Quellcode, label=lst:serverA]
<link rel="stylesheet" href="http://www.example.com:3000/spa/app.css" />
<script src="http://www.example.com:3000/spa/bower_components/angular/angular.js"></script>
<script src="http://www.example.com:3000/spa/app.module.js"></script>
<div ng-app="phonecatApp">
	<div class="view-container">
		<div ng-view class="view-frame"></div>
	</div>
</div>

\end{lstlisting}

\ac{aem} ersetzt nun den Platzhalter der Komponente mit den aufbereiteten Quellcode. Diese HTML-Seite ist nun fertig gerendert und wird wieder an den Client geliefert. \\
\filefolder{AppBuilder.java} nutzt die Java-Bibliothek jsoup \cite{Hedley2016}. Dieser HTML-Parser bietet Klassen und Funktionen zum Laden, Traversieren und Manipulieren von HTML-Seiten an. Der \ac{dom} der geladenen \ac{html}-Seite kann mithilfe von \ac{css}-Selektoren nach HTML-Elementen, wie den script- und link-Elementen, durchsucht werden. Die Installation der Bibliothek zur Nutzung innerhalb von \ac{aem} erfolgt als OSGi-Bundle.