\subsectiontbd{Lösungsansatz \quotes{Proxy}}
\label{sec:sol_proxy}
Ein weiterer Lösungsansatz wird durch Hinzunahme eines Proxy realisierbar. Durch den Einsatz eines Proxy lassen sich die vom Client angeforderten App-Ressourcen während des HTTP-Request-Response-Zykluses manipulieren.\\
Im allgemeinen ist ein Proxy\index{Proxy} ein Rechner, der als Mittelsmann zwischen Client und Server dient. Der Client stellt hierbei die Anfragen an den Proxy. Der Proxy leitet anschließend die Anfrage weiter an den Server. Client und Server kommunizieren somit nie direkt miteinander, sondern nur indirekt über den Proxy.\\
\missing{cite}
Im Rahmen dieser Arbeit handelt es sich bei einem Proxy um einen weiteren speziell konfigurierten Webserver. Die Web-Ressourcen werden dabei von dem eigentlichen Webserver über den Proxy an den Client ausgeliefert. Die angefragten App-Ressourcen lassen sich hierbei über den Proxy manipulieren. 

\subsubsection{Erklärung}
Die Lösung sieht hierbei vor, dass ein weiterer Webserver als Proxy unter dem Hostnamen \hostnameP~ so angepasst wird, dass dieser alle Anfragen wie in \autoref{img:proxy} über ein zentrales Skript umleitet. Besagtes Skript kann in einer beliebigen, vom Proxy unterstützten Programmiersprache realisiert werden, wie hier in \acs{php}. Die Konfiguration ließe sich beim Apache HTTP-Server mit dem Modul \quotes{mod\_rewrite} realisieren \cite{Foundation2017a}. Die entsprechende Konfiguration ist \autoref{lst:rewrite} zu entnehmen. \\
\begin{minipage}{\textwidth}
\begin{lstlisting}[style=jcr, caption=Konfiguration des mod\_rewrite Modules für den Proxy-Lösungsansatz, label=lst:rewrite]
RewriteEngine on
RewriteRule ^/app/(.*)$ /proxy.php?url=$1 [L]
\end{lstlisting}
\end{minipage}
Der Ablauf könnte hierbei wie in \autoref{img:proxy} erfolgen. \\
Die Konfiguration definiert, dass, sofern eine App-Ressource angefragt wird, deren Pfad mit \filefolder{/app/} beginnt, die Anfrage über \filefolder{proxy.php} umgeleitet wird. Der restliche Pfad, hier \filefolder{templates/news.html}, wird dem Skript als Parameter mit übergeben.\\
Dieses ermittelt nun die benötigte App-Ressource vom eigentlichen Webserver, der unter dem Hostnamen und Port \pseudourl{www.exmaple.com:3000} zu erreichen ist. Etwaige Abweichungen vom angeforderten Pfad und der Pfad auf dem Webserver lassen sich über das Skript wie im Beispiel bewerkstelligen und korrigieren.\\
Anschließend bearbeitet das Skript die App-Ressource entsprechend und liefert die bearbeitete App-Ressource als Ergebnis zurück. Diese Aktionen bleiben dem Client verborgen. Aus dessen Sicht wird ihm lediglich die angeforderte App-Ressource geliefert. \\

\begin{figure}[H]
	\begin{center}
		{\footnotesize\includesvg[width=1.0\textwidth]{proxy}}
		\caption{Funktionsweise des Proxy}
		\label{img:proxy}
	\end{center}
\end{figure}

Die Lösung ähnelt dem Servlet aus \autoref{sec:ansatz-java}, da auch hier zur Laufzeit Änderungen vorgenommen werden können.\\
Seitens \ac{aem} kommt auch hier eine \ac{aem}-Komponente zum Einsatz, welche die initiale Anfrage an die HTML-Seite der Webanwendung stellt, was in etwa dem ersten Schritt aus \autoref{img:java} entspräche. Die Logik der \ajc vom Lösungsansatz \quotes{Java-Servlet} wird hier auf das Proxy-Skript ausgelagert. Innerhalb der \ac{aem}-Komponente wird lediglich die Start-URL des Proxy definiert.
\subsubsection{Bewertung}
Die Lösung mit einem Proxy ist wie folgt zu bewerten.

\begin{minipage}[t]{0.5\textwidth}
	\textbf{Vorteile:}
	\begin{itemize}
		\item Die Struktur der Webanwendung bleibt erhalten.
		\item Die Webanwendung ist unabhängig von \ac{aem} testbar.
		\item Die entwickelte \ajc lässt sich wiederverwenden.
		\item Die Logik des Skriptes ist weitestgehend frei wählbar und nicht von \ac{aem} abhängig.
	\end{itemize}
\end{minipage}
\begin{minipage}[t]{0.5\textwidth}
	\textbf{Nachteile:}
	\begin{itemize}
		\item Betreiben von zwei bzw. drei Diensten nötig (AEM und Webserver und Proxy).
		\item Ggf. zusätzliche Konfiguration von Webserver und Webanwendung nötig.
		\item Zusätzliche Laufzeit (gering, ggf. vernachlässigbar).
	\end{itemize}
\end{minipage}