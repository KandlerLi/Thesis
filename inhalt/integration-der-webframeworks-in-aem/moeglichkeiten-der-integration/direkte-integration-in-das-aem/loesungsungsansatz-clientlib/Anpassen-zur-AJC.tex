\paragraph{Anpassen zur AJC}
Zunächst empfiehlt es sich, die App-Ressourcen vor der Anpassung zur \ajc aufzubereiten. Gerade das Konkatenieren der Ressourcen erleichtert die spätere Arbeit ungemein. Dieser Vorgang wird in \autoref{sec:konkat} erklärt. Auch sollte ab hier auf den \ac{aot}-Ansatz verzichtet und dafür der \ac{jit}-Ansatz bevorzugt werden. Beim \ac{aot} werden die Module in jeweils einzelne Dateien aufgeteilt, wohingegen bei \ac{jit} die Module zu einer Datei zusammengeschlossen werden können. \\
\autoref{lst:ajc-clientlib} zeigt, wie sich eine \ajc mit Clientlib zusammensetzen könnte.

\begin{lstlisting}[style=jcr, caption=Eine Beispiel-AJC mit Clientlib, label=lst:ajc-clientlib,escapechar=|]
/apps/integration/components/ |\label{line:component_start}|
  + aurelia-app |\label{line:editconfig}|
    + aurelia-app.html
    + cq:editConfig|\label{line:editconfig}|
      - cq:actions (String[])  = [edit, -, delete, insert] |\label{line:actions}| |\label{line:component_end}|

/etc/design/integration/aurelia-example/ |\label{line:clientlib_app_start}|
  + clientlib
    - jcr:primaryType (Name) = cq:ClientLibrary
    - catagories (String[]) = integration.aurelia.example
    - dependencies (String[]) = aurelia |\label{line:dependencies}|
    + css
      + style.css
    + css.txt
    + javascript
      + app.aurelia.js
    + js.txt
    + resources
      + templates
        + app.html
      + images
        + logo.png |\label{line:clientlib_app_end}|
    
/lib/clientlibs/ |\label{line:clientlib_start}|
  + aurelia
  	+ categories (Sting[]) = aurelia |\label{line:categorie_aurelia}|
    	+ javascript
      	+ aurelia.min.js
    	+ js.txt |\label{line:clientlib_end}|
\end{lstlisting}

Das gezeigte Beispiel stellt eine Aurelia-Anwendung dar und ist in drei Teile aufgeteilt. Der Knoten ab \autoref{line:clientlib_start}  beinhaltet die Ressourcen für das Webframework, hier Aurelia. Wichtig ist dabei, diese mindestens einer entsprechenden Kategorie, wie in \autoref{line:categorie_aurelia} zu sehen, zuzuordnen. \\
Alle weiteren App-Ressourcen werden unter den Knoten ab \autoref{line:clientlib_app_start} abgelegt. In \autoref{line:dependencies} werden die benötigten Abhängigkeiten zu anderen Clientlibs eingetragen. Diese werden anhand der jeweiligen Kategorie aufgelöst, wie diese im Beispiel in \autoref{line:categorie_aurelia} eingetragen wurde. Weitere App-Ressourcen wie HTML-Templates oder Bilder lassen sich auch unter diesen Knoten deponieren. \\
Der letzte Knoten, gezeigt in \autoref{line:component_start} bis \autoref{line:component_end}, stellt die \ajc dar und beinhaltet in \filefolder{aurelia-app.html} den Einstieg für die Webanwendung. Hier wird weiterhin der Aufruf für die Clientlib analog zum \autoref{lst:clientlib_useage} ausgelöst. Der Knoten in \autoref{line:editconfig} dient dazu, dass die Komponente in der Autoren-Bedienoberfläche genutzt und an ihre entsprechende Stelle platziert werden kann. \\
%Diese besteht neben den JavaScript- und CSS-Ressourcen aus einer HTML-Ressource, \autoref{line:html}, und einem cq:editConfig Knoten, \autoref{line:editconfig}. \\
%Im HTML befinden sich wie in \autoref{lst:clientlib_useage} entsprechende Anweisungen für Verwendung der Clientlib. Zusätzlich wird hier der Einsteig für die Webanwendung definiert. \\
%Durch das Hinzufügen des Dialog-Knoten erscheint die \ac{aem}-Komponente zur Auswahl in der Autoren-Bedienoberfläche. Fehlt dieser Knoten hat dies zur Folge, dass sich eine \ac{aem}-Komponente nur über die direkte Konfiguration des \ac{jcr} in eine Webseite einbinden lässt. \\
Um das Erstellen der beiden Clientlibs zu automatisieren, lässt sich die Grunt-Erweiterung \quotes{grunt-aem-clientlib-generator} \cite{wcmio2017} nutzen. Diese lässt sich dahingehend konfigurieren, die beiden Ordner \quotes{css} und \quotes{javascript} und die beiden Dateien \quotes{css.txt} und \quotes{js.txt} automatisiert zu erstellen.