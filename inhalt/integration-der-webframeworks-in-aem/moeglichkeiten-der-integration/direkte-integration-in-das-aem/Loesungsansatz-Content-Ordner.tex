\subsubsection{Lösungsansatz \quotes{Content Ordner}}
\label{sec:sol_content}
Eine simple Lösung wäre es, die Webanwendung unter den Content-Ordner  abzulegen. \\
Jeglicher Inhalt, der unter \filefolder{/content} abgelegt wird, ist direkt für den Client aufrufbar. Beispielsweise wäre unter der Standardkonfiguration von \ac{aem} die Datei \filefolder{/content/myapp/index.html} unter der \ac{url} \pseudourl{http://<Server-Hostname>:<AEM-Port>/content/myapp/index.html} erreichbar. Somit wäre es möglich, die Verzeichnisstruktur der Webanwendung 1:1 beizubehalten, sofern die komplette Webanwendung unter \filefolder{/content} abgelegt wird. Dieser Lösungsansatz verzichtet jedoch darauf, die Webanwendung als \ajc zu realisieren, was deren freie Positionierung und Konfigurierung über die Autoren-Bedienoberfläche ausschließt. Der Lösungsansatz \quotes{Content Ordner} empfiehlt sich somit nur für Webanwendungen, welche eine gesamte Webseite des \ac{aem} ausmachen und keine Konfiguration benötigen. 

\paragraph{Bewertung}

Die Lösung mit dem Content Ordner ist wie folgt zu bewerten.

\begin{minipage}[t]{0.5\textwidth}
	\textbf{Vorteile:}
	\begin{itemize}
		\item Beibehalten der Verzeichnisstruktur.
		\item Schnelle Integration.
		\item \ac{aot}-Ansatz möglich
	\end{itemize}
\end{minipage}
\begin{minipage}[t]{0.5\textwidth}
	\textbf{Nachteile:}
	\begin{itemize}
		\item Keine \ac{ajc}, somit auch kein einbetten in eine bestehende Webseite  über die Autoren-Bedienoberfläche des \ac{aem} möglich.
		\item Keine Konfiguration über die Autoren-Bedienoberfläche möglich.
	\end{itemize}
\end{minipage} 