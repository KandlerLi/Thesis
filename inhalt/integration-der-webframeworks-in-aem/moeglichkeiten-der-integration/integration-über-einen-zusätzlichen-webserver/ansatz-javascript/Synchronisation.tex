\paragraph{Synchronisation}
Eine JavaScript-Ressource nachträglich zu laden lässt sich erreichen, indem mittels JavaScript ein \inlinecode{script}-Element in den \inlinecode{head} eingefügt wird. Dies würde in etwa wie in \autoref{lst:appendchild} aussehen.

\begin{lstlisting}[style=htmlcssjs, caption=Nachladen einer JavaScript-Ressource, label=lst:appendchild]
var s = document.createElement('script');
s.src = "http://www.example.com/jquery.js";
document.head.appendChild(s);
\end{lstlisting}

Wird dieser Code jedoch wiederholt mit unterschiedlichen URLs aufgerufen, so können Abhängigkeitsproblemen auftreten, da mehrere JavaScript-Ressourcen nahezu gleichzeitig geladen und ausgeführt werden. Somit kann es passieren, dass eine JavaScript-Datei, die eine Webanwendung beinhaltet, vor dem Webframework ausgeführt wird, obwohl die Reihenfolge anders angegeben wurde.
Für die Lösung dieses Problems lässt sich die Promise-Schnittstelle verwenden, wie nachfolgendes Codebeispiel zeigt.

\begin{lstlisting}[style=htmlcssjs, caption=Laden und Synchronisieren von JavaScript-Ressourcen, label=lst:js-syncro,escapechar=|]
var i = 0, scriptUrls = ...;
getScript();
function getScript() {
  loadScript(scriptUrls[i++]).then(getScript); |\label{line:load}|
}

function loadScript(src) {
  return new Promise(function (resolve) {
    var s = document.createElement('script');
    s.src = src;
    s.onload = resolve;
    document.head.appendChild(s);
  });
}
\end{lstlisting}

Es wird davon ausgegangen, dass die Variable \inlinecode{scriptUrls} bereits ein Array mit den absoluten Adressen der zu ladenden JavaScript-Ressourcen in der korrekten Reihenfolge beinhaltet. Diese wurden beispielsweise zuvor in der HTML-Seite durch den Einsatz von regulären Ausdrücken gefunden. Anschließend wird die Funktion \inlinecode{getScript} aufgerufen. Die \autoref{line:load} bewirkt nun zwei Dinge. Zunächst wird die erste JavaScript-Ressource geladen und ausgeführt. Anschließend wird, sobald der JavaScript-Code ausgeführt wurde, die Funktion erneut aufgerufen, um die nächste JavaScript-Ressource zu laden. \\
Damit nicht fälschlicherweise eine JavaScript-Ressource geladen wird, bevor die vorherige nicht zum Ende ausgeführt wurde, wird hier ein Promise (eng. Versprechen) verwendet, einen in ECMAScript 6 eingeführten Standard. Die Funktion bewirkt, dass die JavaScript-Ressource geladen und ausgeführt wird. Anschließend bewirkt der Promise, dass die Funktion \inlinecode{loadScript} innerhalb von \inlinecode{then(...)} in \autoref{line:load} ausgeführt wird.