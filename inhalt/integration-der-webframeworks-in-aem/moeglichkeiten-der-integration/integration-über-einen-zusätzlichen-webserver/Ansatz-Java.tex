\subsubsection{Lösungsansatz \quotes{Java-Servlet}}
\label{sec:ansatz-java}

Hier ist die Grundidee, dass sich die App-Ressourcen nicht im \ac{jcr} des \ac{aem} befinden, sondern auf einem separaten Webserver vorliegen. Ziel ist es, dass die Webanwendung unabhängig auf dem separaten Webserver zum Testen lauffähig ist. Über eine \ac{aem}-Komponente wird diese in das \ac{aem} importiert. Die \ac{aem}-Komponente bezieht die App-Ressourcen vom Webserver und passt diese ggf. an, so dass die Webanwendung korrekt in der Webseite des \ac{aem} dargestellt wird, wie in \autoref{img:import} illustriert wird. Da auch diese \ac{aem}-Komponente eine Webanwendung abstrahiert, wird diese im Folgenden ebenfals als \ajc bezeichnet.

\begin{figure}[H]
	\begin{center}
		\includesvg[width=1.0\textwidth]{import}
		\caption{Integration vom Webserver in den AEM Server mittels einer \ajc}
		\label{img:import}
	\end{center}
\end{figure}
Dabei werden die App-Ressourcen zur Laufzeit der \ajc, also wenn die Webseite geladen wird, jeweils erneut vom Webserver geladen. Ein Ablegen in das \ac{jcr} ist nicht vorgesehen. Ein temporäres Zwischenspeichern (Cachen) wäre von Grund auf denkbar ist, wird innerhalb dieser Arbeit jedoch nicht weiter behandelt. \\
Im Folgenden wird der separate Webserver, auf dem sich die App-Ressourcen für die Webanwendung befinden, kurz \quotes{Webserver} und der Server mit der lauffähigen \ac{aem}-Instanz kurz \quotes{\ac{aem}-Server} genannt. Die Integration erfolgt somit vom Webserver in den \ac{aem}-Server. Die Webanwendung ist über \serverB aufrufbar und soll später über den \ac{aem}-Server über \serverA erreichbar sein.\\

\paragraph{Erklärung}

In der Regel erfolgt das Laden einer Webanwendung genau wie bei einer Webseite, wie in \autoref{sec:http} beschrieben, in drei Schritten. Zunächst wird die HTML-Seite angefordert, anschließend wird der Quellcode der HTML-Seite nach weiteren App-Ressourcen durchsucht und auch diese werden geladen. Weitere App-Ressourcen werden nun über Ajax angefordert und vom Server retourniert. Es wird davon ausgegangen, dass diese Schritte unter \serverB korrekt durchlaufen werden und somit die Webanwendung wie gewünscht im Browser erscheint. \\
Nun ist es Aufgabe der \ajc, die App-Ressourcen so aufzurufen und zu bearbeiten, dass die Webanwendung korrekt in einer Webseite des \ac{aem} dargestellt wird. Dafür müssen die HTML-Seite, die Ajax-Anfragen und ggf. auch weitere App-Ressourcen manipuliert werden. Die \ajc dient somit als Bindeglied zwischen \ac{aem} und dem Webserver.\\
Zuvor muss jedoch der Webserver und die Webanwendung gewissen Anpassungen unterliegen. Diese werden in \autoref{sec:anpassen-des-webservers} und \autoref{sec:anpassen-der-webanwendung} beschrieben.
\paragraph{Anpassen des Webservers}
\label{sec:anpassen-des-webservers}
Gängige Webbrowser verbieten, dass JavaScript und somit auch \ac{ajax} auf Web-Ressourcen anderer Webserver zugreifen dürfen, als dem, unter dem das Skript gerade ausgeführt wird. Grund dafür ist das Sicherheitskonzept der \acf{sop}\index{SOP}. Die Herkunft (eng. Origin) setzt sich hierbei aus dem verwendetem Protokoll, Hostname und Port der \ac{url} zusammen. Unterscheidet sich die Herkunft von der Webseite, von der aus die Anfrage gestartet wurde, und von der angefragten Web-Ressource, so wird die Anfrage vom Webbrowser geblockt. Im zuvor beschriebenen Beispiel würden sich \serverA und \serverB in dem Hostnamen (\pseudourl{www.example.com} anstelle von \pseudourl{aem.example.com}) und dem Port (\pseudourl{3000} anstelle vom Standardport für HTTP \pseudourl{80}) unterscheiden. Eine Ajax-Anfragen vom AEM-Server an den Webserver würde einen Fehler erzeugen.\\
Damit der Zugriff doch funktioniert, lässt sich der von W3C definierte Mechanismus des \ac{cors} anwenden.

\begin{figure}[H]
	\begin{center}
		\includesvg[width=1\textwidth]{cors}
		\caption{Ablauf bei CORS}
		\label{img:cors}
	\end{center}
\end{figure}
Wie in \autoref{img:cors} sendet der Browser des Clients bei seiner Anfrage seine Herkunft (Origin) mit. Am Beispiel hier wäre die Herkunft die \ac{url} der Webseite, von der die Anfrage ausgeführt wird, also \serverAN. Der Server antwortet mit einer Liste von erlaubten Hostnamen und der angeforderten Web-Ressource. Der Browser überprüft nun, ob sich die Herkunft der Ajax-Anfrage in der Liste befindet. Im Erfolgsfall wird die Ajax-Anfrage erfolgreich beendet, ansonsten wird eine Fehlermeldung erzeugt. \\
Neben der gezeigten Variante lässt sich \ac{cors}\index{CORS} auch mit anderen Zugriffsregeln verwenden. Durch \inlinecode{Access-Control-Allow-Origin: *} werden alle Anfragen genehmigt, durch \inlinecode{Access-Control-Allow-Methods: GET} werden alle mit der Zugriffsmethode GET gestellten Anfragen erfolgreich ausgeführt \cite{W3C2014a}. \\
Die Konfiguration der erlaubten Herkünfte ist Webserverspezifisch und in der jeweiligen Dokumentation nachzulesen. Für Apache kann hier das Modul \quotes{mod\_headers} \cite{Foundation2017} genutzt werden.
\paragraphtbd{Anpassen der Webanwendung}
\label{sec:anpassen-der-webanwendung}
Auch die Webanwendung muss unter Umständen noch vor dem Produktiveinsatz konfiguriert werden. \\
Sollten HTML-Templates in AngularJS Verwendung finden, kann es hier bei dem Ladevorgang zu einer \ac{sce}-Fehlermeldung kommen. Dies geschieht, da der HTML-Code von extern kommt und potenziell unsichereren Code mitführen kann. AngularJS blockt daher per Standardkonfiguration den Ladevorgang. Doch durch entsprechende Anpassungen der Webanwendung lassen sich Ausnahmen hinzufügen oder das \ac{sce} auch vollständig deaktivieren \cite{Google2016d}.
\missing{Andere Frameworks?}

\paragraph{Ablauf}
Der ungefähre Ablauf ist \autoref{img:java} zu entnehmen. Dieser besteht grob zusammengefasst aus zwei Schritten. Der erste Schritt wird im folgenden \autoref{sec:html-seite-aufbereiten}, der zweite Schritt im darauf folgenden \autoref{sec:ajax-anfragen-anpassen} erklärt. \\
Die beiden Java-Ressourcen \filefolder{AppBuilder.java} und \filefolder{GetData.java} lassen sich mit einem OSGi-Bundle für das \ac{aem} bereitstellen. Die JavaScript-Ressource \filefolder{ajaxdirect.js} lässt sich in einer \ajc in Verbindung mit einer Clientlib ausliefern. 
Bei \filefolder{GetData.java} handelt es sich um ein Servlet. Das bedeutet, dass ein Browser dies über eine bestimmte \ac{url} aufrufen kann und eine entsprechende Antwort erhält.

\begin{figure}[H]
	\begin{center}
		\includegraphics[width=1\textwidth]{servlet.png}
		\caption{Ungefährer Ablauf der AEM-Komponente auf Java-Basis}
		\label{img:java}
	\end{center}
\end{figure}

\subparagraph{HTML-Seite aufbereiten}
\label{sec:html-seite-aufbereiten}
Der erste Schritt dient dazu, die Einstiegsseite der Webanwendung anzufordern. \\
Der Benutzer fordert hierbei zunächst die Webseite \serverA an. In dieser befindet sich die \ajc, an deren Stelle die Webanwendung von \serverB erscheinen soll.\\
Nun wird im ersten Schritt \filefolder{AppBuilder.java} damit beauftragt, den HTML-Quellcode von \serverB anzufordern. \filefolder{AppBuilder.java} muss jetzt den erhaltenen HTML-Quellcode für die Integration aufbereiten. Einige dieser Aufbereitungsaufgaben sind anwendungsspezifisch und vom jeweils verwendeten Webframework abhängig. Mehr zu den Unterschieden ist dem \autoref{sec:aufbereitungsaufgabe} zu entnehmen. Was immer geschieht, ist, dass relative Referenzen auf JavaScript und CSS-Dateien durch absolute ersetzt werden. Eine Referenz auf \pseudourl{./app.module.js} sollte nach besagtem Schritt auf \serverB[app.module.js] verweisen.\\
Der Quellcode von \serverB[index.html] könnte wie in \autoref{lst:serverB} aussehen.

\begin{lstlisting}[style=htmlcssjs, caption=Ausgangssituation auf Server B, label=lst:serverB]
<!DOCTYPE html>
<html ng-app="phonecatAPP">
	<head>
		<link rel="stylesheet" href="app.css" />
		<script src="bower_components/angular/angular.js"></script>
		<script src="app.module.js"></script>
		<title>My Webbapplication</title>
	</head>
	<body>
		<div class="view-container">
			<div ng-view class="view-frame"></div>
		</div>
	</body>
</html>
\end{lstlisting}

Aus der Aufbereitung würde ein Quellcode wie in \autoref{lst:serverA} resultieren.

\begin{lstlisting}[style=htmlcssjs, caption=Aufbereiteter Quellcode, label=lst:serverA]
<link rel="stylesheet" href="http://www.example.com:3000/spa/app.css" />
<script src="http://www.example.com:3000/spa/bower_components/angular/angular.js"></script>
<script src="http://www.example.com:3000/spa/app.module.js"></script>
<div ng-app="phonecatApp">
	<div class="view-container">
		<div ng-view class="view-frame"></div>
	</div>
</div>

\end{lstlisting}

\ac{aem} setzt nun an die Stelle der \ajc den aufbereiteten Quellcode. Diese HTML-Seite ist nun fertig bearbeitet und wird wieder an den Client geliefert. \\
\filefolder{AppBuilder.java} nutzt die Java-Bibliothek jsoup \cite{Hedley2016}. Dieser HTML-Parser bietet Klassen und Funktionen zum Laden, Traversieren und Manipulieren von HTML-Seiten an. Der \ac{dom} der geladenen \ac{html}-Seite kann mithilfe von \ac{css}-Selektoren nach HTML-Elementen, wie den Script- und Link-Elementen, durchsucht werden. Die Installation der Bibliothek zur Nutzung innerhalb von \ac{aem} erfolgt als OSGi-Bundle.
\subparagraph{Ajax-Anfragen anpassen}
\label{sec:ajax-anfragen-anpassen}

Im zweiten Schritt werden Ajax-Anfragen umgeleitet. Diese sind zumeist ebenfalls relativ und würden somit versuchen eine App-Ressource unter \serverA anzufordern, die sich jedoch unter \serverB befindet. Gerade \ac{di}-Bibliotheken laden häufig Skripte nach, aber auch Templates werden häufig nachträglich geladen. \\
Für die Umleitung überschreibt \filefolder{ajaxredirect.js} das \ac{xhr}-Objekt. Die \ac{xhr}-Schnittstelle wird vom \ac{w3c} spezifiziert und ist ein wesentlicher Bestandteil von \ac{ajax}, das dazu verwendet werden kann, Daten asynchron zwischen Server und Client auszutauschen. \\
Das überschriebene \ac{xhr}-Objekt überprüft, ob es sich während einer \ac{ajax}-Anfrage bei der angeforderten \ac{url} um eine relative oder absolute handelt. Falls diese relativ ist, hat dies zu bedeuten, dass die App-Ressource sich unter \serverB befindet. In dem Fall wird die Anfrage nun über ein Servlet, hier mit dem Namen \filefolder{GetData.java}, umgeleitet. Die Umleitung über das Servlet anstelle einer direkten Anfrage an den Webserver hat zwei Vorteile. \\
Zum einen werden hier Probleme mit Zugriffsrechten, wie das in \autoref{sec:anpassen-des-webservers} erläuterte Sicherheitskonzept \ac{sop}, erleichtert. Ohne Servlet würde der Client direkt auf Server-Ressourcen des Webservers zugreifen, was in der Regel zu einem Fehler führt. Der Webserver müsste somit allen Clienten den Zugriff autorisieren. Über ein Servlet jedoch ist es so, dass der \ac{aem}-Server die Anfrage an den Webserver stellt. Somit kann der Webserver dahingehend konfiguriert werden, dass er nur Anfragen vom \ac{aem}-Server erlaubt.\\
Zum anderen können an dieser Stelle noch Änderungen an den angefragten App-Ressourcen erfolgen. Beispielsweise könnten hier die Templates einer An\-gu\-lar\-JS-Web\-an\-wen\-dung nach Hyperlinks durchsucht und diese angepasst werden.



%Ist sie relativ, wird dieser die \serverB am Anfang angefügt. Je nach Anwendung wird auch die \ac{url} vom \ac{aem}, also \serverA, angehängt, zum Beispiel wenn sich die Bilder dort befinden.  Zuletzt wird die \ac{ajax}-Anfrage normal ausgeführt.
\paragraphtbd{Aufbereitungsaufgabe bei unterschiedlichen Frameworks}

\label{sec:aufbereitungsaufgabe}
\missingall
%\paragraphtbd{Konfiguration}
Des weilen wird gewünscht die Komponente zu konfigurieren. Da sich diese nicht direkt im \ac{aem} befindet bedarf es einer Schnittstelle für den Austausch der Konfigurationen zwischen \ac{aem} und dem Webserver.
\missingtext
%\paragraphtbd{Varianten}
\missingall
\paragraph{Bewertung}

Die Lösung mit Java ist wie folgt zu bewerten.

\begin{minipage}[t]{0.5\textwidth}
	\textbf{Vorteile:}
	\begin{itemize}
		\item Die Struktur der Webanwendung bleibt erhalten.
		\item Die Webanwendung ist unabhängig von \ac{aem} testbar.
		\item Die Entwickelte \ajc lässt sich für ähnliche Webanwendungen wiederverwenden.
	\end{itemize}
\end{minipage}
\begin{minipage}[t]{0.5\textwidth}
	\textbf{Nachteile:}
	\begin{itemize}
		\item Betreiben zweier Dienste nötig (AEM und Webserver).
		\item Ggf. zusätzliche Konfiguration von Webserver und Webanwendung nötig.
		\item Zusätzliche Laufzeit (gering, ggf. vernachlässigbar).
		\item Neue \ajc für stark abweichende Webanwendungen nötig.
	\end{itemize}
\end{minipage}