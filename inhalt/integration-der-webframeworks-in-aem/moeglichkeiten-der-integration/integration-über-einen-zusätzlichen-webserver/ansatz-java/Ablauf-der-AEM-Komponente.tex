\paragraph{Ablauf}
Der ungefähre Ablauf ist \autoref{img:java} zu entnehmen. Dieser besteht grob zusammengefasst aus zwei Schritten. Der erste Schritt wird im folgenden \autoref{sec:html-seite-aufbereiten}, der zweite Schritt im darauf folgenden \autoref{sec:ajax-anfragen-anpassen} erklärt. \\
Die beiden Java-Ressourcen \filefolder{AppBuilder.java} und \filefolder{GetData.java} lassen sich mit einem OSGi-Bundle für das \ac{aem} bereitstellen. Die JavaScript-Ressource \filefolder{ajaxdirect.js} lässt sich in einer \ajc in Verbindung mit einer Clientlib ausliefern. 
Bei \filefolder{GetData.java} handelt es sich um ein Servlet. Das bedeutet, dass ein Browser dies über eine bestimmte \ac{url} aufrufen kann und eine entsprechende Antwort erhält.

\begin{figure}[H]
	\begin{center}
		\includegraphics[width=1\textwidth]{servlet.png}
		\caption{Ungefährer Ablauf der AEM-Komponente auf Java-Basis}
		\label{img:java}
	\end{center}
\end{figure}

\subparagraph{HTML-Seite aufbereiten}
\label{sec:html-seite-aufbereiten}
Der erste Schritt dient dazu, die Einstiegsseite der Webanwendung anzufordern. \\
Der Benutzer fordert hierbei zunächst die Webseite \serverA an. In dieser befindet sich die \ajc, an deren Stelle die Webanwendung von \serverB erscheinen soll.\\
Nun wird im ersten Schritt \filefolder{AppBuilder.java} damit beauftragt, den HTML-Quellcode von \serverB anzufordern. \filefolder{AppBuilder.java} muss jetzt den erhaltenen HTML-Quellcode für die Integration aufbereiten. Einige dieser Aufbereitungsaufgaben sind anwendungsspezifisch und vom jeweils verwendeten Webframework abhängig.
%Mehr zu den Unterschieden ist dem \autoref{sec:aufbereitungsaufgabe} zu entnehmen.
Was immer geschieht, ist, dass relative Referenzen auf JavaScript und CSS-Dateien durch absolute ersetzt werden. Eine Referenz auf \pseudourl{./app.module.js} sollte nach besagtem Schritt auf \serverB[app.module.js] verweisen.\\
Der Quellcode von \serverB[index.html] könnte wie in \autoref{lst:serverB} aussehen.

\begin{lstlisting}[style=htmlcssjs, caption=Ausgangssituation auf Server B, label=lst:serverB]
<!DOCTYPE html>
<html ng-app="phonecatAPP">
	<head>
		<link rel="stylesheet" href="app.css" />
		<script src="bower_components/angular/angular.js"></script>
		<script src="app.module.js"></script>
		<title>My Webbapplication</title>
	</head>
	<body>
		<div class="view-container">
			<div ng-view class="view-frame"></div>
		</div>
	</body>
</html>
\end{lstlisting}

Aus der Aufbereitung würde ein Quellcode wie in \autoref{lst:serverA} resultieren.

\begin{lstlisting}[style=htmlcssjs, caption=Aufbereiteter Quellcode, label=lst:serverA]
<link rel="stylesheet" href="http://www.example.com:3000/spa/app.css" />
<script src="http://www.example.com:3000/spa/bower_components/angular/angular.js"></script>
<script src="http://www.example.com:3000/spa/app.module.js"></script>
<div ng-app="phonecatApp">
	<div class="view-container">
		<div ng-view class="view-frame"></div>
	</div>
</div>

\end{lstlisting}

\ac{aem} setzt nun an die Stelle der \ajc den aufbereiteten Quellcode. Diese HTML-Seite ist nun fertig bearbeitet und wird wieder an den Client geliefert. \\
\filefolder{AppBuilder.java} nutzt die Java-Bibliothek jsoup \cite{Hedley2016}. Dieser HTML-Parser bietet Klassen und Funktionen zum Laden, Traversieren und Manipulieren von HTML-Seiten an. Der \ac{dom} der geladenen \ac{html}-Seite kann mithilfe von \ac{css}-Selektoren nach HTML-Elementen, wie den Script- und Link-Elementen, durchsucht werden. Die Installation der Bibliothek zur Nutzung innerhalb von \ac{aem} erfolgt als OSGi-Bundle.
\subparagraph{Ajax-Anfragen anpassen}
\label{sec:ajax-anfragen-anpassen}

Im zweiten Schritt werden Ajax-Anfragen umgeleitet. Diese sind zumeist ebenfalls relativ und würden somit versuchen eine App-Ressource unter \serverA anzufordern, die sich jedoch unter \serverB befindet. Gerade \ac{di}-Bibliotheken laden häufig Skripte nach, aber auch Templates werden häufig nachträglich geladen. \\
Für die Umleitung überschreibt \filefolder{ajaxredirect.js} das \ac{xhr}-Objekt. Die \ac{xhr}-Schnittstelle wird vom \ac{w3c} spezifiziert und ist ein wesentlicher Bestandteil von \ac{ajax}, das dazu verwendet werden kann, Daten asynchron zwischen Server und Client auszutauschen. \\
Das überschriebene \ac{xhr}-Objekt überprüft, ob es sich während einer \ac{ajax}-Anfrage bei der angeforderten \ac{url} um eine relative oder absolute handelt. Falls diese relativ ist, hat dies zu bedeuten, dass die App-Ressource sich unter \serverB befindet. In dem Fall wird die Anfrage nun über ein Servlet, hier mit dem Namen \filefolder{GetData.java}, umgeleitet. Die Umleitung über das Servlet anstelle einer direkten Anfrage an den Webserver hat zwei Vorteile. \\
Zum einen werden hier Probleme mit Zugriffsrechten, wie das in \autoref{sec:anpassen-des-webservers} erläuterte Sicherheitskonzept \ac{sop}, erleichtert. Ohne Servlet würde der Client direkt auf Server-Ressourcen des Webservers zugreifen, was in der Regel zu einem Fehler führt. Der Webserver müsste somit allen Clienten den Zugriff autorisieren. Über ein Servlet jedoch ist es so, dass der \ac{aem}-Server die Anfrage an den Webserver stellt. Somit kann der Webserver dahingehend konfiguriert werden, dass er nur Anfragen vom \ac{aem}-Server erlaubt.\\
Zum anderen können an dieser Stelle noch Änderungen an den angefragten App-Ressourcen erfolgen. Beispielsweise könnten hier die Templates einer An\-gu\-lar\-JS-Web\-an\-wen\-dung nach Hyperlinks durchsucht und diese angepasst werden.



%Ist sie relativ, wird dieser die \serverB am Anfang angefügt. Je nach Anwendung wird auch die \ac{url} vom \ac{aem}, also \serverA, angehängt, zum Beispiel wenn sich die Bilder dort befinden.  Zuletzt wird die \ac{ajax}-Anfrage normal ausgeführt.