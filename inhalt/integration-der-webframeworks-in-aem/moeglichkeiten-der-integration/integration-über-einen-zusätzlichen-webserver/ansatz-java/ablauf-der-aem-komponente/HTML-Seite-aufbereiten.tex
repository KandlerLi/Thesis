\subparagraph{HTML-Seite aufbereiten}
\label{sec:html-seite-aufbereiten}
Der erste Schritt dient dazu, die Einstiegsseite der Webanwendung anzufordern. \\
Der Benutzer fordert hierbei zunächst die Webseite \serverA an. In dieser befindet sich die \ajc, an deren Stelle die Webanwendung von \serverB erscheinen soll.\\
Nun wird im ersten Schritt \filefolder{AppBuilder.java} damit beauftragt, den HTML-Quellcode von \serverB anzufordern. \filefolder{AppBuilder.java} muss jetzt den erhaltenen HTML-Quellcode für die Integration aufbereiten. Einige dieser Aufbereitungsaufgaben sind anwendungsspezifisch und vom jeweils verwendeten Webframework abhängig. Mehr zu den Unterschieden ist dem \autoref{sec:aufbereitungsaufgabe} zu entnehmen. Was immer geschieht, ist, dass relative Referenzen auf JavaScript und CSS-Dateien durch absolute ersetzt werden. Eine Referenz auf \pseudourl{./app.module.js} sollte nach besagtem Schritt auf \serverB[app.module.js] verweisen.\\
Der Quellcode von \serverB[index.html] könnte wie in \autoref{lst:serverB} aussehen.

\begin{lstlisting}[style=htmlcssjs, caption=Ausgangssituation auf Server B, label=lst:serverB]
<!DOCTYPE html>
<html ng-app="phonecatAPP">
	<head>
		<link rel="stylesheet" href="app.css" />
		<script src="bower_components/angular/angular.js"></script>
		<script src="app.module.js"></script>
		<title>My Webbapplication</title>
	</head>
	<body>
		<div class="view-container">
			<div ng-view class="view-frame"></div>
		</div>
	</body>
</html>
\end{lstlisting}

Aus der Aufbereitung würde ein Quellcode wie in \autoref{lst:serverA} resultieren.

\begin{lstlisting}[style=htmlcssjs, caption=Aufbereiteter Quellcode, label=lst:serverA]
<link rel="stylesheet" href="http://www.example.com:3000/spa/app.css" />
<script src="http://www.example.com:3000/spa/bower_components/angular/angular.js"></script>
<script src="http://www.example.com:3000/spa/app.module.js"></script>
<div ng-app="phonecatApp">
	<div class="view-container">
		<div ng-view class="view-frame"></div>
	</div>
</div>

\end{lstlisting}

\ac{aem} setzt nun an die Stelle der \ajc den aufbereiteten Quellcode. Diese HTML-Seite ist nun fertig bearbeitet und wird wieder an den Client geliefert. \\
\filefolder{AppBuilder.java} nutzt die Java-Bibliothek jsoup \cite{Hedley2016}. Dieser HTML-Parser bietet Klassen und Funktionen zum Laden, Traversieren und Manipulieren von HTML-Seiten an. Der \ac{dom} der geladenen \ac{html}-Seite kann mithilfe von \ac{css}-Selektoren nach HTML-Elementen, wie den Script- und Link-Elementen, durchsucht werden. Die Installation der Bibliothek zur Nutzung innerhalb von \ac{aem} erfolgt als OSGi-Bundle.