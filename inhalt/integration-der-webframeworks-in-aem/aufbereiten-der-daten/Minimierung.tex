\subsection{Minimierung}
Für den Produktiveinsatz lassen sich die JavaScript-Ressourcen minimieren. Hierbei werden zum Beispiel unnötige Leerzeichen entfernt oder auch lange Variablennamen durch kürzere ersetzt. Dies minimiert die Übertragungszeit vom Server zum Client, da das zu übertragende Volumen sinkt.\\
Generell existieren für die Minimierung von JavaScript-Quellcode verschiedenste Werkzeuge. Eines davon wäre UglifyJS \cite{Bazon2016}. Hiermit lässt sich JavaScript-Quellcode unter anderem minimieren. Das Werkzeug wird unter der BSD-Lizenz veröffentlicht und über \ac{npm} installiert.	