\subsection{Übertragen von Ressourcen}
Die Übertragung von Ressourcen in das \ac{aem} kann über mehrere Wege erfolgen.

\subsubsection{WebDAV}
Zunächst lässt sich über die \ac{webdav}-Schnittstelle mit einem entsprechenden Programm die Struktur des \ac{jcr} anzeigen und bearbeiten. Bei \acf{webdav}\index{WebDAV} handelt es sich um einen offenen Standard für die Bereitstellung von Daten über ein Netzwerk. Neben dem Löschen und Verschieben von Knoten ist hier auch das Hochladen von lokalen Dateien möglich.

\subsubsection{HTTP}
Weiterhin lassen sich Ressourcen auch über \ac{http} übertragen. Ein hilfreiches Programm hierfür ist \quotes{curl}\index{curl} \cite{Stenberg2017}. Über die Kommandozeile lassen sich so Dateien in das \ac{aem} übertragen. Folgender Befehl in \autoref{lst:curl} würde zum Beispiel eine Datei mit dem Namen \filefolder{flowerpot.png} in das \ac{jcr} unter \filefolder{/content/flowerpot.png} ablegen. Für die Authentifizierung wird ein gültiger Benutzername mit Passwort, hier beides \quotes{admin}, benötigt.

\begin{lstlisting}[style=jcr, caption=Eine Datei mit curl in das AEM übertragen, label=lst:curl]
curl -u admin:admin -T flowerpot.png http://localhost:4502/content/flowerpot.png
\end{lstlisting}


\subsubsection{CRX-Paket}
Sofern die \ac{webdav}-Schnittstelle auf der Zielplattform nicht geöffnet sein sollte, gibt es die Alternative, die App-Ressourcen als \ac{crx}-Paket bereit zu stellen. Hierfür werden zunächst wieder alle Inhalte in das \ac{jcr}, zum Beispiel über \ac{webdav}, geladen, dieses Mal jedoch in die der Entwicklungsumgebung. Anschließend kann über die Autoren-Bedienoberfläche, siehe \autoref{sec:autor_ui}, ein \ac{crx}-Paket erstellt werden. Dieses ist ein Zip-Archiv mit Metainformationen, wie Name und Version des \ac{crx}-Pakets.\\
Das Erstellen eines \ac{crx}-Paketes erfolgt über eine entsprechende Seite des CRXDE Lite. Hier werden besagte Metainformationen gesetzt und alle benötigten Knoten für das \ac{crx}-Paket\index{CRX-Paket} angegeben. Nun erfolgt der Export des \ac{crx}-Pakets. Ist dies geschehen, wird das \ac{crx}-Paket in die Zielplattform importiert. Der Inhalt des \ac{crx}-Paketes wird nun in das \ac{jcr} geschrieben. Bereits bestehende Knoten werden bei diesem Vorgang ggf. überschrieben.
Alternativ lässt sich das \ac{crx}-Paket auch mit Maven erstellen. Durch Maven\index{Maven} lassen sich standardisierte Anwendungen erstellen und verwalten. Eine entsprechende Anleitung zum Erstellen eines Maven-Projektes und wie sich hiermit ein \ac{crx}-Paket erschaffen lässt stellt Adobe zur Verfügung \cite{Incorporated2017};