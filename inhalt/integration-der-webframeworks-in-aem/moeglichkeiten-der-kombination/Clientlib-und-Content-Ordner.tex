\subsection{Clientlib und Content-Ordner}
Diese Kombination sieht vor, dass die App-Ressourcen in zwei Gruppen unterteilt werden. Zum einen wären hier alle App-Ressourcen, die bereits in der HTML-Seite referenziert sind, also zu Beginn geladen werden. Diese CSS- und JavaScript-Ressourcen werden als Clientlib realisiert. Zum anderen werden alle weiteren Ressourcen unter \pseudourl{/content} abgelegt. 
Folgendes Beispiel zur Erklärung. Es wird angenommen, dass sich die \ajc unter \pseudourl{http://aem.example.com/content/app} befindet. Über die \ajc wurde bereits die Clientlib und somit die im ersten Schritt benötigten App-Ressourcen geladen. Nun wird über Ajax versucht eine JavaScript-Ressource unter der relativen \ac{url} \pseudourl{./templates/app.html} aufzurufen. Somit ergibt sich die absolute Adresse \pseudourl{http://aem.example.com/content/app/templates/app.html}, was innerhalb des \ac{jcr} die Knoten \pseudourl{/content/app/templates/app.html}. Der genaue Ablageknoten unter \pseudourl{/content} ist somit von der \ac{url} abhängig, unter dem die Webanwendung auffindbar sein soll.

\subsubsection{Bewertung}

Die Lösungskombination ist wie folgt zu bewerten.

\begin{minipage}[t]{0.5\textwidth}
	\textbf{Vorteile:}
	\begin{itemize}
		\item Schnell zu realisieren.
		\item Keinerlei Anpassung der Webanwendung nötig. Relative Pfade werden korrekt aufgelöst.
		\item Die Realisierung erfolgt als AJC, somit lässt sich diese frei innerhalb einer Webseite platzieren.
	\end{itemize}
\end{minipage}
\begin{minipage}[t]{0.5\textwidth}
	\textbf{Nachteile:}
	\begin{itemize}
		\item App-Ressoucen müssen innerhalb eines entsprechenden Knoten unter \pseudourl{/content} liegen. Wird die Anwendung in eine andere Webseite eingebettet, müssen die App-Ressourcen in einen anderen Knoten verschoben werden.
	\end{itemize}
\end{minipage}