\sectiontbd{Zu integrierende Webanwendungen}
Als Basis für den Versuch der Integration wurde für jedes Webframework eine Webanwendung ausgewählt. Diese stammen größtenteils direkt von den Entwickeln des jeweiligen Webframeworks und gebrauchen bereits zahlreiche Funktionen der jeweils verfügbaren Funktionalitäten. Der Quellcode liegt als eigens eingerichtetes GIT Respository zur Einsicht vor. Eine Übersicht der Webframeworks bietet \autoref{tab:webanwendungen}.

\begin{minipage}{\textwidth}
\begin{longtable}{| c | c | c | c |} 
	\hline 
	\thead{Framework} & \thead{Basiert auf} & \thead{GIT Respository} \\ 
	
	\hline 
	AngularJS 1.5.8 & angular-phonecat \cite{Angular2016} &  \cite{Kandler2016a} \\
	\hline
	AngularJS2 2.0.0-rc.3& Tutorial: Tour of Heroes \cite{Google2016e} & \cite{Kandler2016b}\\ 
	\hline
	Aurelia & & \cite{Kandler2016c}\\ 
	\hline
	React & Redux Tetris \cite{Lugo2016} & \cite{Kandler2016d}\\ 
	
	\hline 
	\caption{Webanwendungen}\label{tab:webanwendungen}
\end{longtable}
\end{minipage}

Im Folgenden wird vermehrt der Begriff \quotes{die Webanwendung} fallen. Auch wenn hier der Singular verwendet wird, sind damit alle vier Webanwendungen gemeint. 

