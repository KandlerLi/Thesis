\section{Zu integrierende Webanwendungen}
\label{sec:zu-integrierende-webanwendungen}

Als Basis für den Versuch der Integration wurde für jedes Webframework eine Webanwendung ausgewählt. Diese stammen größtenteils direkt von den Entwicklern des jeweiligen Webframeworks und gebrauchen bereits zahlreiche der jeweils verfügbaren Funktionalitäten. Der Quellcode liegt als eigens eingerichtetes GIT Respository zur Einsicht vor. Eine Übersicht der Webframeworks bietet \autoref{tab:webanwendungen}.

\begin{minipage}{\textwidth}
\begin{longtable}{| c | c | c | c |} 
	\hline
	\thead{Framework} & \thead{Basiert auf} & \thead{GIT Respository} \\ 
	
	\hhline{|=|=|=|=|} 
	AngularJS 1.5.8 & angular-phonecat \cite{Angular2016} &  \cite{Kandler2016a} \\
	\hline
	AngularJS2 2.0.0-rc.3& Tutorial: Tour of Heroes \cite{Google2016e} & \cite{Kandler2016b}\\ 
	\hline
	Aurelia 1.1.0& Quick Start \cite{Eisenberg2017} & \cite{Kandler2016c}\\ 
	\hline
	React 15.2.0 & Redux Tetris \cite{Lugo2016} & \cite{Kandler2016d}\\ 
	
	\hline 
	\caption{Webanwendungen}\label{tab:webanwendungen}
\end{longtable}
\end{minipage}

Im Folgenden wird vermehrt der Begriff \quotes{die Webanwendung} verwendet. Hiermit ist eine beliebige der vier genannten Webanwendungen genannt. Die Aussage \quotes{Es wird versucht eine Webanwendung in das \ac{aem} zu integrieren} bedeutet somit, dass nacheinander versucht wird die vier in \ref{tab:webanwendungen} genannten Webanwendungen in ein \ac{aem} zu integrieren.
%Auch wenn hier der Singular verwendet wird, sind damit alle vier Webanwendungen gemeint. 

