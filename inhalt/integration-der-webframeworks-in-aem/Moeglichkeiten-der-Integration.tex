\section{Möglichkeiten der Integration}
\label{sec:integrationen}
Abhängig von verschiedenen Faktoren wurden unterschiedliche Möglichkeiten erarbeitet, um eine Webanwendung in eine \ac{aem}-Instanz als \ajc zu integrieren. Neben dem zugrunde liegenden clientseitigen Webframework ist der wichtigste Entscheidungsfaktor dafür, welche Lösung genutzt werden kann, die IT-Landschaft der Zielplattform. Je nach dessen Konfiguration bieten sich gewisse Lösungen mehr an oder sind auch als mögliche Lösung von vornherein ausgeschlossen. Auch gewisse wünsche des Kunden, wie die Bereitstellung der App-Ressourcen auf einen seperaten Webserver wird berücksichtigt werden.
%Insbesondere das Thema Zugriffsrechte spielt hier eine immense Rolle. Diese könnten so geregelt sein, dass ein zu tiefer Eingriff in das \ac{aem} und eine freie Konfiguration des Servers nicht gestattet sind.

\subsection{Direkte Integration in das AEM}
\label{sec:direkte-integration-in-das-aem}
Innerhalb dieses Abschnittes sollen Lösungsansätze dargestellt werden, welche die direkte Integration in die \ac{jcr}-Struktur betreffen. Das bedeutet, dass alle benötigten App-Ressourcen in die \ac{jcr} abgelegt werden. Somit sind alle Ressourcen, die für die Ausführung der Webanwendung vonnöten sind, über das \ac{aem} erreichbar.\\
%Sofern der Zugriff auf das Ziel-\ac{aem}, insbesondere auf dessen \ac{jcr}-Strukur, besteht, lässt sich die Webanwendung direkt in selbiges durch die Verwendung von Clientlibs integrieren.\\
%Im ersten Schritt wird das gewünschte JavaScript-Framework als Clientlib, wie in \autoref{sec:clientlib} beschrieben, bereitgestellt. Sollte das Framework mehrmals innerhalb von \ac{aem} Verwendung finden, empfiehlt es sich dies unter \filefolder{/etc/clientlibs} zu platzieren, um global zur Verfügung zu stehen. Ansonsten wird das JavaScript-Framework in die zu entwickelnde \ajc abgelegt. \\
Um die Bereitstellung der App-Ressourcen für die Webanwendung zu ermöglichen, wurden zwei Lösungsansätze erarbeitet. In der ersten Lösung werden alle App-Ressourcen in eine Clientlib gepackt. Die zweite sieht als Lösung vor, diese unter dem Content-Ordner abzulegen.

\subsubsection{Lösungsansatz \quotes{Clientlib}}
\label{sec:sol_clientlib}

Der erste Lösungsansatz beschäftigt sich mit der Realisierung als Clientlib.

%\paragraphtbd{Transferieren der App-Ressourcen in das AEM}
Zunächst lässt sich über die \ac{webdav}-Schnittstelle mit einem entsprechenden Programm die Struktur des \ac{jcr} anzeigen und bearbeiten. Bei \acf{webdav}\index{WebDAV} handelt es sich um einen offenen Standard für die Bereitstellung von Daten über ein Netzwerk. Neben dem Löschen und Verschieben von Knoten ist hier auch das Hochladen von lokalen Dateien möglich. \\
Sofern die \ac{webdav}-Schnittstelle auf der Zielplattform nicht geöffnet sein sollte, gibt es die Alternative die App-Ressourcen als \ac{crx}-Paket bereit zu stellen. Hierfür werden zunächst wieder alle Inhalte in das \ac{jcr}, zum Beispiel über \ac{webdav}, geladen, dieses mal jedoch in die der Entwicklungsumgebung. Anschließend kann über die Autoren-Bedienoberfläche, siehe \autoref{sec:autor_ui}, ein \ac{crx}-Paket erstellt werden. Dieses ist ein Zip-Archiv mit Metainformationen, wie Name und Version des \ac{crx}-Pakets.\\
Das Erstellen eines \ac{crx}-Paketes erfolgt über eine entsprechende Seite des CRXDE Lite. Hier werden besagte Metainformationen gesetzt und alle benötigten Knoten für das \ac{crx}-Paket\index{CRX-Paket} angegeben. Nun erfolgt der Export des \ac{crx}-Pakets. Ist dies geschehen wird das \ac{crx}-Paket in die Zielplattform importiert. Der Inhalt des \ac{crx}-Paketes wird nun in das \ac{jcr} geschrieben. Bereits bestehende Knoten werden bei diesem Vorgang ggf. überschrieben.
\paragraph{Anpassen zur AJC}
Zunächst empfiehlt es sich, die App-Ressourcen vor der Anpassung zur \ajc aufzubereiten. Gerade das Konkatenieren der Ressourcen erleichtert die spätere Arbeit ungemein. Dieser Vorgang wird in \autoref{sec:konkat} erklärt. Auch sollte ab hier auf den \ac{aot}-Ansatz verzichtet und dafür der \ac{jit}-Ansatz bevorzugt werden. Beim \ac{aot} werden die Module in jeweils einzelne Dateien aufgeteilt, wohingegen bei \ac{jit} die Module zu einer Datei zusammengeschlossen werden können. \\
\autoref{lst:ajc-clientlib} zeigt, wie sich eine \ajc mit Clientlib zusammensetzen könnte.

\begin{lstlisting}[style=jcr, caption=Eine Beispiel-AJC mit Clientlib, label=lst:ajc-clientlib,escapechar=|]
/apps/integration/components/ |\label{line:component_start}|
  + aurelia-app
    + aurelia-app.html
    + cq:editConfig|\label{line:editconfig}|
      - cq:actions (String[])  = [edit, -, delete, insert] |\label{line:actions}| |\label{line:component_end}|

/etc/design/integration/aurelia-example/ |\label{line:clientlib_app_start}|
  + clientlib
    - jcr:primaryType (Name) = cq:ClientLibrary
    - catagories (String[]) = integration.aurelia.example
    - dependencies (String[]) = aurelia |\label{line:dependencies}|
    + css
      + style.css
    + css.txt
    + javascript
      + app.aurelia.js
    + js.txt
    + resources
      + templates
        + app.html
      + images
        + logo.png |\label{line:clientlib_app_end}|
    
/lib/clientlibs/ |\label{line:clientlib_start}|
  + aurelia
  	+ categories (Sting[]) = aurelia |\label{line:categorie_aurelia}|
    	+ javascript
      	+ aurelia.min.js
    	+ js.txt |\label{line:clientlib_end}|
\end{lstlisting}

Das gezeigte Beispiel stellt eine Aurelia-Anwendung dar und ist in drei Teile aufgeteilt. Der Knoten ab \autoref{line:clientlib_start}  beinhaltet die Ressourcen für das Webframework, hier Aurelia. Wichtig ist dabei, diese mindestens einer entsprechenden Kategorie, wie in \autoref{line:categorie_aurelia} zu sehen, zuzuordnen. \\
Alle weiteren App-Ressourcen werden unter den Knoten ab \autoref{line:clientlib_app_start} abgelegt. In \autoref{line:dependencies} werden die benötigten Abhängigkeiten zu anderen Clientlibs eingetragen. Diese werden anhand der jeweiligen Kategorie aufgelöst, wie diese im Beispiel in \autoref{line:categorie_aurelia} eingetragen wurde. Weitere App-Ressourcen wie HTML-Templates oder Bilder lassen sich auch unter diesen Knoten deponieren. \\
Der letzte Knoten, gezeigt in \autoref{line:component_start} bis \autoref{line:component_end}, stellt die \ajc dar und beinhaltet in \filefolder{aurelia-app.html} den Einstieg für die Webanwendung. Hier wird weiterhin der Aufruf für die Clientlib analog zum \autoref{lst:clientlib_useage} ausgelöst. Der Knoten in \autoref{line:editconfig} dient dazu, dass die Komponente in der Autoren-Bedienoberfläche genutzt und an ihre entsprechende Stelle platziert werden kann. \\
%Diese besteht neben den JavaScript- und CSS-Ressourcen aus einer HTML-Ressource, \autoref{line:html}, und einem cq:editConfig Knoten, \autoref{line:editconfig}. \\
%Im HTML befinden sich wie in \autoref{lst:clientlib_useage} entsprechende Anweisungen für Verwendung der Clientlib. Zusätzlich wird hier der Einsteig für die Webanwendung definiert. \\
%Durch das Hinzufügen des Dialog-Knoten erscheint die \ac{aem}-Komponente zur Auswahl in der Autoren-Bedienoberfläche. Fehlt dieser Knoten hat dies zur Folge, dass sich eine \ac{aem}-Komponente nur über die direkte Konfiguration des \ac{jcr} in eine Webseite einbinden lässt. \\
Um das Erstellen der beiden Clientlibs zu automatisieren, lässt sich die Grunt-Erweiterung \quotes{grunt-aem-clientlib-generator} \cite{wcmio2017} nutzen. Diese lässt sich dahingehend konfigurieren, die beiden Ordner \quotes{css} und \quotes{javascript} und die beiden Dateien \quotes{css.txt} und \quotes{js.txt} automatisiert zu erstellen.
\paragraph{Bewertung}

Die Lösung als Clientlib ist wie folgt zu bewerten.

\begin{minipage}[t]{0.5\textwidth}
	\textbf{Vorteile:}
	\begin{itemize}
		\item Lässt sich gut als \ac{crx}-Paket exportieren.
		\item Alles in einem System.
		\item Konfiguration über Autoren-Bedienoberfläche möglich.
	\end{itemize}
\end{minipage}
\begin{minipage}[t]{0.5\textwidth}
	\textbf{Nachteile:}
	\begin{itemize}
		\item Bestehende Verzeichnisstruktur muss abgeändert werden.
		\item Relative Pfade müssen angepasst werden.
		\item Wird etwas an der Webanwendung geändert, so muss diese auch in der \ajc geändert werden.
	\end{itemize}
\end{minipage}

\subsubsection{Lösungsansatz \quotes{Content Ordner}}
\label{sec:sol_content}
Eine simple Lösung wäre es, die Webanwendung unter den Content-Ordner  abzulegen. \\
Jeglicher Inhalt, der unter \filefolder{/content} abgelegt wird, ist direkt für den Client aufrufbar. Beispielsweise wäre unter der Standardkonfiguration von \ac{aem} die Datei \filefolder{/content/myapp/index.html} unter der \ac{url} \pseudourl{http://<Server-Hostname>:<AEM-Port>/content/myapp/index.html} erreichbar. Somit wäre es möglich, die Verzeichnisstruktur der Webanwendung 1:1 beizubehalten, sofern die komplette Webanwendung unter \filefolder{/content} abgelegt wird. Dieser Lösungsansatz verzichtet jedoch darauf, die Webanwendung als \ajc zu realisieren, was deren freie Positionierung und Konfigurierung über die Autoren-Bedienoberfläche ausschließt. Der Lösungsansatz \quotes{Content Ordner} empfiehlt sich somit nur für Webanwendungen, welche eine gesamte Webseite des \ac{aem} ausmachen und keine Konfiguration benötigen. 

\paragraph{Bewertung}

Die Lösung mit dem Content Ordner ist wie folgt zu bewerten.

\begin{minipage}[t]{0.5\textwidth}
	\textbf{Vorteile:}
	\begin{itemize}
		\item Beibehalten der Verzeichnisstruktur.
		\item Schnelle Integration.
		\item \ac{aot}-Ansatz möglich
	\end{itemize}
\end{minipage}
\begin{minipage}[t]{0.5\textwidth}
	\textbf{Nachteile:}
	\begin{itemize}
		\item Keine \ac{ajc}, somit auch kein einbetten in eine bestehende Webseite  über die Autoren-Bedienoberfläche des \ac{aem} möglich.
		\item Keine Konfiguration über die Autoren-Bedienoberfläche möglich.
	\end{itemize}
\end{minipage} 


\subsection{Integration über einen zusätzlichen Webserver}

Neben der Lagerung der App-Ressourcen in dem \ac{jcr} lässt sich diese ebenfalls auf einen zusätzlichen Webserver ablegen und über eine \ac{aem}-Komponente in eine Webseite des \ac{aem} integrieren.

\subsubsection{Lösungsansatz \quotes{Java-Servlet}}
\label{sec:ansatz-java}

Hier ist die Grundidee, dass sich die App-Ressourcen nicht im \ac{jcr} des \ac{aem} befinden, sondern auf einem separaten Webserver vorliegen. Ziel ist es, dass die Webanwendung unabhängig auf dem separaten Webserver zum Testen lauffähig ist. Über eine \ac{aem}-Komponente wird diese in das \ac{aem} importiert. Die \ac{aem}-Komponente bezieht die App-Ressourcen vom Webserver und passt diese ggf. an, so dass die Webanwendung korrekt in der Webseite des \ac{aem} dargestellt wird, wie in \autoref{img:import} illustriert wird. Da auch diese \ac{aem}-Komponente eine Webanwendung abstrahiert, wird diese im Folgenden ebenfals als \ajc bezeichnet.

\begin{figure}[H]
	\begin{center}
		\includesvg[width=1.0\textwidth]{import}
		\caption{Integration vom Webserver in den AEM Server mittels einer \ajc}
		\label{img:import}
	\end{center}
\end{figure}
Dabei werden die App-Ressourcen zur Laufzeit der \ajc, also wenn die Webseite geladen wird, jeweils erneut vom Webserver geladen. Ein Ablegen in das \ac{jcr} ist nicht vorgesehen. Ein temporäres Zwischenspeichern (Cachen) wäre von Grund auf denkbar ist, wird innerhalb dieser Arbeit jedoch nicht weiter behandelt. \\
Im Folgenden wird der separate Webserver, auf dem sich die App-Ressourcen für die Webanwendung befinden, kurz \quotes{Webserver} und der Server mit der lauffähigen \ac{aem}-Instanz kurz \quotes{\ac{aem}-Server} genannt. Die Integration erfolgt somit vom Webserver in den \ac{aem}-Server. Die Webanwendung ist über \serverB aufrufbar und soll später über den \ac{aem}-Server über \serverA erreichbar sein.\\

\paragraph{Erklärung}

In der Regel erfolgt das Laden einer Webanwendung genau wie bei einer Webseite, wie in \autoref{sec:http} beschrieben, in drei Schritten. Zunächst wird die HTML-Seite angefordert, anschließend wird der Quellcode der HTML-Seite nach weiteren App-Ressourcen durchsucht und auch diese werden geladen. Weitere App-Ressourcen werden nun über Ajax angefordert und vom Server retourniert. Es wird davon ausgegangen, dass diese Schritte unter \serverB korrekt durchlaufen werden und somit die Webanwendung wie gewünscht im Browser erscheint. \\
Nun ist es Aufgabe der \ajc, die App-Ressourcen so aufzurufen und zu bearbeiten, dass die Webanwendung korrekt in einer Webseite des \ac{aem} dargestellt wird. Dafür müssen die HTML-Seite, die Ajax-Anfragen und ggf. auch weitere App-Ressourcen manipuliert werden. Die \ajc dient somit als Bindeglied zwischen \ac{aem} und dem Webserver.\\
Zuvor muss jedoch der Webserver und die Webanwendung gewissen Anpassungen unterliegen. Diese werden in \autoref{sec:anpassen-des-webservers} und \autoref{sec:anpassen-der-webanwendung} beschrieben.
\paragraph{Anpassen des Webservers}
\label{sec:anpassen-des-webservers}
Gängige Webbrowser verbieten, dass JavaScript und somit auch \ac{ajax} auf Web-Ressourcen anderer Webserver zugreifen dürfen, als dem, unter dem das Skript gerade ausgeführt wird. Grund dafür ist das Sicherheitskonzept der \acf{sop}\index{SOP}. Die Herkunft (eng. Origin) setzt sich hierbei aus dem verwendetem Protokoll, Hostname und Port der \ac{url} zusammen. Unterscheidet sich die Herkunft von der Webseite, von der aus die Anfrage gestartet wurde, und von der angefragten Web-Ressource, so wird die Anfrage vom Webbrowser geblockt. Im zuvor beschriebenen Beispiel würden sich \serverA und \serverB in dem Hostnamen (\pseudourl{www.example.com} anstelle von \pseudourl{aem.example.com}) und dem Port (\pseudourl{3000} anstelle vom Standardport für HTTP \pseudourl{80}) unterscheiden. Eine Ajax-Anfragen vom AEM-Server an den Webserver würde einen Fehler erzeugen.\\
Damit der Zugriff doch funktioniert, lässt sich der von W3C definierte Mechanismus des \ac{cors} anwenden.

\begin{figure}[H]
	\begin{center}
		\includesvg[width=1\textwidth]{cors}
		\caption{Ablauf bei CORS}
		\label{img:cors}
	\end{center}
\end{figure}
Wie in \autoref{img:cors} sendet der Browser des Clients bei seiner Anfrage seine Herkunft (Origin) mit. Am Beispiel hier wäre die Herkunft die \ac{url} der Webseite, von der die Anfrage ausgeführt wird, also \serverAN. Der Server antwortet mit einer Liste von erlaubten Hostnamen und der angeforderten Web-Ressource. Der Browser überprüft nun, ob sich die Herkunft der Ajax-Anfrage in der Liste befindet. Im Erfolgsfall wird die Ajax-Anfrage erfolgreich beendet, ansonsten wird eine Fehlermeldung erzeugt. \\
Neben der gezeigten Variante lässt sich \ac{cors}\index{CORS} auch mit anderen Zugriffsregeln verwenden. Durch \inlinecode{Access-Control-Allow-Origin: *} werden alle Anfragen genehmigt, durch \inlinecode{Access-Control-Allow-Methods: GET} werden alle mit der Zugriffsmethode GET gestellten Anfragen erfolgreich ausgeführt \cite{W3C2014a}. \\
Die Konfiguration der erlaubten Herkünfte ist Webserverspezifisch und in der jeweiligen Dokumentation nachzulesen. Für Apache kann hier das Modul \quotes{mod\_headers} \cite{Foundation2017} genutzt werden.
\paragraphtbd{Anpassen der Webanwendung}
\label{sec:anpassen-der-webanwendung}
Auch die Webanwendung muss unter Umständen noch vor dem Produktiveinsatz konfiguriert werden. \\
Sollten HTML-Templates in AngularJS Verwendung finden, kann es hier bei dem Ladevorgang zu einer \ac{sce}-Fehlermeldung kommen. Dies geschieht, da der HTML-Code von extern kommt und potenziell unsichereren Code mitführen kann. AngularJS blockt daher per Standardkonfiguration den Ladevorgang. Doch durch entsprechende Anpassungen der Webanwendung lassen sich Ausnahmen hinzufügen oder das \ac{sce} auch vollständig deaktivieren \cite{Google2016d}.
\missing{Andere Frameworks?}

\paragraph{Ablauf}
Der ungefähre Ablauf ist \autoref{img:java} zu entnehmen. Dieser besteht grob zusammengefasst aus zwei Schritten. Der erste Schritt wird im folgenden \autoref{sec:html-seite-aufbereiten}, der zweite Schritt im darauf folgenden \autoref{sec:ajax-anfragen-anpassen} erklärt. \\
Die beiden Java-Ressourcen \filefolder{AppBuilder.java} und \filefolder{GetData.java} lassen sich mit einem OSGi-Bundle für das \ac{aem} bereitstellen. Die JavaScript-Ressource \filefolder{ajaxdirect.js} lässt sich in einer \ajc in Verbindung mit einer Clientlib ausliefern. 
Bei \filefolder{GetData.java} handelt es sich um ein Servlet. Das bedeutet, dass ein Browser dies über eine bestimmte \ac{url} aufrufen kann und eine entsprechende Antwort erhält.

\begin{figure}[H]
	\begin{center}
		\includegraphics[width=1\textwidth]{servlet.png}
		\caption{Ungefährer Ablauf der AEM-Komponente auf Java-Basis}
		\label{img:java}
	\end{center}
\end{figure}

\input{inhalt/integration-der-webframeworks-in-aem/moeglichkeiten-der-integration/integration-über-einen-zusätzlichen-webserver/ansatz-java/ablauf-der-aem-komponente/HTML-Seite-aufbereiten}
\input{inhalt/integration-der-webframeworks-in-aem/moeglichkeiten-der-integration/integration-über-einen-zusätzlichen-webserver/ansatz-java/ablauf-der-aem-komponente/Ajax-Anfragen-anpassen}

\paragraph{Bewertung}

Die Lösung mit Java ist wie folgt zu bewerten.

\begin{minipage}[t]{0.5\textwidth}
	\textbf{Vorteile:}
	\begin{itemize}
		\item Die Struktur der Webanwendung bleibt erhalten.
		\item Die Webanwendung ist unabhängig von \ac{aem} testbar.
		\item Die Entwickelte \ajc lässt sich für ähnliche Webanwendungen wiederverwenden.
	\end{itemize}
\end{minipage}
\begin{minipage}[t]{0.5\textwidth}
	\textbf{Nachteile:}
	\begin{itemize}
		\item Betreiben zweier Dienste nötig (AEM und Webserver).
		\item Ggf. zusätzliche Konfiguration von Webserver und Webanwendung nötig.
		\item Zusätzliche Laufzeit (gering, ggf. vernachlässigbar).
		\item Neue \ajc für stark abweichende Webanwendungen nötig.
	\end{itemize}
\end{minipage}

\subsubsection{Lösungsansatz \quotes{JavaScript}}
\label{sec:ansatz-javascript}
Dieser Lösungsansatz verzichtet auf die beiden Java-Ressourcen vom Lösungsansatz "Java-Servlet" und erfüllt das gewünschte Ziel lediglich mit JavaScript, welche sich ebenfalls in einer \ac{aem}-Komponente befindet. Im Gegensatz zur zuvor vorgestellten Variante wird nun der erste Schritt von \autoref{img:java} durch eine JavaScript-Komponente ersetzt. Das Prinzip ist hierbei das gleiche, jedoch gibt es hier die Besonderheit, dass es die zu ladenden App-Ressourcen zu synchronisieren gilt.
%Und zwar ist darauf zu achten, dass die in der HTML-Seite referenzierten App-Ressourcen in der korrekten Reihenfolge geladen werden. \\
%Der Ablauf beginnt wieder damit, dass zunächst die reine HTML-Seite geladen wird. Auch diese wird wieder nach JavaScript-Ressourcen durchsucht, die es zu laden gibt. 

\paragraph{Synchronisation}
Eine JavaScript-Ressource nachträglich zu laden lässt sich erreichen, indem mittels JavaScript ein \inlinecode{script}-Element in den \inlinecode{head} eingefügt wird. Dies würde in etwa wie in \autoref{lst:appendchild} aussehen.

\begin{lstlisting}[style=htmlcssjs, caption=Nachladen einer JavaScript-Ressource, label=lst:appendchild]
var s = document.createElement('script');
s.src = "http://www.example.com/jquery.js";
document.head.appendChild(s);
\end{lstlisting}

Wird dieser Code jedoch wiederholt mit unterschiedlichen URLs aufgerufen, so können Abhängigkeitsproblemen auftreten, da mehrere JavaScript-Ressourcen nahezu gleichzeitig geladen und ausgeführt werden. Somit kann es passieren, dass eine JavaScript-Datei, die eine Webanwendung beinhaltet, vor dem Webframework ausgeführt wird, obwohl die Reihenfolge anders angegeben wurde.
Für die Lösung dieses Problems lässt sich die Promise-Schnittstelle verwenden, wie nachfolgendes Codebeispiel zeigt.

\begin{lstlisting}[style=htmlcssjs, caption=Laden und Synchronisieren von JavaScript-Ressourcen, label=lst:js-syncro,escapechar=|]
var i = 0, scriptUrls = ...;
getScript();
function getScript() {
  loadScript(scriptUrls[i++]).then(getScript); |\label{line:load}|
}

function loadScript(src) {
  return new Promise(function (resolve) {
    var s = document.createElement('script');
    s.src = src;
    s.onload = resolve;
    document.head.appendChild(s);
  });
}
\end{lstlisting}

Es wird davon ausgegangen, dass die Variable \inlinecode{scriptUrls} bereits ein Array mit den absoluten Adressen der zu ladenden JavaScript-Ressourcen in der korrekten Reihenfolge beinhaltet. Diese wurden beispielsweise zuvor in der HTML-Seite durch den Einsatz von regulären Ausdrücken gefunden. Anschließend wird die Funktion \inlinecode{getScript} aufgerufen. Die \autoref{line:load} bewirkt nun zwei Dinge. Zunächst wird die erste JavaScript-Ressource geladen und ausgeführt. Anschließend wird, sobald der JavaScript-Code ausgeführt wurde, die Funktion erneut aufgerufen, um die nächste JavaScript-Ressource zu laden. \\
Damit nicht fälschlicherweise eine JavaScript-Ressource geladen wird, bevor die vorherige nicht zum Ende ausgeführt wurde, wird hier ein Promise (eng. Versprechen) verwendet, einen in ECMAScript 6 eingeführten Standard. Die Funktion bewirkt, dass die JavaScript-Ressource geladen und ausgeführt wird. Anschließend bewirkt der Promise, dass die Funktion \inlinecode{loadScript} innerhalb von \inlinecode{then(...)} in \autoref{line:load} ausgeführt wird.
\paragraph{Bewertung}

Die Lösung mit JavaScript ist wie folgt zu bewerten.

\begin{minipage}[t]{0.5\textwidth}
	\textbf{Vorteile:}
	\begin{itemize}
		\item Verzicht von Java, somit erfolgt die Integration ausschließlich mit JavaScript.
	\end{itemize}
\end{minipage}
\begin{minipage}[t]{0.5\textwidth}
	\textbf{Nachteile:}
	\begin{itemize}
		\item Zusätzliche Ladezeit, da gerade bei den Skripten auf den richtigen Ablauf geachtet werden muss.
	\end{itemize}
\end{minipage}

\subsection{Lösungsansatz \quotes{Proxy}}
\label{sec:sol_proxy}
Ein weiterer Lösungsansatz wird durch Hinzunahme eines Proxy realisierbar. Durch den Einsatz eines Proxy lassen sich die vom Client angeforderten App-Ressourcen während des HTTP-Request-Response-Zykluses manipulieren.\\
Im allgemeinen ist ein Proxy\index{Proxy} ein Rechner, der als Mittelsmann zwischen Client und Server dient. Der Client stellt hierbei die Anfragen an den Proxy. Der Proxy leitet anschließend die Anfrage weiter an den Server. Client und Server kommunizieren somit nie direkt miteinander, sondern nur indirekt über den Proxy.\\
Im Rahmen dieser Arbeit handelt es sich bei einem Proxy um einen weiteren speziell konfigurierten Webserver. Die Web-Ressourcen werden dabei von dem eigentlichen Webserver über den Proxy an den Client ausgeliefert. Die angefragten App-Ressourcen lassen sich hierbei über den Proxy manipulieren. 

\subsubsection{Erklärung}
Die Lösung sieht hierbei vor, dass ein weiterer Webserver als Proxy unter dem Hostnamen \hostnameP~ so angepasst wird, dass dieser alle Anfragen wie in \autoref{img:proxy} über ein zentrales Skript umleitet. Besagtes Skript kann in einer beliebigen, vom Proxy unterstützten Programmiersprache realisiert werden, wie hier in \acs{php}. Die Konfiguration ließe sich beim Apache HTTP-Server mit dem Modul \quotes{mod\_rewrite} realisieren \cite{Foundation2017a}. Die entsprechende Konfiguration ist \autoref{lst:rewrite} zu entnehmen. \\
\begin{minipage}{\textwidth}
\begin{lstlisting}[style=jcr, caption=Konfiguration des mod\_rewrite Modules für den Proxy-Lösungsansatz, label=lst:rewrite]
RewriteEngine on
RewriteRule ^/app/(.*)$ /proxy.php?url=$1 [L]
\end{lstlisting}
\end{minipage}
Der Ablauf könnte hierbei wie in \autoref{img:proxy} erfolgen. \\
Die Konfiguration definiert, dass, sofern eine App-Ressource angefragt wird, deren Pfad mit \filefolder{/app/} beginnt, die Anfrage über \filefolder{proxy.php} umgeleitet wird. Der restliche Pfad, hier \filefolder{templates/news.html}, wird dem Skript als Parameter mit übergeben.\\
Dieses ermittelt nun die benötigte App-Ressource vom eigentlichen Webserver, der unter dem Hostnamen und Port \pseudourl{www.exmaple.com:3000} zu erreichen ist. Etwaige Abweichungen vom angeforderten Pfad und der Pfad auf dem Webserver lassen sich über das Skript wie im Beispiel bewerkstelligen und korrigieren.\\
Anschließend bearbeitet das Skript die App-Ressource entsprechend und liefert die bearbeitete App-Ressource als Ergebnis zurück. Diese Aktionen bleiben dem Client verborgen. Aus dessen Sicht wird ihm lediglich die angeforderte App-Ressource geliefert. \\

\begin{figure}[H]
	\begin{center}
		{\footnotesize\includesvg[width=1.0\textwidth]{proxy}}
		\caption{Funktionsweise des Proxy}
		\label{img:proxy}
	\end{center}
\end{figure}

Die Lösung ähnelt dem Servlet aus \autoref{sec:ansatz-java}, da auch hier zur Laufzeit Änderungen vorgenommen werden können.\\
Seitens \ac{aem} kommt auch hier eine \ac{aem}-Komponente zum Einsatz, welche die initiale Anfrage an die HTML-Seite der Webanwendung stellt, was in etwa dem ersten Schritt aus \autoref{img:java} entspräche. Die Logik der \ajc vom Lösungsansatz \quotes{Java-Servlet} wird hier auf das Proxy-Skript ausgelagert. Innerhalb der \ac{aem}-Komponente wird lediglich die Start-URL des Proxy definiert.
\subsubsection{Bewertung}
Die Lösung mit einem Proxy ist wie folgt zu bewerten.

\begin{minipage}[t]{0.5\textwidth}
	\textbf{Vorteile:}
	\begin{itemize}
		\item Die Struktur der Webanwendung bleibt erhalten.
		\item Die Webanwendung ist unabhängig von \ac{aem} testbar.
		\item Die entwickelte \ajc lässt sich wiederverwenden.
		\item Die Logik des Skriptes ist weitestgehend frei wählbar und nicht von \ac{aem} abhängig.
	\end{itemize}
\end{minipage}
\begin{minipage}[t]{0.5\textwidth}
	\textbf{Nachteile:}
	\begin{itemize}
		\item Betreiben von zwei bzw. drei Diensten nötig (AEM und Webserver und Proxy).
		\item Ggf. zusätzliche Konfiguration von Webserver und Webanwendung nötig.
		\item Zusätzliche Laufzeit (gering, ggf. vernachlässigbar).
	\end{itemize}
\end{minipage}