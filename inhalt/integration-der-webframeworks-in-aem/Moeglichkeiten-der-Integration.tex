\section{Möglichkeiten der Integration}
\label{sec:integrationen}
Abhängig von verschiedenen Faktoren wurden unterschiedliche Möglichkeiten erarbeitet um eine Webanwendung in eine \ac{aem}-Instanz zu integrieren. Neben dem zugrunde liegenden clientseitigen Webframework ist der wichtigste Entscheidungsfaktor dafür, welche Lösung genutzt werden kann, die IT-Landschaft der Zielplatz. Je nach dessen Konfiguration bieten sich gewisse Lösungen mehr an, oder sind auch als mögliche Lösung von vorne herein ausgeschlossen.
%Insbesondere das Thema Zugriffsrechte spielt hier eine immense Rolle. Diese könnten so geregelt sein, dass ein zu tiefer Eingriff in das \ac{aem} und eine freie Konfiguration des Servers nicht gestattet sind.

\subsection{Direkte Integration in das AEM}
\label{sec:direkte-integration-in-das-aem}
Innerhalb dieses Abschnittes sollen Lösungsansätze dargestellt werden, welche die direkte Integration in die \ac{jcr}-Struktur betreffen. Das bedeutet, dass alle benötigten App-Ressourcen in die \ac{jcr} abgelegt werden. Somit sind alle Ressourcen, die für die Ausführung der Webanwendung vonnöten sind, über das \ac{aem} erreichbar.\\
%Sofern der Zugriff auf das Ziel-\ac{aem}, insbesondere auf dessen \ac{jcr}-Strukur, besteht, lässt sich die Webanwendung direkt in selbiges durch die Verwendung von Clientlibs integrieren.\\
%Im ersten Schritt wird das gewünschte JavaScript-Framework als Clientlib, wie in \autoref{sec:clientlib} beschrieben, bereitgestellt. Sollte das Framework mehrmals innerhalb von \ac{aem} Verwendung finden, empfiehlt es sich dies unter \filefolder{/etc/clientlibs} zu platzieren, um global zur Verfügung zu stehen. Ansonsten wird das JavaScript-Framework in die zu entwickelnde \ajc abgelegt. \\
Um die Bereitstellung der App-Ressourcen für die Webanwendung zu ermöglichen, wurden zwei Lösungsansätze erarbeitet. In der ersten Lösung werden alle App-Ressourcen in eine Clientlib gepackt. Die zweite sieht als Lösung vor, diese unter dem Content-Ordner abzulegen.

\subsubsection{Lösungsansatz \quotes{Clientlib}}
\label{sec:sol_clientlib}

Der erste Lösungsansatz beschäftigt sich mit der Realisierung als Clientlib.

%\paragraphtbd{Transferieren der App-Ressourcen in das AEM}
Zunächst lässt sich über die \ac{webdav}-Schnittstelle mit einem entsprechenden Programm die Struktur des \ac{jcr} anzeigen und bearbeiten. Bei \acf{webdav}\index{WebDAV} handelt es sich um einen offenen Standard für die Bereitstellung von Daten über ein Netzwerk. Neben dem Löschen und Verschieben von Knoten ist hier auch das Hochladen von lokalen Dateien möglich. \\
Sofern die \ac{webdav}-Schnittstelle auf der Zielplattform nicht geöffnet sein sollte, gibt es die Alternative die App-Ressourcen als \ac{crx}-Paket bereit zu stellen. Hierfür werden zunächst wieder alle Inhalte in das \ac{jcr}, zum Beispiel über \ac{webdav}, geladen, dieses mal jedoch in die der Entwicklungsumgebung. Anschließend kann über die Autoren-Bedienoberfläche, siehe \autoref{sec:autor_ui}, ein \ac{crx}-Paket erstellt werden. Dieses ist ein Zip-Archiv mit Metainformationen, wie Name und Version des \ac{crx}-Pakets.\\
Das Erstellen eines \ac{crx}-Paketes erfolgt über eine entsprechende Seite des CRXDE Lite. Hier werden besagte Metainformationen gesetzt und alle benötigten Knoten für das \ac{crx}-Paket\index{CRX-Paket} angegeben. Nun erfolgt der Export des \ac{crx}-Pakets. Ist dies geschehen wird das \ac{crx}-Paket in die Zielplattform importiert. Der Inhalt des \ac{crx}-Paketes wird nun in das \ac{jcr} geschrieben. Bereits bestehende Knoten werden bei diesem Vorgang ggf. überschrieben.
\paragraph{Anpassen zur AJC}
Zunächst empfiehlt es sich, die App-Ressourcen vor der Anpassung zur \ajc aufzubereiten. Gerade das Konkatenieren der Ressourcen erleichtert die spätere Arbeit ungemein. Dieser Vorgang wird in \autoref{sec:konkat} erklärt. Auch sollte ab hier auf den \ac{aot}-Ansatz verzichtet und dafür der \ac{jit}-Ansatz bevorzugt werden. Beim \ac{aot} werden die Module in jeweils einzelne Dateien aufgeteilt, wohingegen bei \ac{jit} die Module zu einer Datei zusammengeschlossen werden können. \\
\autoref{lst:ajc-clientlib} zeigt, wie sich eine \ajc mit Clientlib zusammensetzen könnte.

\begin{lstlisting}[style=jcr, caption=Eine Beispiel-AJC mit Clientlib, label=lst:ajc-clientlib,escapechar=|]
/apps/integration/components/ |\label{line:component_start}|
  + aurelia-app
    + aurelia-app.html
    + cq:editConfig|\label{line:editconfig}|
      - cq:actions (String[])  = [edit, -, delete, insert] |\label{line:actions}| |\label{line:component_end}|

/etc/design/integration/aurelia-example/ |\label{line:clientlib_app_start}|
  + clientlib
    - jcr:primaryType (Name) = cq:ClientLibrary
    - catagories (String[]) = integration.aurelia.example
    - dependencies (String[]) = aurelia |\label{line:dependencies}|
    + css
      + style.css
    + css.txt
    + javascript
      + app.aurelia.js
    + js.txt
    + resources
      + templates
        + app.html
      + images
        + logo.png |\label{line:clientlib_app_end}|
    
/lib/clientlibs/ |\label{line:clientlib_start}|
  + aurelia
  	+ categories (Sting[]) = aurelia |\label{line:categorie_aurelia}|
    	+ javascript
      	+ aurelia.min.js
    	+ js.txt |\label{line:clientlib_end}|
\end{lstlisting}

Das gezeigte Beispiel stellt eine Aurelia-Anwendung dar und ist in drei Teile aufgeteilt. Der Knoten ab \autoref{line:clientlib_start}  beinhaltet die Ressourcen für das Webframework, hier Aurelia. Wichtig ist dabei, diese mindestens einer entsprechenden Kategorie, wie in \autoref{line:categorie_aurelia} zu sehen, zuzuordnen. \\
Alle weiteren App-Ressourcen werden unter den Knoten ab \autoref{line:clientlib_app_start} abgelegt. In \autoref{line:dependencies} werden die benötigten Abhängigkeiten zu anderen Clientlibs eingetragen. Diese werden anhand der jeweiligen Kategorie aufgelöst, wie diese im Beispiel in \autoref{line:categorie_aurelia} eingetragen wurde. Weitere App-Ressourcen wie HTML-Templates oder Bilder lassen sich auch unter diesen Knoten deponieren. \\
Der letzte Knoten, gezeigt in \autoref{line:component_start} bis \autoref{line:component_end}, stellt die \ajc dar und beinhaltet in \filefolder{aurelia-app.html} den Einstieg für die Webanwendung. Hier wird weiterhin der Aufruf für die Clientlib analog zum \autoref{lst:clientlib_useage} ausgelöst. Der Knoten in \autoref{line:editconfig} dient dazu, dass die Komponente in der Autoren-Bedienoberfläche genutzt und an ihre entsprechende Stelle platziert werden kann. \\
%Diese besteht neben den JavaScript- und CSS-Ressourcen aus einer HTML-Ressource, \autoref{line:html}, und einem cq:editConfig Knoten, \autoref{line:editconfig}. \\
%Im HTML befinden sich wie in \autoref{lst:clientlib_useage} entsprechende Anweisungen für Verwendung der Clientlib. Zusätzlich wird hier der Einsteig für die Webanwendung definiert. \\
%Durch das Hinzufügen des Dialog-Knoten erscheint die \ac{aem}-Komponente zur Auswahl in der Autoren-Bedienoberfläche. Fehlt dieser Knoten hat dies zur Folge, dass sich eine \ac{aem}-Komponente nur über die direkte Konfiguration des \ac{jcr} in eine Webseite einbinden lässt. \\
Um das Erstellen der beiden Clientlibs zu automatisieren, lässt sich die Grunt-Erweiterung \quotes{grunt-aem-clientlib-generator} \cite{wcmio2017} nutzen. Diese lässt sich dahingehend konfigurieren, die beiden Ordner \quotes{css} und \quotes{javascript} und die beiden Dateien \quotes{css.txt} und \quotes{js.txt} automatisiert zu erstellen.
\paragraph{Bewertung}

Die Lösung als Clientlib ist wie folgt zu bewerten.

\begin{minipage}[t]{0.5\textwidth}
	\textbf{Vorteile:}
	\begin{itemize}
		\item Lässt sich gut als \ac{crx}-Paket exportieren.
		\item Alles in einem System.
		\item Konfiguration über Autoren-Bedienoberfläche möglich.
	\end{itemize}
\end{minipage}
\begin{minipage}[t]{0.5\textwidth}
	\textbf{Nachteile:}
	\begin{itemize}
		\item Bestehende Verzeichnisstruktur muss abgeändert werden.
		\item Relative Pfade müssen angepasst werden.
		\item Wird etwas an der Webanwendung geändert, so muss diese auch in der \ajc geändert werden.
	\end{itemize}
\end{minipage}

\subsubsection{Lösungsansatz \quotes{Content Ordner}}
\label{sec:sol_content}
Eine simple Lösung wäre es, die Webanwendung unter den Content-Ordner  abzulegen. \\
Jeglicher Inhalt, der unter \filefolder{/content} abgelegt wird, ist direkt für den Client aufrufbar. Beispielsweise wäre unter der Standardkonfiguration von \ac{aem} die Datei \filefolder{/content/myapp/index.html} unter der \ac{url} \pseudourl{http://<Server-Hostname>:<AEM-Port>/content/myapp/index.html} erreichbar. Somit wäre es möglich, die Verzeichnisstruktur der Webanwendung 1:1 beizubehalten, sofern die komplette Webanwendung unter \filefolder{/content} abgelegt wird. Dieser Lösungsansatz verzichtet jedoch darauf, die Webanwendung als \ajc zu realisieren, was deren freie Positionierung und Konfigurierung über die Autoren-Bedienoberfläche ausschließt. Der Lösungsansatz \quotes{Content Ordner} empfiehlt sich somit nur für Webanwendungen, welche eine gesamte Webseite des \ac{aem} ausmachen und keine Konfiguration benötigen. 

\paragraph{Bewertung}

Die Lösung mit dem Content Ordner ist wie folgt zu bewerten.

\begin{minipage}[t]{0.5\textwidth}
	\textbf{Vorteile:}
	\begin{itemize}
		\item Beibehalten der Verzeichnisstruktur.
		\item Schnelle Integration.
		\item \ac{aot}-Ansatz möglich
	\end{itemize}
\end{minipage}
\begin{minipage}[t]{0.5\textwidth}
	\textbf{Nachteile:}
	\begin{itemize}
		\item Keine \ac{ajc}, somit auch kein einbetten in eine bestehende Webseite  über die Autoren-Bedienoberfläche des \ac{aem} möglich.
		\item Keine Konfiguration über die Autoren-Bedienoberfläche möglich.
	\end{itemize}
\end{minipage} 


\subsectiontbd{Ansatz Java}
\label{sec:ansatz-java}
Hier ist die Grundidee, dass die Webanwendung sich nicht im \ac{aem} befindet, sondern sich auf einen separaten Webserver liegt. Ziel ist es, dass die Webanwendung autonom auf dem separaten Webserver lauffähig ist und über eine \ac{aem}-Komponente auf Basis von Java und JavaScript in das \ac{aem} integriert wird.\\
Im Folgenden wird der separate Webserver, auf dem sich die App-Ressourcen für die Webanwendung befinden, kurz Webserver, und der Server mit der lauffähigen \ac{aem}-Instanz kurz \ac{aem}-Server genannt. Die Integration erfolgt somit vom Webserver in den \ac{aem}-Server. Die Webanwendung ist über \serverB aufrufbar und soll später über den \ac{aem}-Server über \serverA erreichbar sein.\\

\subsubsectiontbd{Erklärung}

In der Regel erfolgt das Laden einer Webanwendung wie in \autoref{sec:http} beschrieben in drei Schritten. Zunächst wird die HTML-Seite angefordert, anschließend wird der Quellcode der HTML-Seite nach weiteren App-Ressourcen durchsucht und auch diese geladen. Weitere App-Ressourcen werden nun über Ajax angefordert und vom Server retourniert. \\
Es wird davon ausgegangen, dass die Schritte unter \serverB korrekt durchlaufen werden und somit die Webanwendung wie gewünscht im Browser erscheint. \\
Innerhalb von \ac{aem} hat die \ac{aem}-Komponente die Aufgabe die Webanwendung vom Webserver zu laden. Dabei muss jedoch die HTML-Seite, die Ajax-Anfragen und ggf. auch weitere App-Ressourcen manipuliert werden. Die Komponente dient somit als Bindeglied zwischen \ac{aem} und dem Webserver.\\
Zuvor muss jedoch der Webserver und die Webanwendung gewissen Anpassungen unterliegen. Diese werden in \autoref{sec:anpassen-des-webservers} und \autoref{sec:anpassen-der-webanwendung} beschrieben.
\subsubsectiontbd{Anpassen des Webservers}
\label{sec:anpassen-des-webservers}
Gängige Webbrowser verbieten, dass JavaScript, und somit auch \ac{ajax}, auf externe Webserver zugreifen dürfen. Grund dafür ist das Sicherheitskonzept der \ac{sop}. \\
Unterscheiden sich die \ac{url} von der Webseite von der aus die Anfrage gestartet wurde und vom Ziel in Protokoll, Domain oder Port, so wird die Anfrage vom Webbrowser geblockt. Im zuvor beschriebenen Beispiel würden sich \serverA und \serverB im Host (www.example.com anstelle von aem.example.com) und dem Port (3000 anstelle von 80) unterscheiden. \\
Damit der Zugriff doch funktioniert lässt sich der Mechanismus des \ac{cors} verwenden.
\missing{CORS etwas detailierter erklären}
Hierfür antwortet der Zielwebserver der Anfrage mit einen entsprechenden HTTP-Header. Hier sind alle URLs gelistet, von denen gestattet sind eine Anfrage zu auszuführen. Der Browser vergleicht jetzt den HTTP-Header mit der URL, von dem die Anfrage gestartet wurde. Im Erfolgsfall blockiert der Browser nicht und der Aufruf wird zu Ende ausgeführt.
\subsubsectiontbd{Anpassen der Webanwendung}
\label{sec:anpassen-der-webanwendung}
Auch die Webanwendung muss unter Umständen noch vor dem Produktiveinsatz konfiguriert werden. \\
Sollten HTML-Templates in AngularJS Verwendung finden, kann es hier bei dem Ladenvorgang zu einer \ac{sce}-Fehlermeldung führen. Dies geschieht, da der HTML-Code von extern kommt und potenziell unsichereren Code mitführen kann. AngularJS blockt daher per Standardkonfiguration den Ladevorgang. Doch durch entsprechende Anpassungen der Webanwendung lassen sich Ausnahmen hinzufügen, oder das \ac{sce} auch vollständig deaktivieren \cite{Google2016d}.
\missing{Andere Frameworks?}

\subsubsectiontbd{Ablauf der AEM-Komponente}
Den ungefähren Ablauf der \ac{aem}-Komponente ist \autoref{img:java} zu entnehmen.
\begin{figure}[H]
	\begin{center}
		\includegraphics[width=1\textwidth]{servlet.png}
		\caption{Ungefährer Ablauf der AEM-Komponente auf Java-Basis}
		\label{img:java}
	\end{center}
\end{figure}

Der erste Schritt wird im folgenden \autoref{sec:html-seite-aufbereiten}, der zweite Schritt im darauffolgenden \autoref{sec:ajax-anfragen-anpassen} erklärt.

\paragraphtbd{HTML-Seite aufbereiten}
\label{sec:html-seite-aufbereiten}

Der Benutzer fordert zunächst die Seite \serverA an. In dieser befindet sich die \ac{aem}-Komponente, welche zuvor so konfiguriert wurde, dass an dessen Stelle die Webanwendung von \serverB erscheinen soll. Nun wird \filefolder{AppBuilder.java} damit beauftragt den HTML-Quellcode der Einstiegsseite der Webanwendung anzufordern, welche sich unter \serverB[index.html] befindet. \filefolder{AppBuilder.java} muss jetzt den Quellcode für die Integration aufbereiten. Einige dieser Aufbereitungsaufgabe sind Anwendungsspezifisch und vom jeweils verwendeten Webframework abhängig. Mehr zu den Unterschieden ist dem \autoref{sec:aufbereitungsaufgabe} zu entnehmen. Was immer geschehen muss, ist, dass relative Referenzen auf JavaScript und CSS-Dateien durch absolute ersetzt werden. Eine Referenz auf \pseudourl{./app.module.js} sollte nach besagten Schritt auf \serverB[app.module.js] verweisen.\\
Der Quellcode von \serverB[index.html] könnte wie in \autoref{lst:serverB} aussehen.

\begin{lstlisting}[style=htmlcssjs, caption=Ausgangssituation auf Server B, label=lst:serverB]
<!DOCTYPE html>
<html ng-app="phonecatAPP">
	<head>
		<link rel="stylesheet" href="app.css" />
		<script src="bower_components/angular/angular.js"></script>
		<script src="app.module.js"></script>
		<title>My Webbapplication</title>
	</head>
	<body>
		<div class="view-container">
			<div ng-view class="view-frame"></div>
		</div>
	</body>
</html>
\end{lstlisting}

Aus den Aufbereitung würde ein Quellcode wie in \autoref{lst:serverA} resultieren.

\begin{lstlisting}[style=htmlcssjs, caption=Aufbereiteter Quellcode, label=lst:serverA]
<link rel="stylesheet" href="http://www.example.com:3000/spa/app.css" />
<script src="http://www.example.com:3000/spa/bower_components/angular/angular.js"></script>
<script src="http://www.example.com:3000/spa/app.module.js"></script>
<div ng-app="phonecatApp">
	<div class="view-container">
		<div ng-view class="view-frame"></div>
	</div>
</div>

\end{lstlisting}

\ac{aem} ersetzt nun den Platzhalter der Komponente mit den aufbereiteten Quellcode. Diese HTML-Seite ist nun fertig gerendert und wird wieder an den Client geliefert. \\
\filefolder{AppBuilder.java} nutzt die Java-Bibliothek jsoup \cite{Hedley2016}. Dieser HTML-Parser bietet Klassen und Funktionen zum Laden, Traversieren und Manipulieren von HTML-Seiten an. Der \ac{dom} der geladenen \ac{html}-Seite kann mithilfe von \ac{css}-Selektoren nach HTML-Elementen, wie den script- und link-Elementen, durchsucht werden. Die Installation der Bibliothek zur Nutzung innerhalb von \ac{aem} erfolgt als OSGi-Bundle.
\paragraphtbd{Ajax-Anfragen anpassen}
\label{sec:ajax-anfragen-anpassen}

Im zweiten Schritt werden Ajax-Anfragen umgeleitet. Diese sind zumeist ebenfalls relativ und würden somit versuchen eine App-Ressource unter \serverA anfordern, die sich jedoch unter \serverB befindet. Gerade \ac{di}-Bibliotheken laden läufig Skripte nach. \\
Für die Umleitung überschreibt \filefolder{ajaxredirect.js} das \ac{xhr}-Objekt. Die \ac{xhr}-Schnittstelle wird vom \ac{w3c} spezifiziert und ist ein wesentlicher Bestandteil von \ac{ajax}, da hiermit Daten asynchron zwischen Server und Client auszutauschen. \\
Das überschriebene \ac{xhr}-Objekt überprüft ob es sich bei der angeforderten \ac{url} um eine relative oder absolute handelt. Falls diese relativ ist hat dies zu bedeuten, dass die App-Ressource sich unter \serverB befindet. In dem Fall wird die Anfrage nun über ein Servlet, hier mit dem Namen \filefolder{GetData.java}, umgeleitet. Die Umleitung über das Servlet anstelle einer direkten Anfrage an den Webserver hat zwei Vorteile. \\
Zum einen werden hier Probleme mit Zugriffsrechten, wie das in \autoref{sec:anpassen-des-webservers} erläuterte Sicherheitskonzept \ac{sop}, erleichtert. Ohne Servlet würde der Client direkt auf Server-Ressourcen des Webservers zugreifen, was in der Regel zu einem Fehler führt. Der Webserver müsste somit allen Clienten den Zugriff autorisieren. Über ein Servlet jedoch ist es so, dass der \ac{aem}-Server die Anfrage an den Webserver stellt. Somit kann der Webserver so konfiguriert werden, dass er nur Anfragen vom \ac{aem}-Server erlaubt.
\\
Zum anderen können an dieser Stelle noch Änderungen an den angefragten App-Ressourcen erfolgen. Zum Beispiel könnten hier die Templates einer AngularJS-Webanwendung nach relativen Hyperlinks durchsucht und diese angepasst werden.



Ist sie relativ, wird dieser die \serverB am Anfang angefügt. Je nach Anwendung wird auch die \ac{url} vom \ac{aem}, also \serverA, angehängt, zum Beispiel wenn sich die Bilder dort befinden.  Zuletzt wird die \ac{ajax}-Anfrage normal ausgeführt.
\subsubsectiontbd{Aufbereitungsaufgabe bei unterschiedlichen Frameworks}
\label{sec:aufbereitungsaufgabe}
\missingall
\subsubsectiontbd{Konfiguration}
Des weilen wird gewünscht die Komponente zu konfigurieren. Da sich diese nicht direkt im \ac{aem} befindet bedarf es einer Schnittstelle für den Austausch der Konfigurationen zwischen \ac{aem} und dem Webserver.
\missingtext
\subsubsectiontbd{Varianten}
\missingall

\subsectiontbd{Ansatz JavaScript}
\label{sec:ansatz-javascript}
Im Gegensatz zur Java-Variante wird nun der erste Schritt von \autoref{img:java} durch eine JavaScript Variante ersetzt. Das Prinzip ist hierbei wie bei dem Java Ansatz, jedoch gibt es bei dem zweiten Schritt eine Besonderheit. Und zwar ist darauf zu achten, dass die in der HTML-Seite referenzierten App-Ressourcen in der korrekten Reihenfolge geladen werden.

\missingtext