\chapter{Integration der Webframeworks in AEM}
\label{sec:integration-der-webframeworks-in-aem}
In diesem Kapitel werden verschiedene Lösungsansätze beschrieben, ein Webframework in eine mit \ac{aem} erstellte Webpräsenz zu integrieren. Dies umfasst auch die Aufbereitung der Webanwendung vor der Integration und Lösungen für Probleme, welche nur indirekt mit der Integration zusammenhängen. Zum Ende dieses Kapitels erfolgt ein Vergleich der Lösungen mit einem kurzen Ausblick darauf, wie diese sich auch miteinander kombinieren ließen.


\section{Rahmenbedingungen}
Der Versuch der Integration erfolgt in ein \ac{aem} der Version 6.0.0.20140515, ausgeführt unter Windows 10 x64 Version 1607, Build 14393.479 mit der \ac{jre} Version 1.8.0\_111-b14.

\section{Zu untersuchende Frameworks}

Im Folgenden soll der Fokus auf die vier Webframeworks aus \autoref{sec:auswahl-und-bewertung} liegen. Diese sind AngularJS (\autoref{sec:angularjs}),  AngularJS2 (\autoref{sec:angularjs2}), Aurelia (\autoref{sec:aurelia}) und React (\autoref{sec:react}).
%Grund für diese Auswahl ist eine zuvor festgelegte Rahmenbedingung innerhalb dieser Arbeit.

\section{Zu integrierende Webanwendungen}
\label{sec:zu-integrierende-webanwendungen}

Als Basis für den Versuch der Integration wurde für jedes Webframework eine Webanwendung ausgewählt. Diese stammen größtenteils direkt von den Entwicklern des jeweiligen Webframeworks und gebrauchen bereits zahlreiche der jeweils verfügbaren Funktionalitäten. Der Quellcode liegt als eigens eingerichtetes GIT Respository zur Einsicht vor. Eine Übersicht der Webframeworks bietet \autoref{tab:webanwendungen}.

\begin{minipage}{\textwidth}
\begin{longtable}{| c | c | c | c |} 
	\hline
	\thead{Framework} & \thead{Basiert auf} & \thead{GIT Respository} \\ 
	
	\hhline{|=|=|=|=|} 
	AngularJS 1.5.8 & angular-phonecat \cite{Angular2016} &  \cite{Kandler2016a} \\
	\hline
	AngularJS2 2.0.0-rc.3& Tutorial: Tour of Heroes \cite{Google2016e} & \cite{Kandler2016b}\\ 
	\hline
	Aurelia 1.1.0& Quick Start \cite{Eisenberg2017} & \cite{Kandler2016c}\\ 
	\hline
	React 15.2.0 & Redux Tetris \cite{Lugo2016} & \cite{Kandler2016d}\\ 
	
	\hline 
	\caption{Webanwendungen}\label{tab:webanwendungen}
\end{longtable}
\end{minipage}

Im Folgenden wird vermehrt der Begriff \quotes{die Webanwendung} verwendet. Hiermit ist eine beliebige der vier genannten Webanwendungen genannt. Die Aussage \quotes{Es wird versucht eine Webanwendung in das \ac{aem} zu integrieren} bedeutet somit, dass nacheinander versucht wird die vier in \ref{tab:webanwendungen} genannten Webanwendungen in ein \ac{aem} zu integrieren.
%Auch wenn hier der Singular verwendet wird, sind damit alle vier Webanwendungen gemeint. 



\section{Möglichkeiten der Integration}
\label{sec:integrationen}
Abhängig von verschiedenen Faktoren wurden unterschiedliche Möglichkeiten erarbeitet um eine Webanwendung in eine \ac{aem}-Instanz zu integrieren. Neben dem zugrunde liegenden clientseitigen Webframework ist der wichtigste Entscheidungsfaktor dafür, welche Lösung genutzt werden kann, die IT-Landschaft der Zielplatz. Je nach dessen Konfiguration bieten sich gewisse Lösungen mehr an, oder sind auch als mögliche Lösung von vorne herein ausgeschlossen.
%Insbesondere das Thema Zugriffsrechte spielt hier eine immense Rolle. Diese könnten so geregelt sein, dass ein zu tiefer Eingriff in das \ac{aem} und eine freie Konfiguration des Servers nicht gestattet sind.

\subsection{Direkte Integration in das AEM}
\label{sec:direkte-integration-in-das-aem}
Innerhalb dieses Abschnittes sollen Lösungsansätze dargestellt werden, welche die direkte Integration in die \ac{jcr}-Struktur betreffen. Das bedeutet, dass alle benötigten App-Ressourcen in die \ac{jcr} abgelegt werden. Somit sind alle Ressourcen, die für die Ausführung der Webanwendung vonnöten sind, über das \ac{aem} erreichbar.\\
%Sofern der Zugriff auf das Ziel-\ac{aem}, insbesondere auf dessen \ac{jcr}-Strukur, besteht, lässt sich die Webanwendung direkt in selbiges durch die Verwendung von Clientlibs integrieren.\\
%Im ersten Schritt wird das gewünschte JavaScript-Framework als Clientlib, wie in \autoref{sec:clientlib} beschrieben, bereitgestellt. Sollte das Framework mehrmals innerhalb von \ac{aem} Verwendung finden, empfiehlt es sich dies unter \filefolder{/etc/clientlibs} zu platzieren, um global zur Verfügung zu stehen. Ansonsten wird das JavaScript-Framework in die zu entwickelnde \ajc abgelegt. \\
Um die Bereitstellung der App-Ressourcen für die Webanwendung zu ermöglichen, wurden zwei Lösungsansätze erarbeitet. In der ersten Lösung werden alle App-Ressourcen in eine Clientlib gepackt. Die zweite sieht als Lösung vor, diese unter dem Content-Ordner abzulegen.

\subsubsection{Lösungsansatz \quotes{Clientlib}}
\label{sec:sol_clientlib}

Der erste Lösungsansatz beschäftigt sich mit der Realisierung als Clientlib.

%\input{inhalt/integration-der-webframeworks-in-aem/moeglichkeiten-der-integration/direkte-integration-in-das-aem/loesungsungsansatz-clientlib/Transferieren-der-App-Ressourcen-in-das-AEM}
\input{inhalt/integration-der-webframeworks-in-aem/moeglichkeiten-der-integration/direkte-integration-in-das-aem/loesungsungsansatz-clientlib/Anpassen-zur-AJC}
\input{inhalt/integration-der-webframeworks-in-aem/moeglichkeiten-der-integration/direkte-integration-in-das-aem/loesungsungsansatz-clientlib/Bewertung}

\subsubsection{Lösungsansatz \quotes{Content Ordner}}
\label{sec:sol_content}
Eine simple Lösung wäre es, die Webanwendung unter den Content-Ordner  abzulegen. \\
Jeglicher Inhalt, der unter \filefolder{/content} abgelegt wird, ist direkt für den Client aufrufbar. Beispielsweise wäre unter der Standardkonfiguration von \ac{aem} die Datei \filefolder{/content/myapp/index.html} unter der \ac{url} \pseudourl{http://<Server-Hostname>:<AEM-Port>/content/myapp/index.html} erreichbar. Somit wäre es möglich, die Verzeichnisstruktur der Webanwendung 1:1 beizubehalten, sofern die komplette Webanwendung unter \filefolder{/content} abgelegt wird. Dieser Lösungsansatz verzichtet jedoch darauf, die Webanwendung als \ajc zu realisieren, was deren freie Positionierung und Konfigurierung über die Autoren-Bedienoberfläche ausschließt. Der Lösungsansatz \quotes{Content Ordner} empfiehlt sich somit nur für Webanwendungen, welche eine gesamte Webseite des \ac{aem} ausmachen und keine Konfiguration benötigen. 

\input{inhalt/integration-der-webframeworks-in-aem/moeglichkeiten-der-integration/direkte-integration-in-das-aem/loesungsansatz-content-ordner/Bewertung} 


\subsectiontbd{Ansatz Java}
\label{sec:ansatz-java}
Hier ist die Grundidee, dass die Webanwendung sich nicht im \ac{aem} befindet, sondern sich auf einen separaten Webserver liegt. Ziel ist es, dass die Webanwendung autonom auf dem separaten Webserver lauffähig ist und über eine \ac{aem}-Komponente auf Basis von Java und JavaScript in das \ac{aem} integriert wird.\\
Im Folgenden wird der separate Webserver, auf dem sich die App-Ressourcen für die Webanwendung befinden, kurz Webserver, und der Server mit der lauffähigen \ac{aem}-Instanz kurz \ac{aem}-Server genannt. Die Integration erfolgt somit vom Webserver in den \ac{aem}-Server. Die Webanwendung ist über \serverB aufrufbar und soll später über den \ac{aem}-Server über \serverA erreichbar sein.\\

\subsubsectiontbd{Erklärung}

In der Regel erfolgt das Laden einer Webanwendung wie in \autoref{sec:http} beschrieben in drei Schritten. Zunächst wird die HTML-Seite angefordert, anschließend wird der Quellcode der HTML-Seite nach weiteren App-Ressourcen durchsucht und auch diese geladen. Weitere App-Ressourcen werden nun über Ajax angefordert und vom Server retourniert. \\
Es wird davon ausgegangen, dass die Schritte unter \serverB korrekt durchlaufen werden und somit die Webanwendung wie gewünscht im Browser erscheint. \\
Innerhalb von \ac{aem} hat die \ac{aem}-Komponente die Aufgabe die Webanwendung vom Webserver zu laden. Dabei muss jedoch die HTML-Seite, die Ajax-Anfragen und ggf. auch weitere App-Ressourcen manipuliert werden. Die Komponente dient somit als Bindeglied zwischen \ac{aem} und dem Webserver.\\
Zuvor muss jedoch der Webserver und die Webanwendung gewissen Anpassungen unterliegen. Diese werden in \autoref{sec:anpassen-des-webservers} und \autoref{sec:anpassen-der-webanwendung} beschrieben.
\subsubsectiontbd{Anpassen des Webservers}
\label{sec:anpassen-des-webservers}
Gängige Webbrowser verbieten, dass JavaScript, und somit auch \ac{ajax}, auf externe Webserver zugreifen dürfen. Grund dafür ist das Sicherheitskonzept der \ac{sop}. \\
Unterscheiden sich die \ac{url} von der Webseite von der aus die Anfrage gestartet wurde und vom Ziel in Protokoll, Domain oder Port, so wird die Anfrage vom Webbrowser geblockt. Im zuvor beschriebenen Beispiel würden sich \serverA und \serverB im Host (www.example.com anstelle von aem.example.com) und dem Port (3000 anstelle von 80) unterscheiden. \\
Damit der Zugriff doch funktioniert lässt sich der Mechanismus des \ac{cors} verwenden.
\missing{CORS etwas detailierter erklären}
Hierfür antwortet der Zielwebserver der Anfrage mit einen entsprechenden HTTP-Header. Hier sind alle URLs gelistet, von denen gestattet sind eine Anfrage zu auszuführen. Der Browser vergleicht jetzt den HTTP-Header mit der URL, von dem die Anfrage gestartet wurde. Im Erfolgsfall blockiert der Browser nicht und der Aufruf wird zu Ende ausgeführt.
\subsubsectiontbd{Anpassen der Webanwendung}
\label{sec:anpassen-der-webanwendung}
Auch die Webanwendung muss unter Umständen noch vor dem Produktiveinsatz konfiguriert werden. \\
Sollten HTML-Templates in AngularJS Verwendung finden, kann es hier bei dem Ladenvorgang zu einer \ac{sce}-Fehlermeldung führen. Dies geschieht, da der HTML-Code von extern kommt und potenziell unsichereren Code mitführen kann. AngularJS blockt daher per Standardkonfiguration den Ladevorgang. Doch durch entsprechende Anpassungen der Webanwendung lassen sich Ausnahmen hinzufügen, oder das \ac{sce} auch vollständig deaktivieren \cite{Google2016d}.
\missing{Andere Frameworks?}

\subsubsectiontbd{Ablauf der AEM-Komponente}
Den ungefähren Ablauf der \ac{aem}-Komponente ist \autoref{img:java} zu entnehmen.
\begin{figure}[H]
	\begin{center}
		\includegraphics[width=1\textwidth]{servlet.png}
		\caption{Ungefährer Ablauf der AEM-Komponente auf Java-Basis}
		\label{img:java}
	\end{center}
\end{figure}

Der erste Schritt wird im folgenden \autoref{sec:html-seite-aufbereiten}, der zweite Schritt im darauffolgenden \autoref{sec:ajax-anfragen-anpassen} erklärt.

\input{inhalt/integration-der-webframeworks-in-aem/moeglichkeiten-der-integration/ansatz-java/ablauf-der-aem-komponente/HTML-Seite-aufbereiten}
\input{inhalt/integration-der-webframeworks-in-aem/moeglichkeiten-der-integration/ansatz-java/ablauf-der-aem-komponente/Ajax-Anfragen-anpassen}
\subsubsectiontbd{Aufbereitungsaufgabe bei unterschiedlichen Frameworks}
\label{sec:aufbereitungsaufgabe}
\missingall
\subsubsectiontbd{Konfiguration}
Des weilen wird gewünscht die Komponente zu konfigurieren. Da sich diese nicht direkt im \ac{aem} befindet bedarf es einer Schnittstelle für den Austausch der Konfigurationen zwischen \ac{aem} und dem Webserver.
\missingtext
\subsubsectiontbd{Varianten}
\missingall

\subsectiontbd{Ansatz JavaScript}
\label{sec:ansatz-javascript}
Im Gegensatz zur Java-Variante wird nun der erste Schritt von \autoref{img:java} durch eine JavaScript Variante ersetzt. Das Prinzip ist hierbei wie bei dem Java Ansatz, jedoch gibt es bei dem zweiten Schritt eine Besonderheit. Und zwar ist darauf zu achten, dass die in der HTML-Seite referenzierten App-Ressourcen in der korrekten Reihenfolge geladen werden.

\missingtext

\section{Möglichkeiten der Kombination}

Jeder der genannten Lösungen ist generell für sich alleine stehend funktionell und erfüllt richtig angewendet das gewünschte Ziel. Es macht aber auch durchaus Sinn einige Lösungsansätze zu kombinieren.

\subsection{Clientlib und Content-Ordner}
Diese Kombination sieht vor, dass die App-Ressourcen in zwei Gruppen unterteilt werden. Zum einen wären hier alle App-Ressourcen, die bereits in der HTML-Seite referenziert sind, also zu Beginn geladen werden. Diese CSS- und JavaScript-Ressourcen werden als Clientlib realisiert. Zum anderen werden alle weiteren Ressourcen unter \pseudourl{/content} abgelegt. 
Folgendes Beispiel zur Erklärung. Es wird angenommen, dass sich die \ajc unter \pseudourl{http://aem.example.com/content/app} befindet. Über die \ajc wurde bereits die Clientlib und somit die im ersten Schritt benötigten App-Ressourcen geladen. Nun wird über Ajax versucht eine JavaScript-Ressource unter der relativen \ac{url} \pseudourl{./templates/app.html} aufzurufen. Somit ergibt sich die absolute Adresse \pseudourl{http://aem.example.com/content/app/templates/app.html}, was innerhalb des \ac{jcr} die Knoten \pseudourl{/content/app/templates/app.html}. Der genaue Ablageknoten unter \pseudourl{/content} ist somit von der \ac{url} abhängig, unter dem die Webanwendung auffindbar sein soll.

\subsubsection{Bewertung}

Die Lösungskombination ist wie folgt zu bewerten.

\begin{minipage}[t]{0.5\textwidth}
	\textbf{Vorteile:}
	\begin{itemize}
		\item Schnell zu realisieren.
		\item Keinerlei Anpassung der Webanwendung nötig. Relative Pfade werden korrekt aufgelöst.
		\item Die Realisierung erfolgt als AJC, somit lässt sich diese frei innerhalb einer Webseite platzieren.
	\end{itemize}
\end{minipage}
\begin{minipage}[t]{0.5\textwidth}
	\textbf{Nachteile:}
	\begin{itemize}
		\item App-Ressoucen müssen innerhalb eines entsprechenden Knoten unter \pseudourl{/content} liegen. Wird die Anwendung in eine andere Webseite eingebettet, müssen die App-Ressourcen in einen anderen Knoten verschoben werden.
	\end{itemize}
\end{minipage}
\subsection{Proxy und Java}

Der Lösungsansatz \quotes{Java-Servlet} hat den Nachteil, dass für jedes Webframework, oder gar für jede Webanwendung eine eigene \ajc benötigt wird. Denn je nach Art der Webanwendung muss die \ajc diese unterschiedlich vor der Integration bearbeiten.  \\
Weiterhin vom Nachteil ist, dass die Webwendung über den Proxy für Besucher voll zugänglich ist. \\
Um die jeweiligen Vorteile zu vereinen werden beide Lösungsansätze hier vereint. 
\begin{figure}[H]
	\begin{center}
		\includesvg[width=.8\textwidth]{proxy+java}
		\caption{Aufbau der Proxy und Java Kombination}
		\label{img:proxyjava}
	\end{center}
\end{figure}

Hierbei wird die Logik, welche für die Anpassung der Webanwendung benötigt wird, in den Proxy verschoben. Somit genügt eine \ac{aem}-Komponente, in welcher nur die Einstiegsseite der Webanwendung zu konfigurieren ist. Die Dateien \filefolder{AppBuilder.java}, \filefolder{GetData.java} und \filefolder{ajaxredirct.js} werden auch hier eingesetzt, dienen jedoch nur zum Auslösen der HTTP-Anfragen.\\
Für jede Webanwendung lässt sich nun auf dem Server ein eigenes Proxy-Skript erstellen, welches sich z. B. bei Apache durch \quotes{mod\_rewrite} einem entsprechenden \ac{url}-Muster zuordnen lässt. \\
Neben der Modularität und das nur eine \ac{aem}-Komponente Verwendung findet besteht die Kommunikation hier nur zwischen der \ajc und dem Webserver. Der Client kommuniziert nie direkt mit dem Proxy. Somit lässt sich der Webserver konfigurieren, dass nur der \ac{aem}-Server und ggf. einige Test-Nutzer auf diesen Zugriff haben. Allen anderen Nutzern wird der Zugriff verweigert.

\subsubsection{Bewertung}

Die Lösungskombination ist wie folgt zu bewerten.

\begin{minipage}[t]{0.5\textwidth}
	\textbf{Vorteile:}
	\begin{itemize}
		\item Nur eine \ac{aem}-Komponente.
		\item Striktere Trennung. Alle, was die Webanwendung betrifft, auch die Logik der Integration, ist außerhalb von \ac{aem}.
		\item Jeder Proxy kann in einer eigenen Skriptsprache erfolgen.
		\item Ggf. weitere Logik in der AEM-Komponente möglich, wie z. B. eine zentrale Komprimierung der vom Proxy erhaltenen Ressourcen.
	\end{itemize}
\end{minipage}
\begin{minipage}[t]{0.5\textwidth}
	\textbf{Nachteile:}
	\begin{itemize}
		\item Externer Webserver wird benötigt.
	\end{itemize}
\end{minipage}

\section{Aufbereiten der Daten}
\label{sec:aufbereiten}
Vor der Integration ist es gelegentlich ratsam die entwickelte Webanwendung entsprechend aufzubereiten. Dies kann die Performance verbessern und den Prozess der Integration vereinfachen. In diesen Schritt finden Werkzeuge wie Grunt und Gulp aus \autoref{sec:automatiserungswerkzeuge} ihre Verwendung.

\subsection{Minimierung}
Für den Produktiveinsatz ist es ratsam die Quelldateien zu minimieren und konkatenieren. Bei der Minimierung werden zum Beispiel unnötige Leerzeichen entfernt, oder auch lange Variablennamen durch kürzere ersetzt. Bei der Konkatenierung wird die Anzahl der einzelnen Dateien minimiert, indem diese zu einer großen Datei zusammen gefügt werden. Beides erhöht die Übertragungszeit vom Server zum Client, da weniger Anfragen gestellt werden müssen, und auch das zu übertragende Volumen sinkt. Weiterhin existieren so auch weniger Dateien, die es in das \ac{aem} zu integrieren gilt, was zum Beispiel die direkte Integration wie in \autoref{sec:direkte-integration-in-das-aem} beschrieben erleichtert. \\
Weiterhin lassen sich bei manchen Frameworks die HTML-Templates in JavaScript-Dateien umwandeln. Somit ist auch hier eine anschließende Minimierung und Komprimierung möglich.


\begin{description}
	\item[Allgemein] Generell existieren für die Minimierung von JavaScript-Quellcode verschiedenste Werkzeuge. Eines davon wäre UglifyJS \cite{Bazon2016}. Hiermit lässt sich JavaScript-Quellcode unter anderem minimieren und konkatenieren. Das Werkzeug wird unter der BSD-Lizenz veröffentlicht und wird über \ac{npm} installiert. Bei der Konkatenierung ist darauf zu achten, die JavaScript-Dateien in der korrekten Reihenfolge anzugeben, um Fehler in der Abhängigkeit zu vermeiden. \\
	Um solche Abhängigkeitsfehler zu vermeiden lässt sich hier das Konzept der \ac{di} nutzen. Wurde die Webanwendung unter Verwendung von \ac{di} entwickelt, so ist in den JavaScript-Datei bereits deren Abhängigkeit syntaktisch hinterlegt. Je nach Art der verwendeten \ac{di} darf der Entwickler auf unterschiedliche Werkzeuge zurückgreifen. Alle gängigen Bibliotheken aus \autoref{tab:dilist} liefern bereits Werkzeuge für die Konkatenierung mit.

	\item[AngularJS] Alle AngularJS-Templates, welche sich im Normalfall in einer jeweils eigenen HTML-Datei befinden, lassen sich in eine eigene JavaScript-Datei packen. Als Werkzeug wird hier Grunt in Verbindung mit dem \ac{npm}-Packet Namens grunt-angular-template \cite{Clemmons2016} verwendet. Dies nimmt die AngularJS-Templates und wandelt diese in eine JavaScript-Datei um, welche den Template-Cache \cite{Google2016c} von AngularJS nutzt.
	
	\item[AngularJS2] In der Dokumentation von AngularJS2 wird der \ac{aot}-Ansatz erklärt \cite{Google2016a}. Im Gegensatz zum \ac{jit}-Ansatz, bei dem Skripte und App-Ressourcen erst dann geladen werden, wenn sie benötigt sind, wird hier alles, was später benötigt werden könnte, zum Anfang geladen.
	
	\item[Aurelia] Auch Aurelia bietet ein derartiges Tool an. Dieses nutzt Gulp und ist über \ac{npm} mit den Namen aurelia-bundler \cite{Aurelia2016} zu finden.
	
	\item[React] Um die Laufzeit der Webanwendung zu optimieren, kann man diesen zuvor in normale JavaScript-Anweisung umwandeln. Hierfür empfiehlt sich der Einsatz von Babel \cite{Babel2016}. Das Werkzeug wird über \ac{npm} installiert, kann TypeScript und neueres ECMAScript in ECMAscript 3 und 5 umwandelt, und beherrscht auch \ac{jsx}. Die Umwandlung lässt sich manuell, oder auch wahlweise durch Gulp und Grunt anstoßen.
	
	
	
\end{description}

\section{Bekannte Konflikte zwischen Skripten}
\label{sec:konflikte}
Während der Integration sind keinerlei Problematiken zwischen den Skripten von \ac{aem} und Skripten seitens der Webanwendung bzw. eines Webframeworks aufgetreten. \\
Auch beim Einsatz von unterschiedlichen Versionen des gleichen Webframeworks wurden keinerlei Probleme entdeckt.

\input{inhalt/integration-der-webframeworks-in-aem/Bewertung-der-Lösungen}

