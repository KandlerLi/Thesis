\section{Motivation}

Um moderne Webseiten einer Webpräsenz zu erstellen und zu pflegen, arbeiten meist mehrere Personen zusammen, die diese optisch, inhaltlich und funktionell gestalten. Dabei hat jeder unterschiedliche technische Kenntnisse. Um auch Personen, die keinerlei Programmierkenntnisse besitzen, die Möglichkeit zu geben, eine Webpräsenz zu erstellen, wurden diverse \acfp{cms} entwickelt. Ein CMS\index{CMS} besitzt zumeist eine grafische Bedienoberfläche, welche auch weniger technikaffinen Menschen das Erstellen von Inhalten auf einer Webpräsenz ermöglicht. Diese Inhalte sind meist statisch. Änderungen, die Aussehen und Inhalt betreffen, müssen manuell von einer Person erbracht werden. \\
Viele Unternehmen sind derzeit dabei, ihre Webpräsenz zu reorganisieren. Hierbei fließt neben dem zuvor erwähnten statischen Inhalt oft auch von unterschiedlichen Portalen dynamisch generierter Inhalt mit ein. Dies können z. B. Konfigurationen von Autoherstellern oder Antragsformulare von Banken sein. Diese dynamischen Inhalte werden häufig von sogenannten Microservices generiert. Microservices sind Architekturmuster, welche aus mehreren, voneinander unabhängigen Prozessen bestehen und kombiniert die benötigten Inhalte erzeugen. \\
Eine Lösung, diese dynamischen, von Microservices generierten Inhalte in Webseiten einzubetten, ist es, die komplette Webseite neu zu laden und die erstellten Inhalte auf der Serverseite zu integrieren, wie später in \autoref{sec:server-webanwendung} beschrieben. Hierfür ließe sich das verwendete \ac{cms} nutzen, da dieses zumeist auf serverseitigen Techniken und Programmiersprachen basiert. Jedoch müssten bei dieser Lösung auch die Inhalte, welche unverändert sind, vom Server erneut geladen werden, was zu einem unnötigen Mehraufwand führt, da die komplette Webseite, also auch die unveränderten Teile, vom Server an den Client gesendet werden müssen. Um ein erneutes Laden der Webseite und somit diesen Mehraufwand zu vermeiden und das Einbetten der dynamisch nachgeladenen Inhalte aus dem Portalumfeld in eine Webseite zu beschleunigen, sollen Lösungen gefunden werden, wie sich Webanwendungen, die mit einem Webframework erstellt wurden, integrieren lassen.