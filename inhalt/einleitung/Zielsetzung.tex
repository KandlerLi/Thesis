\section{Zielsetzung}

Als Ziel dieser Arbeit sollen Erkenntnisse gewonnen werden, wie eine Webanwendung, die mit einem clientseitigen Webframework erstellt wurde, sich am Besten in den \ac{aem} (\autoref{sec:aem}) integrieren lässt, um dynamische Inhalte aus dem Portalumfeld nachladen zu können. Der Fokus dieser Arbeit liegt hierbei auf den Webframeworks AngularJS (\autoref{sec:angularjs}),  AngularJS2 (\autoref{sec:angularjs2}), Aurelia (\autoref{sec:aurelia}) und React (\autoref{sec:react}).\\
Dabei ist zu beachten, dass der \ac{aem}-Server unterschiedliche Konfigurationen aufweisen kann. Daher werden mehrere Lösungsansätze für die Integration, angepasst an die jeweilige Konfiguration, benötigt. \\
Weitere wichtige Aspekte für die Integration wären hier die Anpassung der Webanwendung und des \ac{aem}, damit diese von Suchmaschinen korrekt indiziert werden, und das Sicherstellen, dass Hyperlinks zum richtigen Pfad führen. \\




%Zudem soll darauf geachtet werden, wie einfach sich die Webframeworks in ein CMS integrieren lassen. Weitere, noch zu klärende Kriterien könnten sein die Performance, um die Ladezeit auf Clientseite zu minimieren, die Erweiterbarkeit des Frameworks, die Testbarkeit, also ob es z. B. die Möglichkeit gibt, Unit-Tests anzuwenden und eine Einschätzung darüber, ob die Entwickler des Frameworks auch in Zukunft ihr Framework pflegen werden.\\
