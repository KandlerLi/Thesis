\section{Aufgaben}
Damit die zuvor genannten Ziele erreicht werden können, wird zunächst der Ist-Zustand untersucht, also wie bis jetzt eine \ac{aem}-Instanz erweitert wurde. So werden bereits etwaige erste Problematiken erkannt.\\
Anschließend wird der Versuch unternommen jeweils eine Webanwendung der vier zuvor genannten clientseitigen Webframeworks zu integrieren. Dabei ist es zielführend, Webanwendungen zu verwenden, die möglichst zahlreiche Funktionalitäten des jeweiligen Webframeworks nutzen. Der Integrationsversuch wird abschließend mit verschiedenen Konfigurationen der Webanwendung und des \ac{aem} wiederholt. Somit lassen sich weitere Schwierigkeiten mit gewissen Konfigurationen entdecken. \\
Für die gefundenen Probleme gilt es nun, Lösungsansätze zu erarbeiten und diese innerhalb dieser Arbeit zu erläutern. Sollte unter einer Konfiguration die Integration nicht gelungen sein, so wird für diese eine alternative Herangehensweise benötigt.\\
Schlussendlich werden alle Formen der Integration und Lösungsansätze gesammelt, dokumentiert und weitestgehend gegenübergestellt. Ziel soll es dabei nicht sein innerhalb dieser Arbeit die Ergebnisse der Integration zu dokumentieren, sondern das dabei gewonnene Wissen, nämlich wie sich Probleme umgehen lassen und welche Lösungsansätze wann zu empfehlen wären.