\section{Aufgaben}
Damit die zuvor genannten Ziele erreicht werden können, wird zunächst der Ist-Zustand untersucht, also wie bis jetzt eine \ac{aem}-Instanz erweitert wurde. So werden bereits etwaige erste Problematiken erkannt.\\
Anschließend wird der Versuch unternommen jeweils eine Webanwendung der vier zuvor genannten clientseitigen Webframeworks zu integrieren. Dabei ist es zielführend, Webanwendungen zu verwenden, die möglichst zahlreiche Funktionalitäten des jeweiligen Webframeworks nutzen. Der Integrationsversuch wird abschließend mit verschiedenen Konfigurationen der Webanwendung und des \ac{aem} wiederholt. Somit lassen sich weitere Schwierigkeiten mit gewissen Konfigurationen entdecken. \\
Für die gefundenen Probleme gilt es nun, Lösungsansätze zu erarbeiten und diese innerhalb dieser Arbeit zu erläutern. Sollte unter einer Konfiguration die Integration nicht gelungen sein, so wird für diese eine alternative Herangehensweise benötigt.\\
Schlussendlich werden alle Formen der Integration und Lösungsansätze gesammelt, dokumentiert und weitestgehend gegenübergestellt. Ziel soll es dabei nicht sein innerhalb dieser Arbeit die Ergebnisse der Integration zu dokumentieren, sondern das dabei gewonnene Wissen, nämlich wie sich Probleme umgehen lassen und welche Lösungsansätze wann zu empfehlen wären.

%Damit die zuvor genannten Ziele erreicht werden können, soll zunächst eine Anforderungsliste entworfen werden. Diese soll beschreiben, welche Anforderung und Eigenschaften ein Webframework haben muss bzw. optional haben sollte. Dabei sollte jede Anforderung, soweit möglich, einen messbaren Wert sowie eine Einstufung der Bedeutung, also ob es sich bei der Anforderung um ein notwendiges Feature oder um ein nice-to-have Feature handelt. \\
%Daran anschließend soll eine Auswahl an Frameworks getroffen werden. Als mögliche Kandidaten wären hier neben vielen anderen AngularJS und React zu nennen. In Form einer Kriterienliste wird diese Auswahl anhand der Anforderungliste bewertet und nicht geeignete Frameworks werden ggf. ausgesondert. \\
%Um die Tragfähigkeit der gefilterten Auswahl zu überprüfen, sollen die geeignetsten Frameworks in das AEM integriert werden, um ihre Funktionalität zu überprüfen und somit weitere Erkenntnisse über die Praxistauglichkeit zu erlagen. \\
%Anhand der gewonnenen Erfahrungen gilt es schlussendlich ein Fazit und eine finale Beurteilung der Frameworks zu erstellen.


